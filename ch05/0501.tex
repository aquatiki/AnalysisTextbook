%!TEX root =  ../main.tex


\objective{Ability to simplify rational exponents and complex fractions}

Fractions are often shunned by students, who state they prefer decimals.  But decimals really
are fractions to, only in tenths, hundredths, thousandth, etc.  Because 10 has so few divisors,
decimals can actually obscure factors and other structures that would simplify a problem.  
Being comfortable --- even conversant --- with fractions, can make mathematics much more
enjoyable, or at least, less difficult.

\personfeature[0.5in]{\chapdir/pics/Ada_Lovelace_Chalon_portrait.jpg}{Ada Lovelace}{1815-1852}{was an English mathematician and writer, chiefly known as the first to recognize that computation machines had applications beyond pure calculation, and created the first algorithm intended to be carried out by such a machine. As a result, she is often regarded as seeing the full potential of a "computing machine" and the first computer programmer. \href{https://en.wikipedia.org/wiki/Ada_Lovelace}{Wikipedia}}

\subsection{Clear the Fraction}
First, fractions within fractions are often quite unclear and awkward.  It may be tempting to
find a common denominator amongst the terms on the top-half, so that they can be
added together, and then do the same on the bottom.  However, you will still have a four-part
fraction by the end of that process.  In and of itself, this isn't so bad, since you will simply 
flip the lower fraction upside-down and multiply against upper fraction.  Is there a shorter way?

\begin{example}
	\exProblem
Simplify the top and bottom half separately, then the whole: $\cfrac{5}{\frac{25}{4}+\frac{5}{2}}$.

\exSolution
The common denominator of the terms in the denominator is 4, so only the fraction in the
``southeast'' needs alteration: $\cfrac{5}{\frac{25}{4}+\frac{5}{2}\cdot\frac{2}{2}} = 
\cfrac{5}{\frac{25}{4}+\frac{10}{4}} = \cfrac{5}{\frac{35}{4}} = \frac{5}{1}\cdot\frac{4}{35}
=\frac{4}{7}$.
\end{example}

Before we answer the question of complex expressions, let us consider what may be
gleaned from equations involving fractions:

\begin{equation}
\frac{x}{5}+\frac{2}{3}=\frac{11}{6}
\end{equation}

``How many fifths must be added to two-thirds to get eleven-sixths?''  As before, we might might
be a common denominator (30), subtract and then divide.  Of course, your aversion to fractions
might serve you well here, and lead you do solve this problem by \emph{clearing the fraction}.
This process consists of nothing more than ascertaining the Lowest Common Denominator (LCD)
and then multiplying both side of the equation by it:

\begin{align*}
30\cdot\left[\frac{x}{5}+\frac{2}{3}\right] &= \frac{11}{6}\cdot30\\
6x+20 &= 55\\
x &=\frac{35}{6}
\end{align*}

Everyone agrees this is more painless way to approach the problem.  Now, back to our
complex fraction.  Is there an analogous situation going on there?  Yes, you consider the
process linearly.  The fractions on the top half will eventually interface with the fractions on 
the bottom.  It is best to begin treat the top and bottom denominators as related as soon
as possible, and multiply the top and bottom by the LCD.  This is permissible because 
multiplying by a fraction equal one does not change the expression at all.

$$\cfrac{5}{\left(\frac{25}{4}+\frac{5}{2}\right)}\cdot\cfrac{4}{4}=\frac{20}{25+20}=\frac{4}{7}$$

The highest benefit of this approach is that it immediately transforms a complex fraction 
into an ordinary one.  In the exercises, you will simplify such expressions with variables.

\personfeature[-1in]{\chapdir/pics/Katherine_Johnson_in_2008}{Katherine Johnson}{1918-}{is
a is an African-American physicist and mathematician who made contributions to the United States' aeronautics and space programs with the early application of digital electronic computers at NASA,
who was known for her accuracy. \href{https://en.wikipedia.org/wiki/Katherine_Johnson}{Wikipedia}}


\subsection{In Equations}
In equations, it is important to not lose information as you simplify.  As we saw in chapter 2,
expressions such as these can be made plainer, but there are ``holes'' which at smoothed 
over in such simplifications.  You may rewrite such equations in forms that are easier to
deal with, but you must record what the original excluded values were.

\begin{example}
\exProblem
Solve for real solutions of $x$: $\frac{1}{x-6}+\frac{x}{x-2}=\frac{4}{x^2-8x+12}$.

\exSolution
Factoring the quadratic denominator yields $(x-2)(x-6)$, so that is the LCD.  Multiplying
both side of the equation produces $(x-2) + x(x-6)=4$.  Expanding combining like-terms
reveals another quadratic, $x^2-5x-6=0$, whose solutions are 1 and -6. ... Or are they?
Looking back at the original problem, 6 is an excluded value, making one denominator
zero, and hence undefined.
\end{example}

\index{TI-8*!programming}
