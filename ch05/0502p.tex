%!TEX root =  ../main.tex
\renewcommand{\columnseprule}{1.5pt}
\begin{multicols*}{2}
\rule[0.5\baselineskip]{0.4\textwidth}{1pt}
\noindent
\LabSection{The Power of Powers}\label{sec:0502p}
\begin{exercises}{sec:0502p}
\lab{} 1  Let us build up the Rules of Power, so that they make sense.  (Never use a rule you cannot visualize and explain!)  What does $x^4$ mean?  How can you rewrite it in another (longer) way?

\vspace{2cm}
\lab{}  Expand inside each parentheses  $(x^4)(x^5)$.

\vspace{2cm}
\lab{}  How many $x$’s did you write in total?

\vspace{1cm}
\lab{}  What rule can we deduce about exponents on the same base being multiplied?

\vspace{2cm}
\lab{}  Expand inside the parenthesis $\frac{(x^5)}{(x^3)}$.

\vspace{2cm}
\lab{}  Cancel an equal number of $x$’s in the numerator and the denominator.  How many are left in the numerator?  What value can $x$ not be?

\vspace{2cm}
\lab{} Write a rule for simplifying the same base with different exponents in the numerator and denominator.

\vspace{2cm}
\lab{}  What will happen if there are more $x$’s in the numerator?  For example, $\frac{x^3}{x^5}$ can be rewritten as what power on $x$?

\vspace{2cm}
\lab{} How else can you rewrite $\frac{x^3}{x^5}$?  1 over $x$ to the what?

\vspace{2cm}
\lab{} Write a rule for how to handle negative exponents.

\vspace{2cm}
\lab{} What if the exponents had been equal?  What must $x^0$ mean/be?

\vspace{2cm}
\lab{} How else can we write $\frac{1}{x^{-2}}$?

\vspace{2cm}
\lab{} Expand $(x^2)^3$ into sets of parentheses being multiplied.  How many set of $x$’s are there and how many are in each?

\vspace{2cm}
\lab{} What is the total number of $x$’s?

\vspace{1cm}
\lab{} Summarize how to simplify a power to a power.

\vspace{2cm}
\lab{} Describe in technical vocabulary what you think the point of this problem set it, including your three rules and the two cautions (i.e. negatives and zeros).



\end{exercises}
\end{multicols*}