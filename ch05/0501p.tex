%!TEX root =  ../main.tex
\renewcommand{\columnseprule}{1.5pt}
\begin{multicols*}{2}
\rule[0.5\baselineskip]{0.4\textwidth}{1pt}
\noindent
\LabSection{Work Smarter, Not Harder}\label{sec:0501p}
\begin{exercises}{sec:0501p}
\lab{} Over the phone, your young friend tells you all her homework problems are the same format: one over something plus one over something all over one over something plus one of something.  What does she mean in conventional math notation?

\vspace{2cm}
\lab{} You decide to write a computer program to solve all the problems quickly, and make sure she got the right answers by hand.  How many variables do you need per question?

\vspace{2cm}
\lab{} Start a new program in your TI-8* by pressing PRGM and moving over to NEW.  Select \texttt{Create New}.  What should you title your program to find it easily later?  (Consider using 2ND-ALPHA to be able to type a string of letters more easily.)

\vspace{2cm}
\lab{} For your first line of code, you should tell the user what is about to happen.  Press PRGM, move over to I/O, and choose 3: DISP.  To say some text (and not use letters as variables), surround the text with quotation marks.  Where are is that symbol on the TI-8*?

\vspace{2cm}
\lab{} After you have declared your program’s intentions, we need to ask for those variables.  Press PRGM, then I/O, and select PROMPT followed by the first variable name you chose.  Press enter, and create a prompt for each variable you need.

\vspace{2cm}
\lab{} Enter the formula as you wrote it in problem 1.  We will want to answer as a fraction, append your formula with the conversion tool that does so.  Where is ``convert to fraction'' on your TI-8*?

\vspace{2cm}
\lab{} Test your program by quitting back to the main screen and EXECuting (running) your program.  Did it work?

\vspace{2cm}
\lab{} Quickly run through example problems, using 2nd-ENTER to re-execute the program after it finishes.  Record several examples here.

\vspace{2cm}
\lab{} Briefly chart out how you think a program to compute instances of the quadratic formula might flow.  What would you need to know in order to make such a program work?

\vspace{2cm}
\lab{} Describe a hypothetical program that might save you time vis-à-vis a computation you often retype out each time

\vspace{2cm}
\lab{} In your own words, describe what you think the point of this problem set is.

\end{exercises}
\end{multicols*}