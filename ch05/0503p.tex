%!TEX root =  ../main.tex
\renewcommand{\columnseprule}{1.5pt}
\begin{multicols*}{2}
\rule[0.5\baselineskip]{0.4\textwidth}{1pt}
\noindent
\LabSection{With Great Power}\label{sec:0503p}
\begin{exercises}{sec:0503p}
\lab{} What is the normal thing that happens when we square a number: does it get bigger or smaller?

\vspace{2cm}
\lab{} Square 0.1.  What happened?

\vspace{2cm}
\lab{} What is the normal thing that happens when we square root a number: does it get bigger or smaller?


\vspace{2cm}
\lab{} What is the cube root of 0.64?  What happened?

\vspace{2cm}
\lab{} Let us reconsider the problems we have been solving as powers.  On what interval is $x^2 < x$?  

\vspace{2cm}
\lab{} Did you answer include negatives?  Rewrite your answer using absolute values.

\vspace{3cm}
\lab{} Solve for the interval where $\sqrt[3]{x} > x$.  Write your answer using absolute values.

\vspace{2cm}
\lab{} Let us consider how to graph any power function $y= x^{m/n}$.  Begin by thinking about $m$.  Any number to an even power has what sign?

\vspace{2cm}
\lab{} Next, consider $n$.  What are even roots of negative numbers?

\vspace{2cm}
\lab{} Suppose we had $f(x)=x^{\frac{5}{3}}$.  How could we find a lattice point?  Roots are harder than power, so what is the easiest number after 1 to cube root?  What is that number to the fifth?  Write your answer as two lattice point on the graph of $f(x)$, one in the first quadrant, one in the fourth.

\vspace{2cm}
\lab{} Are you two point ``over'' more than they are ``up''?  In other words, how will $|f(x)|$ compare to $y=|x|$?

\vspace{2cm}
\lab{} Create a list of criteria to evaluate the graph of any power function, if you had to do so by hand.

\vspace{2cm}
\lab{} Describe what you think the point of this problem set is, in technical language.


\end{exercises}
\end{multicols*}