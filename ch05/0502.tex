%!TEX root =  ../main.tex

\objective{Solve arbitrary equations involving rational powers.}


Fractional exponents are nothing too special.  It is simply important to remember that in the
fraction, the numerator is what we've been thinking of as powers ($\frac{5}{1}=5$) and the
denominators are roots, i.e. $2^{\frac{1}{2}}=\sqrt{2}$.  As we will see next section, this isn't 
a watertight definition, but it will do for now.

\subsection{Multiplying}\index{Exponents!rules of}
In situations where exponents abound, it may become confusing when one can and cannot
multiply, and what to do with exponents.  What do we mean when we write $5^2\cdot5^3$? 
Is that $5^5$ or $5^6$?  Expanding the notation should help clear it up.  $5^2$ means 
``multiply by five twice'', i.e. $5\cdot 5$.  $5^3$ means ``multiply by five three times'', $5\cdot
5\cdot 5$.  So we can see that $5^2\cdot5^3=5\cdot 5\cdot 5\cdot 5\cdot 5=5^5$.



\begin{derivation}{Multiplying with the Same Base}
$b^m\cdot b^n=b^{m+n}$.  Notice that $b$ must be consistent.
\end{derivation}


Students sometimes want to combine in impossible ways.  $2\cdot5^3$ becomes $10^3$ 
somehow, in their minds.  When we consider what the notation means, the contradiction
becomes clear.  $2\cdot5^3$ means ``two times this: five-times-five-times-five'', which is in no
way the same as ``ten times ten times ten''.

\subsection{Dividing}
If the exponents can be added when multiplying powers of the same base, what would you
expect when dividing?  Yes, it is subtraction.


\begin{derivation}{Dividing with the Same Base}
$\cfrac{b^m}{b^n}=b^{m-n}$.  Notice that $b$ must be consistent.
\end{derivation}


This is a good explanation for negative exponents.  For example, $\cfrac{2^3}{2^8}=2^{-5}$.
There are more two's in the denominator than the numerator.  This is the same as $\frac{1}{2^5}$,
which is a far more useful way to write the fraction.  When cancelling and simplifying are done,
it is conventional to expand the exponent, in this case writing $\frac{1}{32}$.

This also explains the origin of $b^0=1$, unless $b=0$.  Zero exponent arrises when there
are as many of the base in the denominator as there are in the numerator.  Anything
divided by itself is 1.

\subsection{Exponents}
What happens when there is an exponent on an exponent?  Easier than a dream within a
dream, the exponents continue to mean what they have always meant: ``have this many
of this bases be multiplied against themselves''.  For example, $(2^3)^4$ means ``four
groups of two-times-two-times-two'', or $2^12$.

\begin{derivation}{Distribution of Exponents over Multiplication}
$(b^m\cdot c^n)^p=b^{m\cdot p}\cdot c^{n\cdot p}$ and so for, on each element under
the power
\end{derivation}


\begin{example}
\exProblem
Solve $x^{\frac{3}{2}}=27$.

\exSolution
To get $x$ to have a simple exponent of 1, raise both sides to the two-third.\\
$\left(x^\frac{3}{2}\right)^{\frac{2}{3}}=27^{\frac{2}{3}}$
$x=\left(\sqrt[3]{27}\right)^2$\\
$x=3^2=9$
\end{example}

Notice too, there is no distributive property of exponents over addition.  $\sqrt{x^2+1}$
is irreducible, not $x+1$.  We would need to know what $x$ was to be able to proceed.

\subsection{Rational Exponents}\index{exponents!rational}
Fractional exponents must be the same as roots.  For example, we know from
the multiplication property that $4^\frac{1}{2}\cdot 4^\frac{1}{2}$ must equal $4^1$.
That must mean we are looking for a number times itself to equal 4, so $4^\frac{1}{2}$
must equal 2.  This means $\sqrt{4}=4^\frac{1}{2}$.  By the same logic, we would 
find that $4^\frac{1}{3}=\sqrt[3]{4}$, $4^\frac{1}{4}=\sqrt[4]{4}$, etc.  By the power
rule above, $4^\frac{3}{2}$ must be the same as $(4^\frac{1}{2})^3$.  This could also
be written as $4^{1.5}$.

We must be careful with even rational exponents, because they can hide the sign of the base.
For example, $((-5)^2)^\frac{1}{2}=5$.  The safe answer then to $\sqrt[n]{x^n}$ for any
even $n$ is $|x|$.
