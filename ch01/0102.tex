%!TEX root =  ../main.tex


\objective{Connect names of types of equations with graphs.}


We have said that a function is a relation of inputs to outputs, with no more than one output per input.  
We have learned that functions can be defined graphically, algebraically, numerical, or verbally.
A function that is defined by an algebra equation usually has a descriptive name.
In this section, we will look at several groupings of various functions, some of which you should
be very familiar with, and some of which may be new.


You might be used to seeing a  \gls{graph} of certain equations, written algebraically with $y$ in terms
of $x$, like $y=x$.  These have then been graphed on the Cartesian plane, as a continuous curve,
with each point corresponding to an ordered pair, $(x,y)$.  Leonhard Euler created much of the notation
we use today, including \emph{function notation}: $y=f(x)$.

\marginfig[-0in]{\chapdir/pics/Linear_Function_Graph.png}{Linear functions.\label{fig:LinearFunction}}
\subsubsection{Linear}
One of the most obvious things to see is a straight line.  
Humans create straight lines seemingly more often
than anything else.  Hence, lines can feel unnatural or reassuring.  
What are some of the properties of lines?
How might two lines be the same?  How might they differ?  

\marginfig[-0in]{\chapdir/pics/AndraGrad-4}{A quadratic function is a polynomial of degree two.}
Lines could have the same slope, and therefore never run into each other.  
They would only be distinguished by their
heights.  For convenience, we measure the height of a \gls{linear} function 
in Analytic Geometry by its starting value, its $y$ when $x$ 
equals zero.  Conversely, these starting locations might be the same and 
slope might be different.  You have learned \index{linear!intercept form}
to distinguish these two different variables as $m$ and $b$, as in $y=mx+b$, and we shall see
that it is expedient to distinguish them \textbf{constant} functions, $y=k$.
\index{constant!function}

\subsubsection{Quadratic}
Many things in our world operate over two dimensions, such as gravity.  
Hence, Newton found that the force
of two objects upon each other is proportional to the \emph{square} of their distance.  
Squares graphed make a
\textbf{parabola}, a word that in Greek references the path of a falling or thrown object.  
Algebraically, we can see all such shapes\index{quadratic!standard form}
have an equation of the form $y=ax^2+bx+c$, which is called \gls{quadratic}.  
You should already know a great deal about quadratics from previous classes.

\paragraph{Power}
As more dimension interact, the exponent on $x$ can become very 
complicated, and even fractional.  We can generalize\index{power function!standard form}
from $y=x$ to $y=x^2$ and $y=x^3$ to $y=a\cdot x^b$.  We shall study them in more depth
in chapter 5.

\paragraph{Polynomial}
A sum of power functions with whole number exponents is called a
\textbf{polynomial}.  Such equations are among the most well-studied
areas in mathematics.  A polynomial divided by a polynomial is called
a \textbf{rational function}.  Both are the subject of chapter 6.
\index{polynomial}\index{rational function}

\subsubsection{Exponential}
\marginfig[-0in]{\chapdir/pics/2^x_function_graph}{$2^x$ is an  exponential function}
Quantities that experience the same percentage growth or decay 
year over year look similar in the algebra: $x$ is in
the exponent, and hence such an equation is called an \gls{exponential} function.  The general form
is $y=a\cdot b^x$.  \index{exponential function!standard form}
The ``opposite''\footnote{There are \emph{many} things 
which could be called `opposite'
in mathematics, so this is not technical language.  We will define `inverses' of functions in 4.4.} of such a 
function is a called a \gls{logarithm}, and we will follow the TI-8* for now and use the generic equation
$y=a+b\ln{x}$.  \index{logarithmic function}

\subsubsection{Periodic}
Many phenomena in nature reoccur the same way at regular intervals.  
Such functions are said to be \textbf{periodic}.
We will  study `simple harmonic motion,' which comes from components of motion in circles, in section III,
Trigonometric Functions.  For now\footnote{Later, we
will factor the ``inside,'' but this first kind is the sort produced by your grapher.}, use the general equation $y=a\cdot\sin(bx+c)+d$.\index{sine function}

\inlinefig{\chapdir/pics/Periodic_function_illustration.png}{\label{fig:periodic}A periodic function is so called because it repeats at a given interval, called the period (P).}
