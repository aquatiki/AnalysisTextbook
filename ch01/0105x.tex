 % These are the 1-5

\begin{defproblem}{0105:simplequadA}
\begin{onlyproblem}
Find a quadratic model for the given data.
\footnote{CA textbook 13-16}
\begin{enumerate}
\item \{(1,-1), (2,1), (4,8), (5,14), (6,25)\}
\item \{(1.5,-2.2), (2.2,-4.8), (3.4), (-11,2), (5.1,-20.6)\}
\end{enumerate}%
\end{onlyproblem}%
\begin{onlysolution}%
\begin{enumerate}
\item $y\approx1.09578x^2-2.69643x+1.13637$
\item $y\approx-1.48736x^2+5.86598x-8.11229$
\end{enumerate}
\end{onlysolution}%
\end{defproblem}

\begin{defproblem}{0105:simplequadB}
\begin{onlyproblem}
\begin{enumerate}
\item \{(-2,-5), (-1,0), (0,1), (1,4), (2,4)\}
\item \{(1.5,-2.2), (2.2,-4.8), (3.4), (-11,2), (5.1,-20.6)\}
\end{enumerate}%
\end{onlyproblem}%
\begin{onlysolution}%
\begin{enumerate}
\item $y\approx-0.57142x^2+2.2x+1.94286$
\item $y\approx-1.48736x^2+5.86598x-8.11229$
 \end{enumerate}
\end{onlysolution}%
\end{defproblem}

\begin{defproblem}{0105:archeology}
\begin{onlyproblem}
\paragraph{Archeology}
The data below shoe the length of the
humerus and the total wingspan, in
cm, of several pterosaurs, extinct
flying reptiles.
\footnote{CA textbook 17}
\begin{enumerate}
\item Compute a linear regression for these data
\item On the basis of this model, which is the projected
wingspan of a \textit{Quetzalcoatlus northropi}, which is 
estimated to have has a humerus of 54 cm?  Round
to the nearest centimeter.
\end{enumerate}
\end{onlyproblem}
\begin{onlysolution}
\begin{enumerate}
\item $23.557x -24.427$
\item 1248 cm
\end{enumerate}
\end{onlysolution}
\end{defproblem}


\begin{defproblem}{0105:ModelDaylight}
\begin{onlyproblem}
\paragraph{Model of the Daylight Hours}
For a particular day of the year $t$, the number of 
daylight hours in New Orleans can be approximated
by
$$
d(t)=1.792\sin\left(\dfrac{2\pi(t-80)}{365}\right) + 12.145
$$
where $t$ is an integer and $t=1$ corresponds to
January 1.  
\begin{enumerate}
\item Accord to $d$, how many days per year will
New Orleans have at least 10.75 hours of daylight?
\end{enumerate}
\end{onlyproblem}
\begin{onlysolution}
From day 28 to 314 is 286 days
\end{onlysolution}
\end{defproblem}

\begin{defproblem}{0105:hourly}
\begin{onlyproblem}
\paragraph{Hourly Earnings}
The average hourly earnings of U.S. production
workers for the 1990's are show in the table below.
\footnote{Demana Precalc p163}
\begin{enumerate}
\item Produce a scatter plot of the hourly earnings ($y$) as function
of the years since 19990 ($x$).
\item Run the linear regression and store it in $Y_1$.
\item Record the $r^2$ value.  Does it suggest that this model is appropriate?
\item Run the quadratic regression and store it in $Y_2$.
\item Record the $r^2$ value.  Does it suggest that this model is appropriate?
\item Find the difference between the two models' predictions for the average hourly earnings in 2010.
\item Write a four sentence paragraph describing why it can be risky to extrapolate from a mathematical
model, citing your work here in technical language.
\end{enumerate}
\begin{tabular}{c|c}
	\textbf{Year} & \textbf{Average} \\ \hline
	1990 & 10.19\\
	1991 & 10.50\\
	1992 & 10.75\\
	1993 & 11.03\\
	1994 & 11.32\\
	1995 & 11.64\\
	1996 & 12.03\\
	1997 & 12.49\\
	1998 & 13.00\\
	1999 & 13.47\\
	2000 & 14.00\\
	2001 & 14.95\\
	2002 & 14.95\\
	2003 & 15.35\\
\end{tabular}
\end{onlyproblem}
\begin{onlysolution}
\begin{enumerate}
\item insert graphic
\item $y\approx.4089x+9.8601$
\item 98.6\%.  It would seem so...
\item $0.0124x^2+.2473x+10.1241$
\item 99.8\%.  Yes, more so than the linear.
\item Individual results will vary.
\end{enumerate}
\end{onlysolution}
\end{defproblem}




\begin{defproblem}{0105:traffic}
\begin{onlyproblem}
\paragraph{Traffic Safety}
A traffic safety institute measured the breaking
distance, in feet, of a car traveling at certain speeds
in miles per hour.
The data from one of those tests are seen in the table.
\footnote{CA 28}
\begin{tabular}{c|c}
  \textbf{Speed} & \textbf{Breaking Distance} \\ \hline
  20 & 23.9 \\
  30 & 33.7\\
  40 & 40.0\\
  50 & 41.7\\
  60 & 46.8\\
  70 & 48.9\\
  80 & 49.0\\
\end{tabular}
\begin{enumerate}
\item Find the quadratic regression equation for these data
\item Using he regression model, predict the breaking distance
when a car is traveling at 65 mph?  Round to the nearest tenth
of a foot.
\end{enumerate}
\end{onlyproblem}
\begin{onlysolution}
\begin{enumerate}
\item $y\approx-0.00746x^2+1.14821+4.80714$
\item 47.9 ft
\end{enumerate}
\end{onlysolution}
\end{defproblem}




\begin{defproblem}{0105:linearmatrixa}
\begin{onlyproblem}
\paragraph{Linear Matrix 1}
Use a 2x3 matrix and the TI-8* command RREF to solve for the
$A$ and $B$ of a line in intercept form (i.e. $Ax+B=y$) that passes
through the following two points (answer in fractions):
\begin{enumerate}
\item (-3,4) and (7,8)
\item (2,0) and (4,0)
\item $(\frac{2}{3},\frac{1}{6})$ and ($4\frac{1}{3},7\frac{5}{6})$
\item (-2.3,-4.7) and (-6.9,-6.5)
\end{enumerate}
\end{onlyproblem}
\begin{onlysolution}
\begin{enumerate}
\item $\frac{2}{5}x+\frac{5}{2}=y$
\item $0=y$
\item $\frac{23}{11}x-\frac{27}{22}=y$
\item $\frac{9}{23}x-\frac{19}{5}=y$
\end{enumerate}
\end{onlysolution}
\end{defproblem}


\begin{defproblem}{0105:linearmatrixb}
\begin{onlyproblem}
\paragraph{Linear Matrix 2}
\begin{enumerate}
\item (1,-2) and (-3,3)
\item (-4,-4) and (4,4)
\item $(\frac{5}{3},-3)$ and $(-\frac{1}{3},5)$
\item (2.14,-3.2) and (-0.11,1.64)
\end{enumerate}
\end{onlyproblem}
\begin{onlysolution}
\begin{enumerate}
\item $-\frac{5}{4}x-\frac{3}{4}=y$
\item $x=y$
\item $-4x+\frac{11}{3}=y$
\item $-\frac{484}{225}+\frac{7894}{5625}=y$
\end{enumerate}
\end{onlysolution}
\end{defproblem}




\begin{defproblem}{0105:dailyhigh}
\begin{onlyproblem}
\paragraph{Average Daily High}
The table below shoes the highest daily temperature 
averaged over the month for the cities of Syracuse, NY;
Washington DC; and Austin, TX.  
\begin{enumerate}
\item Create sinusoidal regressions for each 12-month pattern and predict in which month the three cities will have the same daily high.
\end{enumerate}
\footnote{from the website http://mathbits.com/ MathBits/ 
TISection/ Statistics2/ sinusoidal.html}
\begin{tabular}{c|c|c|c}
	\textbf{Month} & \textbf{NY} & \textbf{DC} & \textbf{TX} \\ \hline
	Jan. & 32 & 43 & 62\\
	Feb. & 34 & 47 & 65 \\
	Mar. & 43 & 56 & 72 \\
	Apr. & 57 & 67 & 80 \\
	May & 69 & 75 & 87 \\
	Jun. & 78 & 84 & 92 \\
	Jul. & 82 & 88 & 96 \\
	Aug. & 80 & 87 & 97 \\
	Sep. & 72 & 80 & 91 \\
	Oct. & 60 & 68 & 82\\
	Nov. & 48 & 58 & 71 \\
	Dec. & 36 & 47 & 63 \\
\end{tabular}
\end{onlyproblem}
\begin{onlysolution}
\begin{description}
\item[NY] $y\approx25.61\cdot{}\sin(.5090x-2.0685)+56.8797$
\item[DC] $y\approx22.7410\cdot{}\sin(.4946x-1.9503)+65.3889$
\item[TX] $y\approx17.742\cdot{}\sin(.5043x-2.0110)+79.1803$
\end{description}
They will never intersect.
\end{onlysolution}
\end{defproblem}



\begin{defproblem}{0105:AltSun}
\begin{onlyproblem}
\paragraph{Altitude of the Sun}
The table below shows the altitude for the
sun in Dallas, Texas, at selected times times
during September 15, 2013.
\footnote{CA 97}
\begin{tabular}{c|c}
  \textbf{Time} & \textbf{Altitude ($^{\circ}$}) \\ \hline
  7:00 & 9.6 \\
  8:00 & 22.0 \\
  9:00 & 34.1 \\
  10:00 & 45.2\\
  11:00 & 54.3\\
  12:00 & 59.5\\
  13:00 & 58.7\\
  14:00 & 52.2\\
  15:00 & 42.4\\
  16:00 & 30.9\\
  17:00 & 18.7\\
  18:00 & 6.2\\
\end{tabular}
\begin{enumerate}
\item Find the sine regression function that models 
the altitude in degrees of the sun as a function of the
time of day.  Use 24.017 hours (the time from sunrise
on September 15 to sunrise September 16) for the period.
\item Use your regression equation to estimate the altitude
of the sun on the 15th at 9:30 a.m.  Round to the nearest
tenth of a degree.
\end{enumerate}
\end{onlyproblem}
\begin{onlysolution}
\begin{enumerate}
\item $y\approx32.2267\sin(.3993x-.5706)+26.9744$
\item $40.3$
\end{enumerate}
\end{onlysolution}
\end{defproblem}



\begin{defproblem}{0105:newton}
\begin{onlyproblem}
\textbf{Newton's Law}
A $190^\circ F$ cup of coffee is placed on a desk in a $72^\circ F$ room.
Newton's Law of Cooling say that the temperature at time $x$ obeys the
following equation: $T(x) = (T_0-T_a)b^x+T_0$, where $T_0$ is the
initial temperature, $T_a$ is the ambient temperature around, and $b$ is 
a constant the depends upon the substance being cooled.  The following
table shows the recorded temperature at one minute intervals of the coffee:
\footnote{Demana}

\begin{tabular}{c|c}
	\textbf{Time} & \textbf{Temp.} \\ \hline
	1 & 184.3 \\
	2 & 178.5 \\
	3 & 173.5 \\
	4 & 168.6 \\
	5 & 164.0 \\
	6 & 159.2 \\
	7 & 155.1 \\
	8 & 151.8\\
	9 & 147.0\\
	10 & 143.7\\
\end{tabular}
\begin{tabular}{c|c}
	\textbf{Time} & \textbf{Temp.} \\ \hline
	11 & 140.0\\
	12 & 136.1 \\
	13 & 133.5 \\
	14 & 130.5\\
	15 & 127.9 \\
	16 & 125.0\\
	17 & 122.8\\
	18 & 119.9\\
	19 & 117.2\\
	20 & 115.2\\
\end{tabular}
\begin{enumerate}
\item Make a scatter plot of the data, with time in $L_1$ and temperature
in $L_2$.
\item Under STAT-EDIT, move onto $L_3$ and enter that it equals $L_2-72$.  Run the exponential regression, being sure to use $L_1$ and $L_3$, and store it in $Y_1$.  Record the $r^2$ value.
\item Suffix a ``+72'' to $Y_1$.  Record your equation.  
\item How well does your function fit the data, both numerically and visually?
\end{enumerate}
\end{onlyproblem}
\begin{onlysolution}
\begin{enumerate}
\item insert graphic
\item $r^2$=99.98\%
\item $T(x)\approx118.0705\cdot .9511^x+72$. 
\item It seems exceedingly close to the data.
\end{enumerate}
\end{onlysolution}
\end{defproblem}



\endinput
