%!TEX root =  ../main.tex
\columntable[0.1in]{2.4in}{
\begin{center}
\begin{tabular}{c|c}
	\textbf{Year} & \textbf{\$} \\ \hline
	1990 & 10.19\\
	1991 & 10.50\\
	1992 & 10.75\\
	1993 & 11.03\\
	1994 & 11.32\\
	1995 & 11.64\\
	1996 & 12.03\\
	1997 & 12.49\\
	1998 & 13.00\\
	1999 & 13.47\\
	2000 & 14.00\\
	2001 & 14.95\\
	2002 & 14.95\\
	2003 & 15.35\\
\end{tabular}
\end{center}
}{Hourly earning\label{tab:hourly}}

\columntable[0.8in]{1.3in}{
\begin{center}
\begin{tabular}{c|c}
  \textbf{mph} & \textbf{ft} \\ \hline
  20 & 23.9 \\
  30 & 33.7\\
  40 & 40.0\\
  50 & 41.7\\
  60 & 46.8\\
  70 & 48.9\\
  80 & 49.0\\
\end{tabular}
\end{center}
}{Breaking distance\label{tab:break}}

\columntable[0.6in]{2.0in}{
\begin{center}
\begin{tabular}{c|c|c|c}
	\textbf{Month} & \textbf{NY} & \textbf{DC} & \textbf{TX} \\ \hline
	Jan. & 32 & 43 & 62\\
	Feb. & 34 & 47 & 65 \\
	Mar. & 43 & 56 & 72 \\
	Apr. & 57 & 67 & 80 \\
	May & 69 & 75 & 87 \\
	Jun. & 78 & 84 & 92 \\
	Jul. & 82 & 88 & 96 \\
	Aug. & 80 & 87 & 97 \\
	Sep. & 72 & 80 & 91 \\
	Oct. & 60 & 68 & 82\\
	Nov. & 48 & 58 & 71 \\
	Dec. & 36 & 47 & 63 \\
\end{tabular}
\end{center}
}{Temperatures\label{tab:temps}}

\renewcommand{\columnseprule}{1.5pt}
\begin{multicols*}{2}
\rule[0.5\baselineskip]{0.4\textwidth}{1pt}
\noindent%
\ExSection\label{sec:0105x}
\begin{exercises}{sec:0105x}
\prob[0105Quad1] Find a quadratic model for the given data.\footnote{CA textbook 13-16}
\subprob \{(1,-1), (2,1), (4,8), (5,14), (6,25)\}
\subprob \{(1.5,-2.2), (2.2,-4.8), (3,4), (-11,2), (5.1,-20.6)\}

\prob[0105Quad2]
\subprob \{(-2,-5), (-1,0), (0,1), (1,4), (2,4)\}
\subprob \{(1.5,-2.2), (2.2,-4.8), (3.4), (-11,2), (5.1,-20.6)\}

\prob[0105Arch] The data below shows the length of the humerus and the total wingspan, in
centimeterss, of several pterosaurs (extinct flying reptiles). \footnote{CA textbook 17}
\subprob Compute a linear regression for these data
\subprob On the basis of this model, which is the projected wingspan of a 
\textit{Quetzalcoatlus northropi}, which is estimated to have has a humerus of 54 cm?  
Round to the nearest centimeter.



\prob[0105ModelDay] For a particular day of the year $t$, the number of 
daylight hours in New Orleans can be approximated by 
$$
d(t)=1.792\sin\left(\dfrac{2\pi(t-80)}{365}\right) + 12.145
$$
where $t$ is an integer and $t=1$ corresponds to January 1.\\ According to $d$, 
how many days per year will New Orleans have at least 10.75 hours of daylight?



\prob[0105Hourly] The average hourly earnings of U.S. production
workers for the 1990's are shown in Table~\ref{tab:hourly}.\footnote{Demana Precalc p163}


\subprob Produce a scatter plot of the hourly earnings ($y$) as a function
of the years since 1990 ($x$).
\subprob Run the linear regression and store it in $Y_1$.
\subprob Record the $r^2$ value.  Does it suggest that this model is appropriate?
\subprob Run the quadratic regression and store it in $Y_2$.
\subprob Record the $r^2$ value.  Does it suggest that this model is appropriate?
\subprob Find the difference between the two models' predictions for the average hourly earnings in 2010.
\subprob Write a four sentence paragraph describing why it can be risky to extrapolate from a mathematical
model, citing your work here in technical language.


\prob[0105traffic]
A traffic safety institute measured the breaking
distance, in feet, of a car traveling at certain speeds in miles per hour.
The data from one of those tests are seen in Table~\ref{tab:break}.\footnote{CA 28}
\subprob Find the quadratic regression equation for these data points.
\subprob Using the regression model, predict the breaking distance
when a car is traveling at 65 mph?  Round to the nearest tenth
of a foot.


\prob[0105LM1]
\begin{tikzpicture}[remember picture,overlay]
\coordinate (here) at (3,-3);
\draw (current page.west |- here) node[right]{
\begin{tabular}{c|c}
  \textbf{Time} & \textbf{Altitude ($^{\circ}$}) \\ \hline
  7:00 & 9.6 \\
  8:00 & 22.0 \\
  9:00 & 34.1 \\
  10:00 & 45.2\\
  11:00 & 54.3\\
  12:00 & 59.5\\
  13:00 & 58.7\\
  14:00 & 52.2\\
  15:00 & 42.4\\
  16:00 & 30.9\\
  17:00 & 18.7\\
  18:00 & 6.2\\
\end{tabular}
\par\\
\textbf{Table:Sun}\label{tab:sun}
};
\end{tikzpicture}%to prevent adding extra space before text
Use a $2 \times 3$ matrix and the TI-8* command RREF to solve for the
$A$ and $B$ of a line in intercept form (i.e. $Ax+B=y$) that passes
through the following two points (answer in fractions):
\subprob (-3,4) and (7,8)
\subprob (2,0) and (4,0)
\subprob $(\frac{2}{3},\frac{1}{6})$ and ($4\frac{1}{3},7\frac{5}{6})$
\subprob (-2.3,-4.7) and (-6.9,-6.5)

\prob[0105LM2]
\subprob (1,-2) and (-3,3)
\subprob (-4,-4) and (4,4)
\subprob $(\frac{5}{3},-3)$ and $(-\frac{1}{3},5)$
\subprob (2.14,-3.2) and (-0.11,1.64)


\prob[0105high] Table~\ref{tab:temps} shows the highest daily temperature  
averaged over the month for the cities of 
Syracuse, NY; Washington DC; and Austin, TX.  Create sinusoidal regressions for each 12-month 
pattern and predict in which month the three cities will have the same daily high. 
\footnote{from the website \url{http://mathbits.com/ MathBits/ TISection/ Statistics2/sinusoidal.html}}



\prob[0105sun] Table~\ref{tab:sun} below shows the altitude for the
sun in Dallas, Texas, at selected times during September 15, 2013. \footnote{CA 97}
\subprob Find the sine regression function that models 
the altitude in degrees of the sun as a function of the
time of day.  Use 24.017 hours (the time from sunrise
on September 15 to sunrise September 16) for the period.
\subprob Use your regression equation to estimate the altitude
of the sun on the 15th at 9:30 a.m.  Round to the nearest
tenth of a degree.


\prob[0105newton] A $190^\circ F$ cup of coffee is placed on a desk in a $72^\circ F$ room.
Newton's Law of Cooling say that the temperature at time $x$ obeys the
following equation: $T(x) = (T_0-T_a)b^x+T_0$, where $T_0$ is the
initial temperature, $T_a$ is the ambient temperature around, and $b$ is 
a constant that depends upon the substance being cooled.  The following
table shows the recorded temperature at one minute intervals of the coffee
\footnote{Demana}
\begin{tabular}{c|c}
	\textbf{Time} & \textbf{Temp.} \\ \hline
	1 & 184.3 \\
	2 & 178.5 \\
	3 & 173.5 \\
	4 & 168.6 \\
	5 & 164.0 \\
	6 & 159.2 \\
	7 & 155.1 \\
	8 & 151.8\\
	9 & 147.0\\
	10 & 143.7\\
\end{tabular}
\begin{tabular}{c|c}
	\textbf{Time} & \textbf{Temp.} \\ \hline
	11 & 140.0\\
	12 & 136.1 \\
	13 & 133.5 \\
	14 & 130.5\\
	15 & 127.9 \\
	16 & 125.0\\
	17 & 122.8\\
	18 & 119.9\\
	19 & 117.2\\
	20 & 115.2\\
\end{tabular}


\subprob Make a scatter plot of the data, with time in $L_1$ and temperature
in $L_2$.
\subprob Under STAT-EDIT, move onto $L_3$ and enter that it equals $L_2-72$.  Run the exponential regression, being sure to use $L_1$ and $L_3$, and store it in $Y_1$.  Record the $r^2$ value.
\subprob Suffix a ``+72'' to $Y_1$.  Record your equation.  
\subprob How well does your function fit the data, both numerically and visually?

   
\end{exercises}
\end{multicols*}

