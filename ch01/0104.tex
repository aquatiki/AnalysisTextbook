%!TEX root =  ../main.tex
\subsection{Reflection}


\objective{Produce reflections of functions and describe their symmetry.}


As you know, multiplying numbers by $-1$ produces the additive opposite: negative values
become positive and positives become negative.  Zero is unaffected, being neither positive
nor negative.  The same principle applies to functions.  What would it look like to ``flip'' all
$x$ values?  $y$ values?

\paragraph{``Outside''}
$-f(x)$ is a reflection over the $x$-axis, leaving $x$'s untouched, and making $y$'s opposite.

We can obtain both the original and reflection across the $x$-axis by writing $\pm f(x)$.  Why
is that not a function?

\paragraph{``Inside''}
$f(-x)$ is a reflection over the $y$-axis, leaving $y$'s untouched, and making $x$'s opposite.

Does the ``opposite'' rule of inside transformations apply or not to the negative on the inside?
\index{transformation!reflection}


\subsection{Symmetry}
\paragraph{Even}\index{even functions}
\marginfig[-0in]{Playing_card_spade_A.png}{The ace of spades (or of clubs or hearts) displays even symmetry about its center.}
Sometimes, multiplying by a negative makes no difference.  In mathematics, this can be very 
helpful to know.  Functions that are the same left-to-right and right-to-left are called \textbf{even}.
Can you guess why?  Aren't only numbers even (or odd)?



\begin{derivation}{Even Functions}
An even function has the property $f(x)=f(-x)$ for every $x$ in its domain.
\end{derivation}



Among the more basic functions are power functions.  We will study them in great depth in
chapter 5.  The names ``even'' and ``odd'' come from the similar behavior of $x^n$ when
$n$ is even or odd.

Notice that evenness has the visual appearance of putting a mirror on the $y-$axis.  Did you 
know human beings are made to find such left-right symmetry appealing?  Study the faces of
attractive people, and you will find evenness to be a rule of thumb. 



\begin{derivation}{Odd Functions}\index{odd functions}
An odd function has the propety $-f(x)=f(-x)$ for every $x$ in its domain.
\end{derivation}


\marginfig[-0in]{English_pattern_queen_of_hearts.png}{The queen of hearts (and many other playing cards) display an odd symmetry about the center.}
You might have supposed oddness with be top-to-bottom and bottom-to-top symmetry.  Why isn't
that possible for functions?  Instead, we find that these functions are the same whether we 
proceed left from the origin, or flip the right half upside-down.


\paragraph{Rotational}\index{rotation!symmetry}
Looking at the odd power functions, it becomes clear that there are two ways to regard their
symmetry.  Either, they are reflections left-right and up-down (in either order), or they are
$180^\circ$ of themselves.  Is it possible for a function to have any other angle of symmetry
with itself?

Moving on to symmetry \emph{between} functions (or relations), we see a lot of rotational
symmetry exists.  All lines of the form $y=ax$ are rotations of each other.  $f(x)=x^2+k$ and
the relation $y=\pm\sqrt{x-k}$ are $90^\circ$ rotations of each other.

Relations can be their own rotation.  Equilateral triangles with their center at the origin are 
all $120^\circ$ rotations of themselves.  In fact, every regular $n-$gon (polygon) is its
own rotations every $\frac{360}{n}^\circ$.  As we take the limit and let $n$ increase,
we approach circular symmetry, continuous rotational symmetry.

Rotation is hard to discover under normal algebra, but it very easy with a matrix.
You can read in §18.1 how the rotational matrix is 
$\begin{bmatrix} 
	\cos { \theta  }  & -\sin { \theta  }  \\ 
	\sin { \theta  }  & \cos { \theta  }  
\end{bmatrix}$


\inlinefig{Purple_Star}{Rotational symmetry becomes much more complicated in higher dimensions.}

