 % These are the 1-1 Exercises, aka 0101X

\begin{defproblem}{0101X:arrow}
\begin{onlyproblem}
\begin{exercise}[Archery Problem]
\footnote{Foerster Precalc 1-1,1}
An archer climbs a tree near
the edge of a cliff, then shoots an arrow high
into the air. The arrow goes up, then comes
back down, going over the cliff and landing in
the valley, 30 m below the top of the cliff. The
arrow’s height, y, in meters above the top of
the cliff depends on the time, $x$, in seconds
since the archer released it. Figure shows
the height as a function of time.
\begin{enumerate}
\item What was the approximate height of the
arrow at 1 s? At 5 s? How do you explain the
fact that the height is negative at 5 s?
\item At what two times was the arrow at 10 m
above the ground? At what time does the
arrow land in the valley below the cliff?
\item How high was the archer above the ground
at the top of the cliff when she released the
arrow?
\item Why can you say that height is a function of
time? Why is time not a function of height?
\item What is the domain of the function? What is
the corresponding range?
\end{enumerate}%
\end{exercise}%
\end{onlyproblem}%
\begin{onlysolution}%
\begin{enumerate}
\item 20m, -17.5m; below
\item 0.3s, 3.8s, 5.3s
\item 5m
\item only one altitude
\item 0<x<5.3; -30<y<25
\end{enumerate}
\end{onlysolution}%
\end{defproblem}


\endinput
