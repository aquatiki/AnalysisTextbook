%!TEX root =  ../main.tex

\subsection{Exponential Equations}

\objective{Solve exponential/logarithmic equations through a variety of techniques}


Whatever function we are dealing with, if $f(x)=f(3)$ then $x$ must equal 3.  This is true
if $f(x)$ is a logarithmic, exponential or other kind of function.  So, if $\log_{\pi}(x+3) =
\log_{\pi}{7-x}$, then $x+3 = 7 - x$ and $2x = 4$, so $x$ must equal 2.  Similarly,
if $3^{2x+1} = 3^7$ then $2x+1 =7$ and $x=3$.  These are the most basic kinds of
exponential equations.

You are probably pretty good at spotting when numbers are related via multiplication
and division because of so many years of practicing such arithmetic.  $56x+49=0$ should
leap out to you as $7(8x+7)=0$ or even $x=-\frac{7}{8}$ because you have known
your seven times table for so many years now.  Exponents are not memorized to the
same extent --- and nor should they be --- but perhaps you might consider expanding your
familiarity with them just a little more.  The return on investment is pretty low after cubes,
but you will be pleasantly surprised at the added reach knowing the numbers on the following
table will give you:

$$
\begin{matrix} 
2 & 4 & 8 & 16 & 32 \\ 
3 & 9 & 27 & 81 & 243 \\ 
4 & 16 & 64 & 256 & 1024 \\ 
5 & 25 & 125 & 625 &  \\ 
6 & 36 & 216 &  &  \\ 
7 & 49 & 343 &  &  \\ 
8 & 64 & 512 &  &  \\ 
9 & 81 & 729 &  &  
\end{matrix}
$$

\begin{example}{Exponential Base Manipulation}
\exProblem
Solve $9^{x+1}=\left(\frac{1}{27}\right)^{2x}$.

\exSolution
With a familiarity with exponents, we see that both numbers are power of 3.  We can 
re-write each base to show this, and use the same equality principle as before.
\begin{align*}
	9^{x+1} &= \left(\frac{1}{27}\right)^{2x}\\
	(3^2)^{x+1} &= (3^{-3})^{2x}\\
	3^{2x+2} &= 3^{-6x}\\
	2x + 2 &= -6x\\
	8x &= -2\\
	x &= -\frac{1}{4}
\end{align*}
\end{example}

\begin{example}{Logarithmic Base Manipulation}
\exProblem
Solve $\log_7{x} = \log_{49}{2x}$.

\exSolution
There are lot of transformation we could try at this point which would be valid.
Let us consider how we can make both sides have a logarithmic base of 49.
$\log_7{7} = \log_{49}{49}$, which shows that is you square the base, you must square
what you are taking the log of.  This will allow us to rewrite the equation and solve.
\begin{align*}
	\log_7{x} &= \log_{49}{2x}\\
	\log_{49}{x^2} &= \log_{49}{2x}\\
	x^2 &= 2x \\
	x^2 - 2x &= 0\\
	x(x-2) &= 0\\
	x &= \{0, 2\}
\end{align*}
If we check our two solutions, however, we find a contradiction.  $\log_b{0}$ does
not exist: there is no exponent you can raise a number to and get 0.  We can check that
2 works as a solution in our TI-8*.  (Newer TI-8*'s have a LOGBASE function under MATH, 
but everyone can check by typing $\frac{\log{2}}{\log{7}}$ etc.)
\end{example}

\begin{example}{Combining Logs}
\exProblem
Solve for $x$: $\log_4{(x+10)}+\log_4{(x+34)}=4$

\exSolution
We can (must) combine the two logs, in order to make the problem simpler.
\begin{align*}
	\log_4{(x+10)(x+34)} &= 4\\
	\log_4{x^2+44x+340)} &= 4\\
	4^4 &= x^2+44x+340\\
	0 &= x^2+44x+340-256\\
	0 &= (x+2)(x+42)
\end{align*}
Of the two solutions, only -2 works in the original problem; -42 does not
\end{example}

\subsection{Substitution}
Finally, there are some problems which do not yield to combination, manipulation, or
changing log-form.  These problems require substitution.  For example, nothing in 
$e^{2x}+5e^x=1$ seems to fit what we have so far described.  We cannot combine
any terms of the left, so we must ``explain away'' for a moment the troublesome
exponents.  We pick a variable to represent what we cannot deal with: $e^x$.  

The problem now becomes $u^2+5u=1$.  This is not magic: we must un-substitute at the
end, or else we are solving a different problem.  But the magical aspect is that we can now
see that this is a quadratic problem.  $u^2+5u-1=0$ does not yield to factoring, so
we must resort to the Quadratic Formula.

$$
u=\frac{-5\pm\sqrt{29}}{2}
$$

We check with the TI-8*, and the ```plus'' solution is positive, while the ``minus'' one is not.
Since $u$ is really just a cipher for $e^x$, we need only the positive solution.  This means
$e^x=\frac{-5+\sqrt{29}}{2}$ or $x=\ln{-5+\sqrt{29}}-\ln{2}$, which is around -1.65.
