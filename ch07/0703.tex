%!TEX root =  ../main.tex

\subsection{Like Powers}

\objective{Simply log expressions}


Whether we use the Triangle of Power or not, it is very powerful to recognize
that logs are simply the inverse of exponents.  Without this insight, expressions
like $\log_{10}{100} + \log_{10}{1000}$ would be very intimidating.  But with this
insight, we can paraphrase as we go in our minds: ``There is some exponent
to put on 10 and get 100.  Add to that, some exponent we put on 10 to get 
1000.  Adding exponents comes from multiplying bases, so this is the same as
$\log_{10}10000$.  That is asking the question, what exponent do we put on 10
to get 10,000?  5!''

We think it is even easier in Triangles, but we might show you both styles for
the time being:

$$
\tripow{10}{}{100} + \tripow{10}{}{1000} = \tripow{10}{}{10000} = 5
$$

In other words, two logs added, is the same as one long of a multiplication.  This
works for the inverse operation of subtraction: two logs subtracted is the same
as one log of a division.  Lastly, it also works for an exponent: the log of a number
with an exponent is the same as the exponent multiplied against the log of number.

Here, Triangle notation really shines superior, because notation like $\log_2{64^2}$
is confusing.  Does it mean $\log_2{64} \cdot \log_2{64}$ or $\log_2{4096}$, which
is the difference between 36 and 12?  How much clearer is

$$
\tripow{2}{}{\tripow{64}{2}{}} vs \tripow{\tripow{2}{}{64}}{2}{}
$$

Even more consequential is the two-fold possibilities for $2^{3^4}$.  Is that $(2^3)^4$
or $2^{(3^4)}$?  That is the difference between 4096 and $2^81$, the latter of which
is a 25 digit number!  But no one can make the same mistake with

$$
\tripow{\tripow{2}{3}{}}{4}{} vs \tripow{2}{\tripow{3}{4}{}}{}
$$


\subsection{Names}
There are some logs which occur so commonly, that they base is not written.  Normally,
the word ``log'' with no base written means $\log_{10}$.  Log base $e$ is very common,
and has its own symbol $\ln$.  This acronym comes from the French, \textit{log
natural}, since $e$ is the natural number.\footnote{In French, as in Spanish, adjective
most often follow their noun, not precede it.}  

Computer scientists most often used $\log_2$, which is called the binary log,
written lb.  In fact
in many disciplines, their flavor of log is the only one used, so it is assumed and the name
``log'' is written, which can fool outsiders into assuming $\log_{10}$.  For example,
Wolfram Alpha is fantastically powerful website for mathematics and other fields,
and so they use ``log'' to mean ``ln''!  We will not be so tricky, and you may assume
the solution to $\log{x}=2$ is 100.

\subsection{Change of Base}
Logs are amazing.  But given the rarity of most bases, it become necessary to ask,
Is there a way to convert them?  If we know how to re-write a log as an exponent,
and take an exponent as a multiplier, then we can:

\begin{align*}
	\log_b{a} & = c \\
	b^c &= a\\
	\ln{b^c} &= \ln{a}\\
	c \cdot \ln{b} &= \ln{a} \\
	c = \frac{\ln{a}}{\ln{b}} 
\end{align*}

Notice that it does not matter what base we chose in the third line.  $\ln$ ($\log_e$)
works just as well as $\log_{\pi}$.  This rearrangement to have an arbitrary
base is called the the Change of Base Formula.


\begin{derivation}{Change of Base Formula}
$$
\tripow{b}{}{a} = \frac{\tripow{c}{}{a}}{\tripow{c}{}{b}} \quad \text{a.k.a.} \quad
\log_b{a} = \frac{\log_c{a}}{\log_c{b}}
$$
\end{derivation}

\begin{derivation}{Negative Logarithm}
$$
\tripow{b}{}{1/y} = -\tripow{b}{}{y}  = \tripow{1/b}{}{y}
\quad \text{a.k.a.} \quad
\log_b{\frac{1}{y}} = -\log_b{y} = \log_{\frac{1}{b}}{y}
$$
\end{derivation}



Students often confusing ``the sum of logs'' with ``the log of a sum''.  What can we say
about $\log_b{(a+c)}$?  On the face of it, not much.  But through some creative manipulation,
one identity can be made, which is used in Probability Theory:


\begin{align*}
	\log_b{(a+c)} 	&= + \log_b{a} - \log_b{a} + \log_b{(a+c)}\\
				&= \log_b{a} + \log_b{(a+c)} - \log_b{a}\\
				&= \log_b{a} + \log_b{\frac{a+c}{a}} \\
				&= \log_b{a} + \log_b{1 + \frac{c}{a}}\\
	\tripow{b}{}{a+c} &= \tripow{b}{}{a} + \tripow{b}{}{\left(1 + \frac{c}{a}\right)}
\end{align*}

Lastly, we have already made with the Triangle of Power:

\begin{derivation}{P-Plus Logs}
$$
\tripow{m}{}{x} \pplus \tripow{n}{}{x} = \tripow{m\cdot{}n}{}{x}
\quad \text{a.k.a.} \quad
\log_m{x} \oplus \log_n{x} = \log_{mn}{x}
$$

$$
\tripow{m}{}{x} \pminus \tripow{n}{}{x} = \tripow{m\div{}n}{}{x}
\quad \text{a.k.a.} \quad
\log_m{x} \pminus \log_n{x} = \log_{\frac{m}{n}}{x}
$$

$$
\tripow{\tripow{m}{n}{}}{}{x} = \frac{\tripow{m}{}{x}}{n}
\quad \text{a.k.a.} \quad
\log_{m^n}{x} = \frac{1}{n}\log_m{x}
$$
\end{derivation}
