%!TEX root =  ../main.tex
\subsection{Growth}

\objective{Model and predict year-over-year percentile growth}


In linear equations, everything is relative to the number 0.  Have a derivative or slope 
of 0 constitutes a flat line.  Numbers greater than 0 are called positive slope, while
numbers less than it are called negative slope.  If the initial value is not 0, then it is
added to the equation, unlike the slope, which is multiplied against the independent
variable.

Everything about exponential equations is moved to the next level of operators.  We saw
back in §1.4 that the standard form of an exponential equation is $y=a\cdot{}b^x$.
All such equations begin at the point $(0,a)$, because plugging in 0 yields $b^0$
and regardless of what $b$ is, anything to the zero power is 1.  Multiplying by $a$ 
therefore changes our initial value.  And because powers are simply repeated multiplication,
have $b$ as a base means the equation multiplies by $b$ every unit step of $x$.

What if $b=1$?  No matter what exponent we put on 1, it will remain 1.  If $b>1$,
then every increment of $x$ will grow the output.  If $b<1$, then every increment
of $x$ will reduce the output.  Because exponential equations are build off of 
multiplication --- not addition --- everything is relative to 1, not 0.

Hopefully, you are very familiar with percentages, and how even the word comes
from the Latin \textit{``out of one hundred''}.  This give us the clue to convert
percentages to decimals: divide by 100.  So if a mathematical model verbally says a 
system is growing at 5\%, means that any given year, the value is 105\% of the
value from the year before.  This means $b$ in our equation is $1.05$.  Were we 
to use a $b$ of $0.05$, that would be a 95\% loss year-over-year!

\subsubsection{TI-8*}
The TI-8* is a very useful tool for modeling exponential growth.  In many 
real-life situations, everything takes place in QI, but the $y$-scale often
varies greatly from problem to problem.  For example, what is a good
window for the following problem?  A house was purchases for \$500,000
and appreciates 3.2\% per year.  What is its future value in 5, 10, and 30
years?  When will it exceed 2 million dollars?

Because this is a strictly increasing function, the value at the beginning will be 
the lowest.  This means our Xmin and Ymin should be 0 and 500000 respectively.
We know we need to see at least to the year 30, so our Xmax must be at least that,
and our Ymax must be at least 2000000.  These turn out to be overly conservative
estimates, but they help us see how to guess.  The TABLE is also very useful in
setting up windows.

\subsection{Compound Interest}
In most banking institutions, money moves hands and changes more often than once
a year, so interest is compounds (or calculated) more often.  If you are making
five percent per annum, but the bank compounds monthly, that does not mean you
make 5\% twelve times a year!   Instead, they give you a twelfth of 5\% twelve times
per year.  This makes our equation more complicated:

$$
A = P\left(1+\frac{r}{n}\right)^{n\cdot{}t}
$$

$A$ is the amount at the end of the term, $P$ is the principle which the investment began
at (the initial amout of money), $r$ is the interest rate, $n$ is the number of times per year
it is compounded, and $t$ is the number of years the investment is left in.  If banks think
in terms of periods (be they month, quarters, years, etc.) then the exponent is the number
of periods the money is left in.


\subsection{Doubling Time}
How can we build an equation when we don't know the rate of growth, except as a time?
For example, suppose your grandfather noticed that movie theater prices have doubled 
every eight years.  When he was a kid, it was a quarter!  One way would be to generate
numerical data and run a regression on them.  Let $x$ be years since your grandfather
was a kid.  You know $a$ is 0.25, a quarter dollar.  So far, we have $y=\frac{1}{4}b^x$.
We know every eight years, the price has doubled, so that makes points (8,0.5) ; (16,1)
; (24,2), etc.  The TI-8*'s EXPREG yields $y=0.25(1.090507733)^x$.

Another way might be to recognize that we want to be multiplying by 2 every time (since
we are talking about doubling) but that only every 8 years should it happen.
$y=\frac{1}{4}2^x$ would double every day, so dilate the function to be 8 times wider:
$y=\frac{1}{4}2^\frac{x}{12}$.  This shows us that the same exponential function
can have multiple (in fact, infinite number of) representations.

\paragraph{e}
We will not explain its origin until next chapter, but you should know that many (if not most)
exponential equations involve the number $e$.  Like $\pi$, it is around 3, only $e$ is a little less,
not a little more.  You can find $e$ on your TI-8* as 2ND $\div$, or the more useful $e^x$
as 2ND-LN.