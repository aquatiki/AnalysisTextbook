%!TEX root =  ../main.tex

\objective{Understand and simplify the relationships between logs, powers, and roots.}

\subsection{The Three Components}
Our modern mathematical notation obfuscates one relationship with three, different notations. 
The following equations all express the same things:
\begin{enumerate}
\item $\log_2{8}=3$
\item $2^3=8$
\item $\sqrt[3]{8}=2$
\end{enumerate}

All three embody the same relationship: 2 is the base, 3 is the exponent, and 8 is result.  
Three elements suggest a three-sided shape, a \emph{triangle of power}.
$\tripow{2}{3}{8}$

Leaving off any side of the triangle of power suggests that the missing number is needed.
\begin{enumerate}
\item $\log_2{8}$ can be represented as $\tripow{2}{}{8}$
\item $2^3$ can be represented as $\tripow{2}{3}{}$
\item $\sqrt[3]{8}$ can be represented as $\tripow{}{3}{8}$
\end{enumerate}

Some people complain that this new notation ruins the line height, that is is too tall.
But these pedants rarely write\\ 
$(2\div(3+4))\div((5+6)\div(7+8))$.  Indeed, it is preferable to see:

$$\frac{\frac{2}{3+4}}{\frac{5+6}{7+8}}$$

In the same way, one might write $2\triangle^3$, $2\triangle_8$, and $\triangle^3_8$,
but expand in two-dimensions when the occasion permits.

\subsection{Inverses}
The true usefulness of the triangle of power is revealed when we try to present more
complicated relationships.  Some students immediately grasp what $e^{\ln{x}}$ is saying,
others struggle for years with the notation.  \emph{There is a power we can put on \emph{e}
to get \emph{x}.  Raise \emph{e} to that power.}  If you get it, the answer is obviously $x$.
But the symbols certainly don't help you see it.  Instead, triangles make the relationship 
more obvious: 

\begin{equation}
\tripow{e}{ \tripow{e}{}{x}}{}=x \quad \text{vs} \quad e^{\ln{x}}
\end{equation}

The top triangle is blank in the same place it occupies in the larger triangle.  Because the
$e$'s are in the same place, everything cancels, leaving only the $x$.  
Other hard expressions which are simple inverses are equally obscure in traditional notation,
and quite clear in triangle form:
\begin{equation}
\tripow{}{e}{\scriptstyle \tripow{x}{e}{}} = x \quad \text{vs} \quad \sqrt[e]{x^e} = x
\end{equation}
\begin{equation}
\tripow{\tripow{}{e}{x}}{2}{} = x \quad \text{vs} \quad \sqrt[e]{x}^e = x
\end{equation}
\begin{equation}
\tripow{e}{}{\tripow{e}{x}{}} = x \quad \text{vs} \quad \ln{e^x} = x
\end{equation}
\begin{equation}
\tripow{}{\tripow{x}{}{e}}{e} = x \quad \text{vs} \quad \sqrt[\log_x{e}]{e} = x
\end{equation}
\begin{equation}\tripow{\tripow{}{x}{e}}{}{e} = x \quad \text{vs} \quad \log_{\sqrt[x]{e}}{e} = x
\end{equation}


\subsection{P-Plus}
The properties of logs, exponents, and roots become much more transparent in triangle notation.
For example, the sum of exponents look like this:
$$\tripow{b}{m}{}\cdot{}\tripow{b}{n}{}=\tripow{b}{m+n}{}$$

We shall see that this bears a strong resemblance to a similar property of logs:
$$\tripow{b}{}{m} + \tripow{b}{}{n} = \tripow{b}{}{m+n}$$

Graphically, keeping the base the same but switching from exponent to result changes where
the addition and multiplication happen.  You will make all the various versions of the rules in the
exercises and problems, but there is one relationship which might appear overly perplexing at first.
Consider the products of roots:

$$
\tripow{}{x}{z} \cdot{} \tripow{}{y}{z}
$$

We have not had occasion to contemplate this before.  What operation should govern this 
relationship?  Given the thorough treatment of rational exponents in chapter 5, perhaps it would
be more clear for you to rewrite this problem as fractional powers:

{\Large
$$z^{\frac{1}{x}} \cdot z^{\frac{1}{y}}$$
}

The answer is a root which is the sum of the reciprocals of $x$ and $y$, or a power which is the
reciprocal of that!  This unusual operation is actually rather common in practical applications and
deserving of its own symbol in this book, $\pplus$.  This symbol was chosen because
the reciprocal of the sum of reciprocal is used in parallel resistance, whose symbol is $\parallel$.


\begin{derivation}{P-plus}
$$x\pplus y = \cfrac{1}{\frac{1}{x}+\frac{1}{y}} = \cfrac{1}{\frac{y}{xy}+\frac{x}{xy}} = 
\cfrac{1}{\frac{x+y}{xy}} = \frac{xy}{x+y}$$
\end{derivation}


This strange operation is necessary in a world where power and roots are reciprocals of
each other:
$$
\tripow{a}{x}{} = \tripow{}{\frac{1}{x}}{a}
$$

There are many more intriguing relationship that can be written clearly and intuitively
on the Triangle of Power, e.g. $\tripow{m}{}{x}\pplus\tripow{n}{}{x} = \tripow{m\cdot{}n}{}{x}$
or $\tripow{x}{}{a}\cdot{}\tripow{a}{}{y} = \tripow{x}{}{y}$  You are encouraged to experiment
and tinker with this powerful tool.
