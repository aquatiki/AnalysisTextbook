%!TEX root =  ../main.tex
%\renewcommand{\columnseprule}{1.5pt}
%\rule[0.5\baselineskip]{0.4\textwidth}{1pt}
\noindent
\LabSection{Triangular Tables}\label{sec:0703p}
\begin{exercises}{sec:0703p}
\begin{multicols}{2}
\lab{} Write the Triangle for each
\subprob 4 raised to the power of 2 results in the number 16
\vspace{5mm}
\subprob 5 raised to the power of -2 results in the number 1/25
\vspace{5mm}
\subprob 2 is the exponent that is placed on base 4 resulting in the number 16
\vspace{5mm}
\subprob 1/3 is the exponent placed on base 8 to obtain the number 2.
\vspace{5mm}
\subprob $p$ is the exponent placed on base 7 to obtain the number 49.
\vspace{5mm}
\subprob $p$ is the exponent placed on base $b$ to obtain the number $n$.
\vspace{5mm}
\end{multicols}

\lab Complete the chart below, which details the relationship $\tripow{4}{n}{p}$.  Remember that $\tripow{b}{m}{}\cdot\tripow{b}{n}{}=\tripow{b}{m+n}{}$.
\begin{tabular}{ c | c || c | c || c | c || c | c }
	\textbf{$n$} & \textbf{$p$} & \textbf{$n$} & \textbf{$p$} & \textbf{$n$} & \textbf{$p$} & \textbf{$n$} & \textbf{$p$} \\
	\textbf{Number} & \textbf{Exponent} & \textbf{Number} & \textbf{Exponent} & \textbf{Number} & \textbf{Exponent} & \textbf{Number} & \textbf{Exponent} \\ \hline \hline
	1 & & 7 & 1.404 & 13 & 1.850 & 19 & 2.124 \\ \hline
	2 & & 8 & & 14 & & 20 & \\ \hline
	3 & 0.792 & 9 & & 15 & & 21 & \\ \hline
	4 & 1.000 & 10 & & 16 & & 22 & \\ \hline
	5 & 1.161 & 11 & 1.730 & 17 & 2.044 & 23 & 2.262 \\ \hline
	6 & & 12 & & 18 & & 24 & \\ \hline
\end{tabular}

\lab Apart from `4', which kind of numbers had already been filled in the table for $p$?  Why?
\vspace{2cm}
\begin{multicols}{2}
\lab{} Unfortunately, it is important to be familiar with traditional notation, if for no other reason that to be able to use your TI-8*!  $2^3=8$ ; $\sqrt[3]{8}=2$ ; and $\log_2{8}=3$ are all obtuse ways of writing $\tripow{2}{3}{8}$.  Use your chart above (and not a calculator) to solve for $x$:
\subprob $1.730 = \log_4{x}$
\vspace{5mm}
\subprob $x = \log_4{19}$
\vspace{5mm}
\subprob $3 = \log_4{x}$
\vspace{5mm}
\subprob $\log_4{x} = 1.161$
\vspace{5mm}
\subprob $\log_4{16} = x$
\vspace{5mm}
\subprob $\log_4{x} = 2.808$
\vspace{5mm}
\end{multicols}

\lab{} Complete the mystery chart and tell what the base \emph{must} be:\\
\begin{tabular}{ c | c || c | c || c | c || c | c }
	\textbf{Number} & \textbf{Exponent} & \textbf{Number} & \textbf{Exponent} & \textbf{Number} & \textbf{Exponent} & \textbf{Number} & \textbf{Exponent} \\ \hline \hline
	1 & & 7 & 1.210 & 13 & 1.594 & 19 & 1.829 \\ \hline
	2 & 0.431 & 8 & & 14 & & 20 & \\ \hline
	3 & 0.683 & 9 & & 15 & & 21 & \\ \hline
	4 &  & 10 & & 16 & & 22 & \\ \hline
	5 &  & 11 & 1.490 & 17 & 1.760 & 23 & 1.948 \\ \hline
	6 & & 12 & & 18 & & 24 & \\ \hline
\end{tabular}


\end{exercises}