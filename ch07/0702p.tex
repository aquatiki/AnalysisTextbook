%!TEX root =  ../main.tex
\renewcommand{\columnseprule}{1.5pt}
\begin{multicols*}{2}
\rule[0.5\baselineskip]{0.4\textwidth}{1pt}
\noindent
\LabSection{Money Matters}\label{sec:0702p}
\begin{exercises}{sec:0702p}
\lab{} 1  A gentleman who wasn’t very bright invested his money in a bank that paid 0\% interest.  Once he realized this, he pulled out him money, saying, ``It will all be gone soon!  Year over year, they multiply my money by zero!”  At 0\% interest, what would a bank actually multiply the money by?


\vspace{2cm}
\lab{} I slightly less dense fellow invested his money at a bank with 2\% annual  interest.  Does his money get multiplied by 2 every year?  Explain.


\vspace{3cm}
\lab{}  A man who was not a complete imbecile invested his money at 5\% annual interest.  He says that means he gets \$0.05 for every dollar he has in the bank.  With \$1,000 invested, he says he make \$50 every year in interest.  Is that true?  Explain.

\vspace{2cm}
\lab{} Your uncle bought a home for \$100,000 and it appreciates by 3\% per year.  How much will it be worth in 5 years?


\vspace{2cm}
\lab{} Did you solve L4 via repeated multiplication?  Show how it could be solved using exponents.


\vspace{3cm}
\lab{} Explain why interest rates are all calculated relative to the number 1, not 0.

\vspace{3cm}
\lab{} Suppose there is some amount $A$ one ends up with after time $t$, where one began with a principle $P$, gaining or losing a signed interest rate $r$.  Write a formula for $A$ in terms of $P$, $r$, and $t$.


\vspace{3cm}
\lab{} You bought a house for \$200,000 five years ago, and today it is worth \$322,102.  Using the simpler equation $y=ab^x$, where is time since five years ago, solve first for a and then b.  Use b to answer how much interest per year your house is appreciating by.

\vspace{4cm}
\lab{}  Describe what you think the point of this problem set is, using whole sentences.
\end{exercises}
\end{multicols*}