%!TEX root =  ../main.tex
\renewcommand{\columnseprule}{1.5pt}
\begin{multicols*}{2}
\noindent
\rule[0.5\baselineskip]{0.5\textwidth}{1pt}

\noindent
\subsection{3-in-1}
\noindent
1.  Consider a new notation proposed for mathematics.  The 3-way relationship of 2 and 3 and 8 is expressed like this: $\tripow{2}{3}{8}$.  This is called the \textbf{Triangle of Power}, and a blank corner is asking you to solve for it.  This means $\tripow{2}{3}{} = 8$.  (It looks suspiciously like exponentiation.)  What does $\tripow{3}{4}{}$ equal?

\vspace{3cm}
\noindent
2.  Returning to the first example, we can see that $\tripow{}{3}{8}=2$.   What is $\tripow{}{4}{16}$?

\vspace{3cm}
\noindent
3.  Rounding out the triangle, $\tripow{2}{}{8}=3$.  What is $\tripow{10}{}{100}$?

\vspace{3cm}
\noindent
4.  The triangle is much clearer than exponents, roots, and logarithms.  If $\tripow{2}{3}{} \cdot \tripow{2}{4}{} = \tripow{2}{c}{}$ what must $c$ be?

\vspace{2cm}
\noindent
5.  If we try to move a number from one spot to another, it forces certain changes.  $\tripow{4}{x}{} = \tripow{}{2}{4}$  What number must $x$ be?  

\vspace{2cm}
\noindent
6.  You once wrote a program to find the reciprocal of the sum of reciprocals, written $\oplus$.  Using that symbol, create a rule for multiplying two triangles, where the bottom right never changes, but the top does.  i.e. What is $\tripow{}{m}{x}\cdot{}\tripow{}{n}{x}$?

\vspace{3cm}
\noindent
7.  One of the most powerful uses of the Triangle is in clearly symbolizing inverse.  For example,

$$\tripow{2}{}{\tripow{2}{x}{}} = x \quad \text{and} \quad \tripow{2}{\tripow{2}{}{x}}{} = x$$  Write the other
five.

\vspace{8cm}

Describe in technical vocabulary what you think the point of this problem set is, using 
complete sentences.
\end{multicols*}