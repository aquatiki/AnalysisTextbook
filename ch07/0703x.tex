%!TEX root =  ../main.tex
\renewcommand{\columnseprule}{1.5pt}
\begin{multicols*}{2}
\rule[0.5\baselineskip]{0.4\textwidth}{1pt}
\noindent
\ExSection\label{sec:0703x}
\begin{exercises}{sec:0703x}
\prob[0703All6] There are six ways to arrange $x$ and $y$ and $9$ on the Triangle of Power (or on $a^b=c$, $\sqrt[a]{b}=c$, or $\log_a{b}=c$). Write down all six and rearrange each into a form enterable in your TI-8*.  
\subprob Graph each under ZOOM-STANDARD and sketch.
\subprob Find two lattice points on each.
\subprob How many lattice points are there on each?
\subprob Which are inverses of each other?
\subprob Which are reciprocals of each other?
\subprob Find reciprocals of the others


\prob[0703:LogProofs] True or False?
\subprob $-\log_b{x} = \log_x{b}$ 
\subprob$\ln{|x|} = |\ln{x}|$
\subprob $\log_{\sqrt{3}}{27} =6$
\subprob $e^{x\ln{x}} = x^x$ when $x>0$
\subprob $\sqrt{\log{x}} = \log{\sqrt{x}}$
\subprob $x\log_x{x^x} = x^2$
\subprob $\log_{\sqrt{5}}{\frac{1}{125}} = -4$
\subprob ${2^{\frac{1}{2}^{\frac{1}{2}^{\iddots}}}} = 1$



\prob[0703SolveLogs] Solve for $x$.  (Rewriting with the Triangle of Power is very helpful!)
\subprob $\log_x{(x+6)}=2$
\subprob $\log{\left(\log{\left(\log{(x+1)} \right)} \right)} = 0$ 


\prob[0703SimplifyLogs] Prove
\subprob $\ln{\left(\frac{\sqrt{x}}{x}\right)} + \ln{\sqrt[4]{ex^2}}=0.25$
\subprob $\log_b{a} = \frac{1}{\log_a{b}}$
\subprob $\log_b{x} \cdot{} \log_x{y} = \log_b{y}$
\subprob $x^{\log_b{y}} = y^{\log_b{x}}$  (Hint, rewrite $x$ as $b$ to a certain power)
\subprob $\log_b{y^x} = \log_{\sqrt[x]{b}}{y}$
\subprob $\log_y{x^b} = -\log_x{y^b}$
\subprob $2^{-\log_x{2}} = \sqrt[-\log_2{x}]{2}$


\prob[0703DescribeDifference] Describe the difference between $\log_5{x} + 2$ and $\log_5{(x+2)}$ and show how Triangle notation might help.

\prob[0703DescribeCalclog] Using your TI-8* and reasoning skills, find the asymptotes, hole(s), and local extrema of:
\subprob $y=x\log{x^2}$
\subprob $y=x^2\ln{|x}$
\subprob $y=\sqrt[\log_{x^2}{9}]{|x|}$
\subprob $y=\log_{|x|}{9}$



\end{exercises}
\end{multicols*}