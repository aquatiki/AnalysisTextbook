%!TEX root =  ../main.tex

\subsection{$xy\theta$}

\objective{Produce and derive trigonometric function graphs}


On a piece of paper or on the screen of an electronic device, we typically graph only two
variables.  3D graphs are usually ``faked'', as in orthographic projections or the like.  The 
Unit Circle is actually three things graphed at once, and so somewhat obscures each:
the height (sine), the width (cosine), and the angle turned ($\theta$).  We will tease apart
various attributes of the Unit Circle and their relationships to $\theta$, in order to understand
each more clearly.

\subsubsection{$y\theta$}
\marginfig[-0in]{\chapdir/pics/TIsinx}{$y=\sin(x)$ on the TI-8*}
If we graph the height we have ascended to as the dependent variable, and the angle we
have turned as the independent variable, this corresponds to $y=\sin(x)$.  This is a good
beginning, because we are graphing the $y$ of the Unit Circle as a our $y$, but we have
changed the $x$ to $\theta$.  Enter this equation in your TI-8* and hit ZOOM-TRIG, begin
sure to be in radians mode.

The graph has not been shifted up or down at all, so its midline or \textbf{axis} is the line
$y=0$.  The graph oscillates as much as one up or one down from that axis, so 1
is the \textbf{amplitude}.  We know that the pattern of the Unit Circle (relative to
the angle turned) takes $360^\circ$ or $\tau$ radians to repeat, so that should
be the \textbf{period}.  What has the calculator done to make us get back to the
same point on the way in four tick marks?  Press WINDOW, and you will see that
the Xscl is $1.57\dots$, a.k.a. $\frac{\tau}{4}$.

Notice how the graph spends a great deal of time near the origin looking like $y=x$.
In other words, its derivative at 0 is 1.  Practice moving your writing hand in a smooth
wave, tracing the unit circle at a consistent pace, but focusing your attention on your
height above and below the $x$-axis.  This should feel that same as tracing $y=\sin(x)$.
Because this graph is so ubiquitous in nature, anything like it is described with the
adjective \textbf{sinusoidal}, and anything which moves as it does is said to be in 
\textbf{simple harmonic motion}.

\subsubsection{$x\theta$}
A cosine graph is very, very similar.  Tracing the Unit Circle while focusing on your
$x$ position is nearly identical, except the cycle begins on 1.  Change you TI-8* to
$y=\cos(x)$ but change nothing else.  Now you are graphing $x$ from the Unit Circle
on $y$, and $\theta$ from the Unit Circle on $x$.

\begin{example}{Watermill}
\exProblem
A mill is powered by a water-wheel in a river, which is 12 feet in diameter.  You observe
that it takes 20 seconds to complete a rotation and only the bottom 1ft is in the water.
There is a flag or marker at the top of the wheel right now.
Create an equation to model the water-wheels behavior, graph it, and determine the
time the flag will go into the water, and how long it spends underwater each rotation.


\exSolution

We begin by reasoning from the Unit Circle to the water-wheel.  Unit Circle graphs have an
amplitude of 1 because the are waves on a circle of radius 1.  The water-wheel has
a radius of 6 ft, so that will be our amplitude.  Graphically, that would make our wave
six times taller than normal, so we need a vertical dilation of 6.  So far we have
$$
y=6\sin(x)
$$
\marginfig[-0in]{\chapdir/pics/6sinx}{$y=6\sin(x)$}
Which produces a graph as in Fig.~\ref{fig:6sinx}.  The water-wheel must have its axis
translated, if only the bottom 1ft is in the water.  With an amplitude of 6ft, our current
version is 5ft too low, so we can revise our equation to $y=6\sin(x)+5$.  We also note
that as a sine wave, our graph begins in the middle and is rising, whereas we need to
model an object beginning at the top.  Therefore, we switch our equation to a cosine 
function.

The normal period of a sinusoidal wave is $\tau$, but we need the period to be 20.
Horizontal dilation is accomplished by dividing --- not multiplying, as in vertical
dilation --- so we must ask what to divide $\tau$ by in order to get 20.  The answer is
$\frac{\tau}{20}$.  Our final equation is (including the form we must use in the TI-8*):
$$
y=6\cos\frac{\tau}{20}(x)+5 = 6\cos\frac{\pi}{10}(x)+5
$$

According to the zero function of the grapher, the height is zero at $x=8.135705$
and again at $x=11.864295$, meaning the flag will go under just after 8 seconds
from now, and be underwater for a little less than 4 seconds.

\marginfig[1in]{\chapdir/pics/waterwheelgraph}{The entire mill setup visualized}
\end{example}

