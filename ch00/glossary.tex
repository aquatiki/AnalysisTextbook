\newglossaryentry{translation}
{
	name={translation},
	description={the transformation that shifts a graph (without distortion) some finite distance in the plane or space}
}
\newglossaryentry{pre-image}
{
	name={pre-image},
	description={The original function or relation, before it under goes transformations and become an image}
}
\newglossaryentry{image}
{
	name={image},
	description={A transformed version of the original function or relation}
}
\newglossaryentry{asymptote}
{
	name={asymptote},
	description={A shape that the graph of function approaches and gets arbitrarily close to on some open interval}
}
\newglossaryentry{periodic}
{
	name={periodic},
	description={Repeating on some regular interval, called the period}
}
\newglossaryentry{linear}
{
	name={linear},
	description={An algebraic function, equivalent to a polynomial of degree 1.  The general equation is $ax+b$, where $a$ and $b$ are constants.  The graph is a straight line.  The numerical data displays an add-add pattern.  Verbally, one variable is directly proportional to another, but with a potentially different starting point}
}
\newglossaryentry{quadratic}
{
	name={quadratic},
	description={An algebraic function, equivalent to a polynomial of degree 2.  The general equation is $ax^2+bx+c$, where $a,b,c$ are constants and $a\ne0$.  The graphical shape is called a parabola.  The numerical data displays an add-second difference pattern.  Verbally, one variable is directly proportional to the square of another}
}
\newglossaryentry{exponential}
{
	name={exponential},
	description={A transcendental function, where the independent variable appears as an exponent.  An untranslated, general equation form is $a\cdot{}b^x$, where $a$ and $b$ are constants}
}
\newglossaryentry{logarithm}
{
	name={logarithm},
	description={The inverse of an exponential function.  $y=\log_a{x}$ is equivalent to $a^y=x$}
}
\newglossaryentry{closure}
{
	name={closure},
	description={The quality of some operations over certain sets, that performance of that operation on members of the set always produces a member of the same set}
}
\newglossaryentry{transcendental}
{
	name={transcendental},
	description={Not algebraic.  That is, not derived from a polynomial with whole number exponents and rational coefficients}
}
\newglossaryentry{irrational numbers}
{
	name={irrational numbers},
	description={Numbers that cannot be represented as a ratio integers.  When written as a decimal, the digits are non-repeated and non-terminating.  Denoted with the symbol $\mathbb{I}$}
}
\newglossaryentry{rational numbers}
{
	name={rational numbers},
	description={Numbers of the form of a ratio of an integer over an integer, denotes with the symbol $\mathbb{Q}$}
}
\newglossaryentry{integers}
{
	name={integers},
	description={The natural numbers, their multiplicative opposites, and zero, i.e. $\dots, -3, -2, -1, 0, 1, 2, 3, \dots$.  Denoted with the symbol $\mathbb{Z}$}
}
\newglossaryentry{natural numbers}
{
	name={natural numbers},
	description={The quantities visible in nature, namely $1, 2, 3, \dots$.  Denoted with the symbol $\mathbb{N}$}
}
\newglossaryentry{set-builder notation}
{
	name={set-builder notation},
	description={A writing system for describing a set by enumerating its elements or stating the properties that its members must satisfy.  A variable and its definition are set in curly brackets, with the variable separated by a colon or pipe.  On the left, the variable may have a given domain.  On the right, there may be a predicate indicated via operators and logical symbols, or the values may simple be enumerated}
}
\newglossaryentry{interpolation}
{
	name={interpolation},
	description={The act of estimating values between given data points via a function}
}
\newglossaryentry{extrapolation}
{
	name={extrapolation},
	description={The act of estimating values outside of given data points via a function}
}
\newglossaryentry{mathematical model}
{
	name={mathematical model},
	description={A function which is used to describe and/or predict a real-world phenomenon, expressing relationships between quantities}
}
\newglossaryentry{function notation}
{
	name={function notation},
	description={A way of writing functions invented by Leonhard Euler in 1734, where the function is named first, followed by in the input variables in parentheses, separated by commas.  This is equated then to the function.  In this notation, inverse functions are written $f^{-1}(x)$}
}
\newglossaryentry{amplitude}
{
	name={amplitude},
	description={The height of a wave, above the midline.  On a sinusoidal graph, the distance from axis to maximum or minimum}
}
\newglossaryentry{Cartesian plane}
{
	name={Cartesian plane},
	description={A coordinate system, uniquely identifying every point thereupon by a pair of numerical coordinate, which are the signed distances from two, perpendicular axes}
}
\newglossaryentry{function}
{
  name=function,
  description={A relationship between two variable quantities for which there is exactly one value of
	the dependent variable for each value of the independent variable in the domain}
}
\newglossaryentry{set}
{
  name=set,
  description={A well-defined collection of distinct objects, as well as an object in its own right}
}
\newglossaryentry{domain}
{
  name=domain,
  description={The set of values that the independent variable of a function can have.  For Reals,
  this is typically only limited to exclude even roots of negative numbers and division by zero}
}
\newglossaryentry{range}
{
  name=range,
  description={The set of all values of the dependent variable that correspond to values of the
  independent variable in the domain}
}
\newglossaryentry{independent variable}
{
  name={independent variable},
  description={The input of a function, equation, or formula}
}
\newglossaryentry{dependent variable}
{
  name={dependent variable},
  description={The output of a function, equation, or formula}
}
\newglossaryentry{numerical}
{
  name=numerical,
  description={Referring to constants, such as 2 or {\ensuremath{\pi}}, rather than to parameters
  	or variables}
}
\newglossaryentry{discrete}
{
  name=discrete,
  description={Not continuous.  That is, dealing with countable sets, and therefore excluding
  calculus and analysis}
}
\newglossaryentry{graph}
{
  name=graph,
  description={In two dimensions, the representation of the collection of all ordered pairs $(x, f(x))$, in the form of a curve on a Cartesian plane, together with the axes, and other labels}
}
\newglossaryentry{algebraic}
{
  name=algebraic,
  description={Involving only the operations addition, subtraction, multiplication, division, powers, and roots a finite number of times}
}
\newglossaryentry{pi}
{
  name={\ensuremath{\pi}},
  description={The ratio of the circumference of circle to its diameter},
  sort=pi
}
\newglossaryentry{scalar}
{
  name=scalar,
  description={A quantity --- such as time, speed, or volume --- that has magnitude but no direction}
}
\newglossaryentry{acceleration}
{
  name=acceleration,
  description={The instantaneous rate of change of velocity}
}
\newglossaryentry{anti-derivative}
{
  name=antiderivative,
  description={$g(x)$ is an antiderivative of $f(x)$ if and only if $g'(x)=f(x)$.  Also called an \textit{indefinite integral}}
}
\newglossaryentry{carrying capacity}
{
  name={carrying capacity},
  description={The maximum population that can be sustained by a particular environment}
}
\newglossaryentry{even function}
{
  name={even function},
  description={A function $f$ is even if $f(-x) = f(x)$ for all $x$ in its domain.  Visually, the
  left and right side of the graphs are reflections of each other, across the $y$-axis}
}
\newglossaryentry{odd function}
{
	name={odd function},
	description={A function $f$ is odd if $f(-x) = -f(x)$ for all $x$ in its domain.  Visually, the graph is identical if it undergoes a $180^\circ$ rotation about the origin.}
}
\newglossaryentry{indeterminate form}
{
	name={indeterminate form},
	description={A quasi-numeric state of a limit that does not yield enough information to constitute an answer.  Typical forms are $0/0, \infty/\infty, 0 \cdot{} \infty, \infty - \infty, 0^0, 1^\infty, \infty^0$}
}








