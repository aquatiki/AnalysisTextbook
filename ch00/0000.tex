%!TEX root =  ../main.tex

\chapterimage{pano-5.jpg}
\mychapter{Introduction}{introduction}

\section{Motivation}

This textbook exists to facilitate a class coming after Algebra II and before Calculus II.  It is meant
to be a coherent mixture of Paul A Foerster's \textit{Precalculus with Trigonometry} and his
\textit{Calculus: Concepts and Applications}.  Some students can handle moving at a much 
faster pace through Precalculus topics, and hence cover the first four chapters of \textit{Calculus}.  
A student should not be expected to 
do well on the AP exam after this class (without additional side work), because of inadequate
time spent on integrals.  

\personfeature[-2.2in]{foerster}{Paul
    Foerster}{1936-, U.S.A.}{taught mathematics at Alamo Heights 
    High School in San Antonio, Texas, from 1961 to 2011\cite{BuckleyFoerste}
    . After earning 
    a BS in chemical engineering, he served four years as an engineering 
    duty officer with the Navy's Nuclear Propulsion Program. Tutoring 
    high school students while on active duty, he found his true passion 
    --- seeing students get the ``Aha!'' reaction as they finally grasp new 
    concepts. After completing his Navy service, he went back to college 
    for a spring and a summer to get his teaching certification. His teaching 
    has been interrupted only by a year's leave of absence to earn a master's 
    degree in mathematics through an academic year institute awarded by 
    the National Science Foundation.}
    

Once a school is over a certain size, and the pool of students to draw from is sufficient, there may
be the critical mass needed to create an accelerated class for those who can proceed quickly through 
the skills and understanding needed to appreciate calculus.  While it may be possible in such
situations to promote kids sooner, it may also make sense to wait until their junior year, in which case
an analysis class is a possible solution.  Students are introduced to categories of 
functions, with an eye towards their uses in calculus, at the same time.  At best, a student would proceed
from this class to a Calculus II class, which would move quickly through derivatives and 
spend most of its time on integrals.  At worst, if a student (barely) made it through this class, they would
no doubt do well in an AB class, and have time to go back and master derivatives.


\section{Prerequisites}
  
Students are expected to have performed well in a strong Algebra II class, and retain excellent
arithmetic, graphing, and factoring skills.  This textbook contains an extensive set of appendices 
, some of which should
be assigned as summer work.  We recommend thorough consultation with the sophomore teacher
to see which areas need more work and require additional assignments.  It is burdensome, but we
also highly recommend making assignments due at regular intervals throughout the summer, perhaps
via electronic submission.
American summer breaks are too long, and students forget a great
deal over three months.  Because of the exploratory nature of assignments in the appendix, it is
possible to assign exercises which were not covered at all in Algebra II, with little or no instruction.

% used to have calculator here

This textbook contains many calculator lessons for the Texas Instrument 83 or 84 (hereafter \textbf{TI-8*}).
This antiquated device may be taken into the AP, ACT, and SAT exams, and hence 
should be thoroughly mastered by a student who wishes to maximize his or her tools.  Other
devices may be used, but are unwise, since the TI-8* is the most powerful tool allowed in the 
standardized tests.



\section{Worldview}

It is highly remarkable how many  mathematics textbooks begin without  
prolegomena of any kind.  Such an oversight belies all of mathematical work, which seeks 
to establish foundations and ascertain unassailable truths.  We will attempt to counter this trend
in the shortest space.

\personfeature[-1.0in]{Immanuel_Kant_001}{Immanuel
    Kant}{1724-1804, Prussian}{was a German philosopher, and is
    is considered a central figure in philosophy. He argued that the 
    human mind creates the structure of experience, that reason 
    is the source of morality, and that the world as it is ``in-itself'' is 
    independent of our concepts of it. He is sometimes credited with 
    being the greatest Modernist philosopher, and yet laying the
    ground from which Post-Modernism sprung.
    \href{https://en.wikipedia.org/wiki/Immanuel_Kant}{(Wikipedia)}}

The universe in which we find ourselves doing mathematics is a 
reasonable and orderly one.  The Ancient Greek distinction between `cosmos' 
and `chaos', or the Biblical separation of the watery abyss from the peaceable land are
metaphors in recognition of this fact.  The ethical and personal disorder which we bring to the world
does not belie its inherent predictability.  
God has not absconded from his creation, but has
chosen to mediate himself, leaving us with the ability and responsibility to discover
what is hidden in the material world (Proverbs 25:2).  It is only by suppressing this truth that one could
pronounce as Einstein did that, ``The most incomprehensible thing about the universe 
is that it is comprehensible.''\footnote{Albert Einstein, \emph{Journal of the Franklin Institute}, Physics and Reality, p221,1936.}.
Only by \textit{a priori} rejecting the Judeo-Christian worldview -- that a reasonable and rational God
has made human beings in his image, and that we are therefore capable of thinking his thoughts
after him, and making true statements about the world -- can we understand comments such 
as Kant's ``[W]ir auch, gleich als ob es ein glücklicher unsre Absicht begünstigender Zufall wäre, 
erfreuet (eigentlich eines Bedürfnisses entledigt) werden, wenn wir eine solche systematische 
Einheit unter bloß empirischen Gesetzen antreffen.''\footnote{
[We rejoice when, just as if it were a lucky chance favoring 
our aim, we do find such systematic unity among merely empirical laws.] \textit{Critique of Judgment}, 1790.}
Rationality in human discourse and orderliness in the natural world are not things
which need to be proven but which must be presupposed for any proofs to be made.

Mathematics is not a construct of the human mind.  It is entirely foreseeable and 
understandable that advances in number theory later yield results in cryptography,
or that esoteric processes such as analytic continuation, produce useful dividends in
quantum physics, long after they are discovered.  Mathematics is the representation
and conceptualization of the invisible, created world.  Human beings are the representation
of the maximally personal God in the impersonal media of time, energy, and matter. 
`Number' and other mathematical features are part of human language:
doing math is part of what it means to be human.  It is part of \emph{what} we are,
it is good, and it is imperative --- an integral component of our purpose.

\section{Method}

\personfeature[-0.6in]{Albert_Einstein_(Nobel)}{Albert
    Einstein}{1879-1955, German}{was a theoretical physicist, and
    TIME's person of the century.  He is considered to be one of the
    smartest men of the the modern era for his revolutionary Special
    Theory of Relativity, and again for General Relativity.  He stated
    on numerous occasions that he could not ``imagine some will or 
    goal outside the human sphere'' and hence it is a happy
    coincidence that mathematics and science are so
    mutually affirming.
    \href{https://en.wikipedia.org/wiki/Albert_Einstein}{(Wikipedia)}}

Human beings are not mere containers of information.  Teaching is not plugging-in
a download wire from the side of the teacher's head into that of the students.  Most
things of value are \emph{caught}, not taught.  It is expected that a teacher will show the 
lessons in this book to their students in the context of a safe, loving relationship, which
is of necessity personal, and therefore costly.  Eventually, students should feel safe to
fail in this class, because only failing produces change and ultimately learning.  Success
only confirms us in our biases.
Of course, no one likes to fail, so every class must balance a three-legged stool: structure, 
support, and challenge.

If at all possible, teachers should present the material from a given lesson from a different
perspective than it is shown herein.  This gives the student the opportunity to read the book,
should they wish for another angle on the matter.  If students know how to learn, they should
be able to do the \textbf{problems} on their own, and spend class time presenting their 
thoughts to others, or working through the \textbf{exercises} and discussing the book.  
The first section of each chapter (after the first) is somewhat informal and self-explanatory, 
and should be assigned as homework after the previous test without instruction.

Math assignments should never consist of 10 easy exercises, rotely completing the task defined
in the lesson, followed by 10 harder exercises, followed by 5 word-problems.  Problem sets
should be a journey, that is, should possess a narrative arc.  Class periods should do the same.
Students must learn to ask themselves questions, a process they should see modeled in the
teacher.
``What did we just learn?  How can I summarize it in my own words?  How do I think it
will be applied?  How does it relate to what I already know?  When was this encountered
for the first time, and in what context (history)? How can it help me 
understand more of the myriad things above my level?  How do I solve difficult problems
involving this principle? What might this become in my life and in the lives of my
neighbors?  How can I communicate this to others?  What are some common misunderstandings
people have when encountering this material?''

Finally, while we do have a volume of material we wish students to learn and retain,
it is irresponsible to allow students to think they have ``mastered'' mathematics.  Anyone
with a Masters degree in any field has been dislodged from his or her notion that one
can know \emph{everything} about anything.  When the Ph.Ds tell us they do not know all
there is to know about even a sliver of their discipline, we did not believe them ... at first.
But slowly, over years of paper-writing and drilling down into the subject, we see even they
can not know how much there is to know about one topic within their discipline! This is
only discouraging if we are delusional enough to think that our achievement is the 
\textit{summum bonum}, the highest good.
We can indeed learn true things and asymptotically approach full understanding, but no
amateur should be so deluded as to think they know all there is to know about some
field within mathematics.  Hence, many lessons should include a taste of just how much
more there is to know, including problems that cannot be fully solved.


\section{To the Student}
\marginfig[-0in]{TI-84_Plus}{The TI-84+.\label{fig:TU84}}

This is both a textbook and a workbook.  ``Math is not a spectator sport'', and that means
you do not understand something until you've tried it.  Don't shortchange yourself by reading a
section before you've attempted the problems.  If you want to read ahead, that's fine, but
try the assignment before the reading.  Even if you don't get it all, consider it part of the reading.
Unless specifically asked to, we recommend you do not Google the answers.
Do you want to understand
the world?  Do you know that you are in a protected space, a safe time to explore and 
make mistakes, a time that you will not be afforded later?  Take the time to learn as you go, 
and you will learn how to learn, a lesson you can use for the rest of your existence.  If you get
desperate, cultivate relationships with human beings who can ask you questions that
allow you to discover the answer yourself, not be spoon-fed.

Always look out for \textbf{bold} words, in the problems, in the readings, and in the exercises.
Copy them down into your notes and try to define them for yourself, before you look in the
glossary or online. We
encourage you to take notes in the Cornell style.  This simply involves you drawing a large,
left-of-center capital `I' over your entire note page before you begin, taking notes as usual
in the right section, adding subject and paragraph headings on the left, and making
succinct summaries at the bottom of the page, perhaps a few before the test.  Headings,
your name, and the date should go across the top.
There are many videos on the internet about `Cornell Notes'.

\section{To the Teacher}

All the material in the prerequisites needs to be covered, ideally before the year begins.  
Chapters 1-12 and 14 are the minimum.  Chapter 16 and 18-20 exist for the teacher's sake,
to allow variety over the years.  Ideally, that next class would use Foerster's excellent
\textit{Calculus} textbook, taking a month to review chapters 1-4, and then moving 
on to the rest of the book, perhaps augmented the material with some lessons from
multivariable calculus and/or differential equations.

In a perfect world, every capable student would take Calculus II and Statistics
before graduating high school, but we don't live in that world.  Time permitting, discrete
topics should be broached, especially with students who will not take Stats.  The other discrete topics
are here as aides for those who have never seen them before.  
Appendices B, C, and D exist for the teachers who love those topics and wish
to expand or vary the chapters from year to year.  There is always one student who can tell
they are not getting the whole story and will not let it rest!

Classes should consist of a brief introduction or grabber, the ``problem'' set, a succinct lesson, and then
discussion as much as possible.  Ideally, every class period will end with students summarizing what
they have learned and speculating about what it could become, or what they will do next.  Homework
should consist of careful note-taking and selected exercises.

