%!TEX root =  ../main.tex

\subsection{Separating $x$ and $y$}

\objective{Use parametric functions to determine a path and velocity}


Physics teaches us that it is often preferable to separate $x$ and $y$, and to
consider them both functions of $t$ and not one a function of the other.  This is 
typically done because gravity (which Einstein showed to be the same as 
acceleration, and hence is a quadratic equation) pull in one dimension, while
the others are simple linear equations.

Another benefit to parameterizing an equation, is that we also can \emph{when}
something happened.  So far, we have been differentiating $y$ with respect to 
$x$, but what if we want to know how quickly $y$ is changing over time?

\subsubsection{Parameter Domain}
Parameters (most often $t$) also have a domain.  In science, it is often quite
irrelevant to reference any of the variable before the experiment begins or
after it ends.  If an object is thrown on Earth, and we model its path with an
upside-down parabola, what happens before $t=0$ or after it hits the ground
is meaningless.  Parameters' domains provide a natural and intuitive way to
model this limited domain. 

\index{TI-8*!zoom standard}
Pay careful attention to the WINDOW on your TI-8*.  In RADIANS mode, the
default (i.e. ZOOM-STANDARD) parameter domain is $[0,2\pi]$.

\subsubsection{Eliminating the Parameter}
Sometimes, we still want to construct a typical equation from a parametric 
set.  This is called \textbf{eliminating the parameter}.  In chapter 10, we will
learn many trigonometric identities to remove sines and cosines of $t$, but in 
this section on algebraic functions, we will restrict ourselves to
\textbf{substitution} and \textbf{elimination}.

\subsubsection{Implicit Differentiation}\index{derivative!notation}
Until now, we have been solely using LeGrange notation for derivatives, e.g.
$y'$ (pronounced, ``$y$ prime'').  However, this isn't always helpful, especially
if we want to know what we have been differentiating with respect to.  Until now,
that hasn't mattered!  Now we might be looking for the differential of $y$ with 
respect to $x$, or with respect to $t$.

A differential is a measure of sensitivity to change.  How much does $y$ alter,
given minuscule nudges to $x$ (or $t$).  This differential can also be written with the
letter `d', and so ``the differential of $y$ over the differential of $x$'' can symbolized
as $\frac{dy}{dx}$.  It can also be used like a function, to say ``take the derivative with 
respect to $x$ of ...'', as in $\frac{d}{dx}\left( y\right)$.

This fractional way of writing derivatives is called \textbf{Leibniz's notation}, after
Gottfried Leibniz.  It helps show how we can find the derivative of a parametric equation.
If we seek $\frac{dy}{dx}$, that is equivalent to $\cfrac{\frac{dy}{dt}}{\frac{dx}{dt}}, \frac{dx}{dt}\ne
0$.

\index{derivative!higher order}
Derivatives of derivatives (called \textbf{higher order derivative}) were written 
with more tick-marks (e.g. $f''(x), f'''(x)$, etc.) or parenthetical numbers (e.g. $f^{(2)}(x),
f^{(3)}(x)$, etc.) in Lagrange notation.  For Leibnitz's, they get more complicated:
$\cfrac{d\left(\frac{d\left(\frac{dy}{dx}\right)}{dx}\right)}{dx} = 
\left(\frac{d}{dx}\right)^3y = \frac{d^3y}{dx^3}$.
