%!TEX root =  ../main.tex
\renewcommand{\columnseprule}{1.5pt}
\begin{multicols*}{2}
\rule[0.5\baselineskip]{0.4\textwidth}{1pt}
\noindent
\LabSection{x, y, ...t?}\label{sec:0402p}
\begin{exercises}{sec:0402p}
\lab{} We are most accustomed to seeing functions where y is dependent upon x, but that is not the only kind that can exist.  Press MODE on your TI-8* and switch to PAR (short for PARAMETRIC).  What are x and y both functions of?  What has your X button become?

\vspace{3cm}
\lab{} Try setting X and Y both equal to T.  In RAD mode, view under ZOOM-STANDARD.  Sketch what you see.


\vspace{4cm}
\lab{} Press WINDOW and explain why the graph only exists in the first quadrant.


\vspace{3cm}
\lab{} Lets get a little more complicated.  Make $X1=T^2-3T+1$ and $Y1=-T^2+2T+3$.  Sketch the graph.  Explain why Y cannot be a function of X.


\vspace{3cm}
\lab{}  Sometimes, we graph a function parametrically and we see that it CAN be converted to a function of x.  In that case, we typically see to eliminate the parameter and create an ordinary equation for it.  In this case, it will be difficult to solve for t in one equation and substitute it into the other.   Instead, let us eliminate the difficulty, namely $t^2$.  In other words, add the two equations from L4.  Solve for $t$.


\vspace{3cm}
\lab{} Now substitute the definition of t you found make into either original equation.  What about your new equation confirms that y is not a function of x?

\vspace{3cm}
\lab{} Still in RAD, change to X1=3sin(T) ; Y1=3cos(T) and X2=5cos(T) ; Y2=5sin(T).  Under MODE, switch to SIMULtaneous mode, and graph both sets of equations at once.  Observe carefully.  If you miss the “animation”, try adding or subtracting 0 (i.e. change any equation in any way, and the TI-8* will start over again.)  What is the difference in the two graphs?

\vspace{3cm}
\lab{} What if both 5’s had been 3’s?  Assuming we can eliminate the parameter, what would be different then about equations sets 1 and 2?

\vspace{3cm}
\lab{} In your own word, describe what you thin the point of this problem set it.



\end{exercises}
\end{multicols*}