%!TEX root =  ../main.tex

\subsection{Inside and Outside}

\objective{Graph and apply absolute value transformation}


\index{Absolute Value!Transformation}
Consider what applying absolute value does as a \emph{transformation}.
First of all, what would $y=|f(x)|$ do?  Obviously, this will prevent $y$ from ever being negative.
But what happens to those negative segments of the function?  They --- and they alone ---
are reflected over the $x$-axis.  This creates cusps at every turn, moments that will
be un-differentiable.

What if we try $f(|x|)$?  This will cause negative inputs to receive the output as if they were
positive.  Graphically, this means the left-half of our graph will be the mirror image of the
right half.

\index{Absolute Value!of motion, terms}
\subsection{Distance, Velocity, Jerk}
Absolute value was invented to describe distance.  The verbal question ``how far is $x_1$ 
from $x_2$?'' is best represented as $|x_1-x_2|$.  Typically, we have some function modeling 
\textbf{position} and then we contrast that position with something else, 
and are asking a question of  \textbf{distance}.  What would have to be true about 
some graph to leave it susceptible to change 
by the application of absolute values?  How can absolute value transformation produce
undifferentiable moments in a graph?

Another common function to deal with in physics is a speed graph.  This too can be negative,
which when converted to \textbf{velocity}, might make issues. Another source of difficulty
is when we disregard the sign of our input, effectively transforming into $f(|x|)$.  Thinking
graphically, how would the left side correspond to the right?  What kind of symmetry would
such a transformation enforce?  The derivative of speed is position.

Lastly, we will often consider acceleration.  Most processes effected by acceleration 
do not care if it is positive or negative (deceleration).  The absolute value of the 
acceleration is sometimes called  ``jerk'', though this term is not universally standardized.  
The derivative of acceleration is speed.

Studies have begun to come out about higher orders, functions whose derivative is acceleration 
(called snap, crackle, and pop), but these terms are so rarely needed that the names are trivial.
  
