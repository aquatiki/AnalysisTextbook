%!TEX root =  ../main.tex
\renewcommand{\columnseprule}{1.5pt}
\begin{multicols*}{2}
\rule[0.5\baselineskip]{0.4\textwidth}{1pt}
\noindent
\LabSection{Mirror, Mirror, in R3}\label{lab:DD01}
\begin{exercises}{lab:DD01}
\lab[] You will need a light source, protractor, and graph paper for this lab.  Hold the light at various distances, until you are able to achieve a circle about the origin with radius 4.  Trace the circle of light and write the equation for such a shape here.

\vspace{3cm}
\lab[] Tilt your light until the shape is clearly no longer a circle, but is still a closed shape.  What is the common name for such a shape (beginning with O) and the technical name (beginning with E)?

\vspace{2cm}
\lab[] Move your light-picture until it has the same center as the circle (the origin), and the longest line through it lies on the $x$-axis.  Trace your new shape and record its domain and range here.

\vspace{2cm}
\lab[] Determine the ratio $b$ of the width to the height on your oblong object.  Record it in an equation of the form $x^2+\left(\frac{1}{b}y\right)^2=r^2$, where $r$ is the distance from the origin to the right edge.

\vspace{3cm}
\lab[] Use the tools of algebra to manipulate your equation until it equals 1.

\vspace{3cm}
\lab[] Predict what the graph of $\left(\frac{x}{6}\right)^2 + \left(\frac{y}{7}\right)^2 = 1$ would look like.

\vspace{3cm}
\lab[] Solve the equation for $y$ and enter the positive and negative version into your TI-8*.  Graph the two halves as a continuous relation on your graph paper.

\vspace{2cm}
\lab[] Graph the parabola $y=\frac{1}{4}x^2$ on your graph paper.  Move your light source until it best approximates that shape.  Record the angle your light makes with the surface of the paper, using your protractor.

\vspace{2cm}
\lab[] Algebraically, what feature do all these shapes have in common in their equations?

\vspace{2cm}
\lab[] Record in whole sentences what you think the point of this problem set is.






\end{exercises}
\end{multicols*}