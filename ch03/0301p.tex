%!TEX root =  ../main.tex
\renewcommand{\columnseprule}{1.5pt}
\begin{multicols*}{2}
\rule[0.4\baselineskip]{0.4\textwidth}{1pt}
\noindent
\LabSection{In Pieces}\label{sec:0301p}
\begin{exercises}{sec:0301p}
\lab{} There is a function called the \textbf{floor} function, written
$f(x)=\left\lfloor x \right\rfloor$.  In the TI-8*, it is called \texttt{int(},
found under MATH-NUM.  Graph the function under a standard 
window and sketch a copy below.  Describe what you
think the function does to the input in your own words.

\vspace{3cm}
\lab{}  In the function continuous?  If not, where not?  

\vspace{3cm}
\lab{} Select an appropriate locus and calculate the instantaneous rate of change
of the function.  Extrapolate for all Reals, and write a piecewise function
for the derivative of the floor function.

\vspace{3cm}
\lab{} Another common function from computer science is the \textbf{remainder}
function, typically called \texttt{mod} in most programming languages, or is just written with
a percent sign (\%).  It can be reproduced quite simply on the TI-8*.  For example,
$g(x)=x\%5$ can be entered as $Y_2=X-5int(X/5)$.  Sketch a graph below.
How would \emph{you} define the remainder function?  

\vspace{4cm}
\lab{} Find $g'(2)$.  The recurrent discontinuities should be obvious.  Write
a piecewise function for the derivative of the modulus function $y=x\%n$
for any $n$.

\vspace{4cm}
\lab{} A linear but periodic function is hard to define.  We will cheat and simply
linearize a periodic function: $h(x)=2\cos^{-1}(\cos(x\pi\div2))\div\pi-1$.%
%Let $Y_1=|fpart(X/2)|$. 
%Let $Y_2=remainder(iPart(|X|),2)$.  
%But only leave on $Y_3=2|Y_2(X)-Y_1(X)|$.
Sketch a graph below.

\vspace{4cm}
\lab{} Unlike the previous functions, this one has no discontinuities.  However,
there are an infinite number of places where the instantaneous rate of change
is incalculable.  Define \emph{all} the places where the difference quotient will not
work and we cannot find the derivative.

\vspace{3cm}
\lab{} What did you learn as a result of doing this exploration that you did not
know before?
\end{exercises}	
\end{multicols*}