%!TEX root =  ../main.tex

\subsection{Various Forms}
\begin{objective}
Find attributes and graphs of parabolas from equations and/or geometric criteria.
\end{objective}

You should have a fair amount of experience graphing, factoring, and describing
parabola and quadratic equations.  The principles of algebra one learns dealing with quadratics
apply generally to the rest of mathematics and will serve you well for the rest of your life.
The slope of lines tangent to parabola are easy to calculate.
Find the difference quotient of $f(x)=x^2+6x+5$ for any $x$.


The spread out for of $ax^2+bx+c$ has three advantages.  
It can be easily differentiated
It shows the y-intercept plainly
It can be plugged into the quadratic formula as is

However, for graphing, it can be a little obtuse.  We might start by factoring, and see that it is
$(x+5)(x+1)$.  This form --- called intercept form --- \index{quadratic!intercept form}
has only one advantage: it is immediately obvious what the zeros of the function are.

\index{quadratic!vertex form}
Lastly, it is a bit more work, but completing the square and writing in the form $a(x-h)^2+k$
is the easiest to graph from, its one advantage:
the vertex is immediately obvious

\subsection{Angles of Incidence}
Not usually taught in an Algebra class, parabolas have many geometric properties.  The most
physically useful is that every incoming line has an angle of incidence that will take them all
through the same place, called the focus.  This is why headlights and telescopes are paraboloids,
3D parabolas.  A corollary to this property is that every parabola is the set of points equidistant 
from a point and a line.  The distance from the vertex to this focus (and directrix) is $p$, in the
formula\index{quadratic!parabolic form}

$$
y-k = \frac{1}{4p}(x-h)^2
$$

~\vfill