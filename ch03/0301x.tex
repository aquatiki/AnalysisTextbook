\begin{defproblem}{0301:ParaPerpA}%
\begin{onlyproblem}%
\begin{exercise}
Write the slope-intercept forms of the equations of the lines
passing through the given point and a) parallel to the given
line and b) perpendicular to the given line.\footnote{Larson CPC
87-96}
\begin{enumerate}
\item $(-3,2), x+y=7$
\item $(\frac{7}{8},\frac{3}{4}), 5x+3y=0$
\item $(2,5), x=4$
\item $(2,1), y=2$
\item $(-3.9,-1.4), 6x+2y=9$
\end{enumerate}
\end{exercise}
\end{onlyproblem}

\begin{onlysolution}
\begin{enumerate}
\item a) $y=-x-1$ b) $y=x+5$
\item a) $y=-\frac{5}{3}x+\frac{53}{24}$ b) $y=\frac{3}{5}x+\frac{9}{40}$
\item a) $x=2$ b) $y=5$
\item a) $y=1$ b) $x=2$
\item a) $y=-3x-13.1$ b) $y=\frac{1}{3}x - 0.1$
\end{enumerate}
\end{onlysolution}
\end{defproblem}

\begin{defproblem}{0301:ParaPerpB}
\begin{onlyproblem}
\begin{exercise}
Same as above.
\begin{enumerate}
\item $(2,1), 4x-2y=3$
\item $(-\frac{2}{3},\frac{7}{8}), 3x+4y=7$
\item $(-1,0), y=-3$ 
\item $(5,-3), x=-2$
\item $(2.5,6.8), x-y=4$
\end{enumerate}
\end{exercise}
\end{onlyproblem}

\begin{onlysolution}
\begin{enumerate}
\item yup
\item yeah
\item fo sho
\item uh huh
\item you got it
\end{enumerate}
\end{onlysolution}
\end{defproblem}

\begin{defproblem}{0301:GrapherA}%
\begin{onlyproblem}%
\begin{exercise}
Identify any relationships that exist among the lines, and then use
your TI-8* to graph the three equations in the same window.  Adjust
the viewing window so the slope appears visually correct.  Record
your window and appropriate scale.
\footnote{Larson CPC 97-100}
\begin{enumerate}
\item $y=\frac{2}{3}x$
\item $y=-\frac{3}{2}x$
\item $y=\frac{2}{3}x+ 2$
\end{enumerate}
\end{exercise}
\end{onlyproblem}

\begin{onlysolution}
$a$ is parallel to $c$ and $b$ is perpendicular to them both
\end{onlysolution}
\end{defproblem}

\begin{defproblem}{0301:GrapherB}%
\begin{onlyproblem}%
\begin{exercise}
Same as above
\begin{enumerate}
\item $y=2x$
\item $y=-2x$
\item $2y=x$
\end{enumerate}
\end{exercise}
\end{onlyproblem}

\begin{onlysolution}
Stuff
\end{onlysolution}
\end{defproblem}

\begin{defproblem}{0301:GrapherC}%
\begin{onlyproblem}%
\begin{exercise}
Same as above
\begin{enumerate}
\item $y+8=x$
\item $y=x+1$
\item $y+x=3$
\end{enumerate}
\end{exercise}
\end{onlyproblem}
\begin{onlysolution}
$a$ is parallel to $b$ and $c$ is perpendicular to both
\end{onlysolution}
\end{defproblem}

\begin{defproblem}{0301:GrapherD}%
\begin{onlyproblem}%
\begin{exercise}
Same as above
\begin{enumerate}
\item $2y=-x$
\item $2y+x=3$
\item $y+4=2x$
\end{enumerate}
\end{exercise}
\end{onlyproblem}
\begin{onlysolution}
thingy
\end{onlysolution}
\end{defproblem}

\begin{defproblem}{0301:TFA}%
\begin{onlyproblem}%
\begin{exercise}
True or False: A line with derivative $-\frac{5}{7}$ is steeper than a line with a derivative of $-\frac{6}{7}$.\footnote{Larson PCP 127-}
\end{exercise}
\end{onlyproblem}
\begin{onlysolution}
False.  Steepness is measured by the absolute value of the slope/derivative.
\end{onlysolution}
\end{defproblem}

\begin{defproblem}{0301:TFB}%
\begin{onlyproblem}%
\begin{exercise}
True or False: The line through (-8,2) and (-1,4) and the line through (0,-4) and (-7,7) are parallel.
\end{exercise}
\end{onlyproblem}
\begin{onlysolution}
False.
\end{onlysolution}
\end{defproblem}

\begin{defproblem}{0301:TFC}%
\begin{onlyproblem}%
\begin{exercise}
True or False: Any two distinct points on a non-vertical line can be used to calculate the slope of the line.
\end{exercise}
\end{onlyproblem}
\begin{onlysolution}
True.
\end{onlysolution}
\end{defproblem}

\begin{defproblem}{0301:TFD}%
\begin{onlyproblem}%
\begin{exercise}
True or False: It is impossible for two lines with positive slopes to be perpendicular.
\end{exercise}
\end{onlyproblem}
\begin{onlysolution}
True.
\end{onlysolution}
\end{defproblem}


\endinput
