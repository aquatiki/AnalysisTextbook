%!TEX root =  ../main.tex
\renewcommand{\columnseprule}{1.5pt}
\begin{multicols*}{2}
\rule[0.5\baselineskip]{0.4\textwidth}{1pt}
\noindent
\ExSection\label{sec:0301x}
\begin{exercises}{sec:0301x}
\prob[0301ParaPerpA] Write the slope-intercept forms of the equations of the lines
passing through the given point and a) parallel to the given
line and b) perpendicular to the given line.\footnote{Larson CPC
87-96}
\subprob $(-3,2), x+y=7$
\subprob $(\frac{7}{8},\frac{3}{4}), 5x+3y=0$
\subprob $(2,5), x=4$
\subprob $(2,1), y=2$
\subprob $(-3.9,-1.4), 6x+2y=9$


\prob[0301ParaPerpB] Same as above.
\subprob $(2,1), 4x-2y=3$
\subprob $(-\frac{2}{3},\frac{7}{8}), 3x+4y=7$
\subprob $(-1,0), y=-3$ 
\subprob $(5,-3), x=-2$
\subprob $(2.5,6.8), x-y=4$

\prob[0301GrapherA] Identify any relationships that exist among the lines, and then use
your TI-8* to graph the three equations in the same window.  Adjust
the viewing window so the slope appears visually correct.  Record
your window and appropriate scale. \footnote{Larson CPC 97-100}
\subprob $y=\frac{2}{3}x$
\subprob $y=-\frac{3}{2}x$
\subprob $y=\frac{2}{3}x+ 2$


\prob[0301GrapherB] Same
\subprob $y=2x$
\subprob $y=-2x$
\subprob $2y=x$


\prob[0301GrapherC] Same
\subprob $y+8=x$
\subprob $y=x+1$
\subprob $y+x=3$


\prob[0301:GrapherD] Same
\subprob $2y=-x$
\subprob $2y+x=3$
\subprob $y+4=2x$


\prob[0301TFA] True or False: A line with derivative $-\frac{5}{7}$ is 
steeper than a line with a derivative of $-\frac{6}{7}$.\footnote{Larson PCP 127-}


\prob[0301TFB] True or False: The line through (-8,2) and (-1,4) and the 
line through (0,-4) and (-7,7) are parallel.


\prob[0301TFC] True or False: Any two distinct points on a non-vertical line 
can be used to calculate the slope of the line.


\prob[0301TFD] True or False: It is impossible for two lines with 
positive slopes to be perpendicular.

\end{exercises}
\end{multicols*}