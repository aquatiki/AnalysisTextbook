%!TEX root =  ../main.tex

\subsection{Local Linearity}

\objective{Define derivatives and find numerical approximations}


When two variables grow proportionally to each other, they obey the algebraic relationship
$y=ax$.  This means when one goes up, the other goes up, and visa versa.  But many things
in nature are \emph{inversely} proportional to each.  \index{proportion!inverse}
That is, when one goes up, the other goes
\emph{down}.  This is the algebraic relationship $y=\frac{a}{x}$.

\personfeature[-1in]{\chapdir/pics/Henri-Poincare}{Jules Henri Poincar\'{e}
    }{1854-1912, French}{was a French mathematician and physicist
    who discovered many amazing facets of modern physics and non-
    traditional mathematics.  He was true polymath of his time, and
    never worked very long on a problem, relying (successfully)
    on his subconscious to keep working on the ideas.}

The fundamental postulate of calculus is that smooth, continuous functions are \textbf{locally
linear}, that is, if you zoom in far enough, they all appear as lines.  Algebraically, we should be able
to construct a tangent line to any function, using the difference quotient.  Chaos Theory, and
Fractal Geometry oppose this precept, in much the same way that Non-Euclidean Geometries
are build from a rejection of Euclid's Parallel Postulate (i.e., that parallel lines never meet).  Euclidean
Geometry and Calculus may be two viewpoints among many, but they are certainly useful ones to know!

\subsection{Derivative}
\index{Derivative!definition}
For all the aura surrounding calculus as the \textit{summum bonum} --- highest good --- it is not
some process to hard to just grasp.  We just did it graphically.  Algebraically, it is the difference 
quotient, an tiny step away from the point producing a line which therefore touches the
curve at only one point.  This line gets more and more accurate, the smaller our step away becomes.


\subsubsection{Theorems of Derivatives}

\begin{itemize}
\item[sum] the derivative of a sum is the sum of the derivatives 
\item[scalar] the derivative of a constant times a function is a constant times a derivative
\item[constant] the derivative of a constant is 0
\end{itemize}

Do no assume beyond this!  Note that the first two allow you to build a ``difference property'' too.
