%!TEX root =  ../main.tex

\subsection{Properties of Derivatives}
\begin{objective}
Calculate derivates using the Power, Product, and Quotient Rules.
\end{objective}

\index{Derivative!properties}
The derivative of a sum of two functions is the sum of the derivatives of each function.

The derivative of a constant times a function is the constant times the derivative of the function.

The derivative of a constant is 0.

\subsubsection{Product Rule (inductive)}
\index{Product Rule}\index{Derivative!of a product}
The following proof is very arbitrary and does not feel like something anyone would try
unprompted (especially in the second line: who would ever think of adding and subtracting the same
term in just that way?!).  We will use the following \textbf{Product Rule} some over the coming chapters,
but a full proof that you are responsible for reproducing does not come until §8.3

\begin{align*}
  (fg)'(x) & = \lim_{h\rightarrow0}\frac{f(x+h)g(x+h) - f(x)g(x)}{h} \\
  	&= \lim_{h\rightarrow0}\frac{f(x+h)g(x+h) - f(x)g(x+h) + f(x)g(x+h) + f(x)g(x)}{h} \\
	&= \lim_{h\rightarrow0}\frac{f(x+h)g(x+h) - f(x)g(x+h)}{h} + \lim_{h\rightarrow0}\frac{f(x)g(x+h)-f(x)g(x)}{h} \\
	&= \lim_{h\rightarrow0}\left[\frac{f(x+h) - f(x)}{h} \cdot g(x+h)\right] + \lim_{h\rightarrow0}\left[\frac{g(x+h)-g(x)}{h} \cdot f(x)\right] \\
	&= \lim_{h\rightarrow0}\left[\frac{f(x+h) - f(x)}{h}\right] \cdot g(x) + \lim_{h\rightarrow0}\left[\frac{g(x+h)-g(x)}{h}\right] \cdot f(x)\\
	&=f'(x)g(x) + f(x)g'(x)
\end{align*}

\subsubsection{Quotient Rule}
\index{Quotient Rule}\index{Derivative!of a quotient}
The \textbf{Quotient Rule} is presented here with a very similar arbitrary trick.  Wait to memorize the proof until chapter 8.

\begin{align*}
\frac{f}{g}' & = \lim_{h\rightarrow0}\cfrac{\frac{f(x+h)}{g(x+h)}-\frac{f(x)}{g(x)}}{h} \\
	&=  \lim_{h\rightarrow0}\frac{g(x)f(x+h)-f(x)g(x+h)}{g(x)g(x+h)h} \\
	&=  \lim_{h\rightarrow0}\frac{g(x)f(x+h)-f(x)g(x)+f(x)g(x)-f(x)g(x+h)}{g(x)g(x+h)h} \\
	&=  g(x)\left[\lim_{h\rightarrow0}\frac{1}{g(x)g(x+h)}\cdot{}\frac{f(x+h)-f(x)}{h}\right]-f(x)\left[\lim_{h\rightarrow0}\frac{1}{g(x)g(x+h)}\cdot{}\frac{g(x+h)-g(x)}{h}\right] \\
	&= \frac{g(x)f'(x) - f(x)g'(x)}{[g(x)]^2}
\end{align*}

\subsection{Power Rule}
\index{Power Rule}\index{Derivative!of a power}
The algebraic derivative of $x^n$ is $n\cdot x^{n-1}$.  While we will furnish a systematic proof 
later (also in section 8.3), we have seen enough inductive
examples of the \textbf{Power Rule} in this chapter
that you are formally responsible for knowing it and using it in all cases.