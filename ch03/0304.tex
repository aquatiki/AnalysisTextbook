%!TEX root =  ../main.tex

\subsection{Babylonian Method}

\objective{Graph and understand the slope of root functions.}

Square roots are useful whenever someone tell you the square footage of an area.  If it 
were a square, how long would each side be?  Cube roots are similar for volume. 

Dealing with square roots without calculator can be rather intimidating.  But it doesn't have
to be.  3,000 years ago, the Babylonians found a method that truly repays the effort put into it:
each iteration double the number of digits in the estimate!  Few formula converge so quickly.

The square root of $A$:

$R_{n+1}=\cfrac{R_n + \frac{A}{R_n}}{2}$

In short:
\begin{enumerate}
\item Pick the best number you can.
\item Divide the original by ``the number''
\item Average the answer with ``the number''
\item Return to step 2 with this as ``the number''
\end{enumerate}


\subsection{Even vs. Odd}
OK, so you found the square or cube root of some number to as many decimal places as
you need.  Rather than doing \emph{that} more than once, wouldn't it be preferable to find
out how much adding to the inside of a square root changes the output?  In other words,
what is the rate of change, or the derivative?

You will need to multiply top and bottom of the difference quotient by the conjugate of the
numerator.


\begin{example}
\exProblem
A woman rode a train where the cost of train ride is directly proportional to the square 
root of the distance ridden.  Her ticket to ride 140 miles cost \$24.40.  
Find the final per mile rate of a man who got off after 35 miles, 
compared the woman's final rate.

\exSolution
Directly proportional means we can set up an equation $c = k \sqrt{d}$, where $c$ and $d$
are the cost and distance.  If $24.4 = k\sqrt{140}$ then $k=\frac{24.4}{\sqrt{140}}$.  The
derivative of $\frac{24.4}{\sqrt{140}}\sqrt{x} = \frac{12.2}{\sqrt{140x}}$.  At 35, the
derivative is equal to $\frac{24.4}{sqrt{140}} \approx 0.17$, and at 140 it is 
$\frac{12.2}{\sqrt{140}} \approx 0.09$.  This means the woman exited paying over
\$0.08 less per mile than the man.
\end{example}

~\vfill
