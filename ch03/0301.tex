%!TEX root =  ../main.tex
\subsection{Properties of Lines}

\objective{Classify linear and linear-like functions, and explain relationships of slopes.}


All non-vertical lines have slope, which is defined as rise over run, $\frac{\Delta x}{\Delta y}$ 
or $m$.  The initial value
of a linear equation might not be zero, but an ordered pair of the form $(0,b)$.  The variable
$b$ is called the $y-$intercept.  All together, this is the famous $y=mx+b$ form of lines.

\marginfig[-0.1in]{ceiling_function}{The ceiling function.}
However, that formula only makes sense when the origin is within view and we might become 
concerned with the output-value when the input is 0. In calculus and other areas of mathematics, 
we are more
often simply interested in the slope and some arbitrary point, $(x_1,y_1)$.  Hence we find a majority
of time from here on out, the point-slope form of \index{linear!point-slope form}
linear equation is most helpful, $y-y_1=m(x-x_1)$,
a form which allows us to see the slope, and a (random?) point on the line very easily.
This form is nearly as easy to put in your TI-8* as your old friend slope-intercept: simply add $y_1$
to the other side, and you have a $y=$ form ready to be entered in your grapher.

\subsection{Parallel vs. Perpendicular}
If a line has slope expressed via the fraction in lowest terms $\frac{a}{b}$, 
then any other fraction that reduces to the same ratio will produce a line parallel to the first.  
Consider the sketch of how to make a line rotated $90^\circ$ clockwise or counterclockwise 
to the first.  They will have slopes of $\frac{-b}{a}$ or $\frac{b}{-a}$, which are the same
things.  In short, perpendicular lines have opposite reciprocal slope.\index{linear!parallel}
\index{linear!perpendicular}

\subsection{Linear-like}
Several functions are linear in pieces, and are used in computer programming and other
systems of functions

\subsubsection{Ceiling, Floor, and Round}
Various forms of rounding are present in computer systems and your TI-8*.  Rounding down
in all cases, rounding up in call cases, and the familiar rounding to the closest.

\marginfig[0in]{floor_function}{The floor function}
\subsubsection{Modulus}
The most used function in this respect, and the foundation of an entire species of mathematics
(called Modular Arithmetic) is the modulus function.  It can be thought of as taking two arguments,
one is what to divide by, and the other is what to divide.  The function returns the remainder.  For
example, 10 mod 3 is 1, and 49 mod 7 is 0.  Consider the graph of y=x mod 5.

\subsubsection{Rates of Change}
A constant function is one that never changes (by definition).  \index{constant!derivative}
Algebraically, that means it is
of the form $y=c$, where $c$ is some numbers.  The slope is always zero and the graph
is always a horizontal line.  A vertical line is not a function and its slope is undefined.  Every
other line has a constant rate of change, it's slope.  In calculus terminology,
it's derivative is a constant.






