%!TEX root =  ../main.tex
\renewcommand{\columnseprule}{1.5pt}
\begin{multicols*}{2}
\rule[0.5\baselineskip]{0.4\textwidth}{1pt}
\noindent
\LabSection{Classic Ladder Problem}\label{sec:0304p}
\begin{exercises}{sec:0304p}
\lab{} Suppose you have a 10 ft ladder leaning up against a wall.  Draw a sketch of the scene, using $x$ for the distance of the base of the ladder from the wall, and $y$ for the height of the ladder up from the floor.


\vspace{3cm}
\lab{} Relate the length of the ladder and $x$ and $y$ in an equation.  Solve the equation for $y$.


\vspace{3cm}
\lab{} You are struck with the morbid curiosity of the mathematician!  You want to know how fast the height of the ladder is changing as you move the base of the ladder through the point 8 ft from the wall.  Set up the limit that will find this rate of change.


\vspace{2cm}
\lab{} You should have an instance of your old friend, the removable discontinuity.  Multiply the numerator and denominator by the conjugate of the numerator, but DO NOT distribute in the denominator.


\vspace{3cm}
\lab{} Factor a negative out of the numerator.  Trust us.

\vspace{2cm}
\lab{} Factor and then cancel the like term in the numerator and denominator.  You should now be able to plug in 8 and get you the derivative.  


\vspace{2cm}
\lab{} Be careful with the sign of your answer!  Think back to the real-life situation and decide what the sign of the derivative truly depends upon.  Write it down in here in a complete sentence.


\vspace{2cm}
\lab{} Find the derivative equation $f’(x)$ for $f(x)=\sqrt{x}$, using the full difference quotient limit.



\vspace{5cm}
\lab{} Rewrite the square root function and its derivative using fractional exponents and nothing in the denominator.


\vspace{3cm}
\lab{} Describe in technical vocabulary what you think the point of this problem set is.

\end{exercises}
\end{multicols*}
