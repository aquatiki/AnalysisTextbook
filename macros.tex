%!TEX root =  main.tex
\newcommand{\chapdir}{glub}

%
%		Chapter image boolean
%
\newif\ifusechapterimage
\usechapterimagetrue
\newcommand{\thechapterimage}{}%
\newcommand{\chapterimage}[1]{\ifusechapterimage\renewcommand{\thechapterimage}{#1}\fi}%
		

\newcommand{\mychapter}[2]{					% Start chapters and demarcate mini TOCs
\chapter{#1}
\label{ch:#2}
\startcontents[chapters]
}

\newcommand{\mychapters}[3]{					% Start chapters and demarcate mini TOCs
\chapterimage{#3}%
\chapter{#1}%
\label{ch:#2}%
\startcontents[chapters]%
\WriteChap{#1}%
}

\newcommand{\marginlessinput}[1]{%			% marginless input
\newgeometry{left=2cm,right=2cm,bottom=1cm,top=2cm}%
\input{#1}%
\restoregeometry%
}

\newcommand{\marginlesspdf}[1]{
\newgeometry{left=2cm,right=2cm,bottom=2cm,top=2cm}
\noindent\makebox[\textwidth]{\includegraphics[width=\paperwidth]{#1}}
\restoregeometry
}

%									mini toc
\newcommand\chapterminitoc{%
\printcontents[chapters]{}{1}{\setcounter{tocdepth}{2}}%
}

%									objective
\newcommand\objective[1]{%
\vspace{0.2cm}
\begin{derivation}{Objective}
#1
\end{derivation}
\vspace{0.2cm}
}

\newcommand\invisiblesection[1]{%
  \refstepcounter{section}%
  \addcontentsline{toc}{section}{\protect\numberline{\thesection}#1}%
  \sectionmark{#1}}




%----------------------------------------------------------------------------------------
%	Math specific stuff
%----------------------------------------------------------------------------------------
%
% https://tex.stackexchange.com/questions/37912/how-to-draw-the-parallel-circuits-sign
%
\newcommand{\pplus}{\mathbin{\!/\mkern-5mu+\mkern-5mu/\!}}
\newcommand{\pminus}{\mathbin{\!/\mkern-5mu-\mkern-5mu/\!}}


% Copyleft
\newcommand{\copyleft}{\reflectbox{\copyright}}

% long division
\newcommand\longdiv[2]{%
$\strut#1$\kern.25em\smash{\raise.3ex\hbox{$\big)$}}$\mkern-8mu
        \overline{\enspace\strut#2}$}
        
        
% new decimal symbols
\def\0{\mbox{\begin{picture}(11,12)(0,0)
\put(5.5,5){\circle{9}}
\end{picture}
}}
\def\6{\mbox{\begin{picture}(11,12)(0,0)
\put(1,0){\line(1,0){10}}
\put(1,0){\line(0,1){10}}
\put(1,10){\line(1,-1){10}}
\end{picture}
}}
\def\7{\mbox{\begin{picture}(11,12)(0,0)
\put(1,10){\line(1,0){10}}
\put(11,10){\line(0,-1){10}}
\end{picture}
}}
\def\8{\mbox{\begin{picture}(11,12)(0,0)
\put(1,0){\line(0,1){10}}
\put(1,0){\line(1,1){10}}
\put(1,10){\line(1,0){10}}
\end{picture}
}}
\def\9{\mbox{\begin{picture}(11,12)(0,0)
\put(1,0){\line(1,0){10}}
\put(11,0){\line(0,1){10}}
\end{picture}
}}
\def\5{\mbox{\begin{picture}(11,12)(0,0)
\put(5.5,0){\oval(8,20)[t]}
\end{picture}
}}
\def\1{\mbox{\begin{picture}(11,12)(0,0)
\put(1,0){\line(1,0){10}}
\put(1,0){\line(0,1){10}}
\end{picture}
}}
\def\2{\mbox{\begin{picture}(11,12)(0,0)
\put(1,10){\line(1,0){10}}
\put(11,10){\line(0,-1){10}}
\put(1,10){\line(1,-1){10}}
\end{picture}
}}
\def\3{\mbox{\begin{picture}(11,12)(0,0)
\put(1,0){\line(0,1){10}}
\put(1,10){\line(1,0){10}}
\end{picture}
}}
\def\4{\mbox{\begin{picture}(11,12)(0,0)
\put(1,0){\line(1,0){10}}
\put(11,0){\line(0,1){10}}
\put(1,0){\line(1,1){10}}
\end{picture}
}}

%%%%%%%%%%%%%%%%
%
%        TRIANGLE OF POWER
%
% https://tex.stackexchange.com/questions/307833/how-to-represent-the-triangle-of-power-in-latex
% 
%%%%%%%%%%%%%%%

\newcommand{\dotriangle}[1]{%
  \raisebox{-.7ex}{$\vcenter{#1\kern.2ex\hbox{$\triangle$}\kern.2ex}$}%
}

\newcommand{\tripow}[3]{% Syntax: \tripow{#1}{#2}{#3} gives you #1 ^ {#2} = #3
   \mathop{% We want it to an operator
      \mathchoice% We want different functionality in text and display mode
         {% DISPLAY MODE
            \vphantom{\dotriangle\LARGE}% \vphantom off-set: places the bottom entries.
            \rule[-1.4ex]{0.1em}{0pt}% Syntax: [<vetical drop #1>]{<left margin>}{<Should be 0>}
            _{\scriptstyle #1}% style of #1 entry
            {\overset{\scriptstyle #2}% style of #2 entry
            {\dotriangle\LARGE}}% Size of the displayed operator - should match the \vphantom off-set.
            \rule[-1.4ex]{0em}{0pt}% Syntax: [<vetical drop #3>]{<Should be 0>}{<Should be 0>} 
            _{\scriptstyle #3}% style of #3 entry
            \rule[0ex]{0.1em}{0pt}% Syntax: [<Should be 0>]{<right margin>}{<Should be 0>}
          }%
        {% TEXT MODE
            \vphantom{\dotriangle\normalsize}%
            \rule[-1.05ex]{-0.7ex}{0pt}%
            _{#1}% 
            \overset{#2}% 
            {\dotriangle\normalsize}% size in text mode
            \rule[-1.05ex]{0pt}{0pt}%
            _{#3}%
            \rule[0ex]{-0.2em}{0pt}%
          }%
        {% SCRIPT MODE
            \vphantom{\dotriangle\normalsize}%
            \rule[-1.05ex]{-0.8ex}{0pt}%
            _{\scriptstyle #1}%
            {\overset{\scriptstyle #2}%
            {\dotriangle\normalsize}}% size in script mode
            \rule[-1.05ex]{0pt}{0pt}%
            _{\scriptstyle #3}%
            \rule[0ex]{-0.3em}{0pt}%
         }%
        {}% SCRIPTSCRIPT MODE
   }%
}

