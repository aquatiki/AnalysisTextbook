%!TEX root =  ../main.tex


\mychapters{Radices}{radix}{\chapdir/pics/All_In_A_Spin_Star_trail} 

\newpage
\chapterminitoc

%									16 - 1
\newpage
\section{Base 10}\label{sec:base10}
\subsection{Problems}
Harry Potter money
inch, feet, yard, mile
cm, m, km, pole-to-pole
\subsection{Bases}
We don't often evaluate our number system, unless we encounter another one.  The Chinese, for
example, have no symbol for zero, and historically did not have ``places''.  234 is not written just
with the characters for 2, 3, and 4, but as ``2-hundred-3-ten-4''.  Anyone who has studied French
is to be pitied for the effort they had to extend in understanding French numerals, which involve
20's and subtraction!  There are even languages from around the world that keep counting past
their 10 fingers onto various body parts, up the arm and all around.

Before we delve into other systems, we need to understand our own.  What does 593.81 mean, 
and how is it related to 5, 9, 3, 8, and 1?  Single-digit numbers are not hard to comprehend, but
have you thought through how many of them there are?  You might have said nine, but there are
in fact ten.  Zero is often forgotten but plays a crucial role in our system.  Commas are common
outside of mathematics, but are there to break up large numbers and make them easier to read,
not distinguish numbers from numbers.  But there is one other symbol which is vital: the decimal
place.

Our numeral system is based on 10.  Once we've used all our numbers counting from zero to nine,
we resort to adding another digit to the left, to indicate how many times we have completed a cycle
of ten.  For example, 56 means $5\times10 + 6\times1$.  Each new digit to the left tells you about 
the next power of 10.  Obviously, 100 is $10\times10$, 1000 is $10\times100$, etc.  An eight-digit
number, therefore, uses its left-most digit to indicate how many $10^7$s it has.

\subsection{New Symbols}
Before we get to other bases, it will be good practice to loosen up our rigid obedience to Arabic 
numerals.  We, therefore, propose the following new symbols for 0 to 9:

\0, \1, \2, \3, \4, \5, \6, \7, \8, \9

\ExSection[Exercises]
Requires logs
\begin{exercises}{sec:base10}
\prob Find the number of digits in $2^{123}$
\prob find the pattern in the times tables, modulo 10
\prob find the pattern in the exponents table, modulo 10
\end{exercises}

%									16 - 2
\newpage
\section{Powers of 2}
\subsection{Problems}
Bits of computer number
\newpage

Computers are made out of circuitry which has only two possible states: electricity-flowing or
electricity-not-following.  These are represented via the symbols 0 and 1, called `binary'.  
To indicate whether we are using these symbols in their traditional, base-10 way, or in 
this new, binary way, we write which base we are using as a base.

How can a computer add $1_2+1_2$?  There is no third state --- nothing other than 1's and 0's
to use --- so it must resort to another columns of numbers, namely $10_2$.  

\subsection{Binary, Quaternary, Octal }
Some aboriginal tribes in South America count not with their fingers, but with the spaces between
their fingers.  This can result in a number system that is not base-5, but base-4, called quaternary.
It has a lot in common with binary, only twice as many symbols.

The next power of two produces a system called octal, and the one after that is called hexadecimal
(16).  This system use more than the 10 symbols we have, and commonly uses A-F to extend the
count.  This leads to counting like 0,1,2,3,4,5,6,7,8,9,A,B,C,D,E,F,$10_{16}$.

\subsection{Exercise}
tetrasexigesimal
\subsubsection{Ternary}
We have said that binary became popular as a result of computer circuitry, in which 
electricity can either flow or not.  But in the Soviet Union, an attempt was made to build
computers on another system, one where electricity might flow, not flow, or \emph{flow
backwards}.  This was called ``balanced ternary'', with `ternary' meaning base-3.




%									16 - 3
\newpage
\section{Duodecimal and Sexagesimal}
Our clocks are base-12 because it is the number with the most divisors under 50.  Our minutes, 
seconds, and degrees are base-60, because the babylonians knew it was the number with the most
number of divisors under 100.
\subsection{Exercises}
Clock math progressing into modular arithmetic



%									16 - 4
\newpage
\section{Phi and 2i}
Donald Knuth invented a system of base 2i.

The golden ratio, $\dfrac{1+\sqrt{5}}{2}$ as a base?  It has the most amazing property, that $\phi^2
=\phi+1$.
\subsection{Exercises}
Constructing 2i numbers
constructing phi numbers


%									16 - 5
\newpage
\section{Diophantine Equations}
\subsection{Euclid's Algorithm}
\subsection{Exercises}

%									16 - 6
\newpage
\section{Review}
\subsection{Chapter Review}
\subsection{Chapter Test}
\subsection{Cumulative Review}
\subsection{Cumulative Test}


