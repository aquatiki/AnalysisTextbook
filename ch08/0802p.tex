%!TEX root =  ../main.tex
\renewcommand{\columnseprule}{1.5pt}
\begin{multicols*}{2}
\rule[0.5\baselineskip]{0.4\textwidth}{1pt}
\noindent
\LabSection{It Don't Stop}\label{sec:0802p}
\begin{exercises}{sec:0802p}
\lab{} Seemingly grouped between the Equator and the Tropic of Capricorn, there exist languages in South America, Africa, and Australia without words for numbers after two.  Speculate how could speakers of these language keep track of their belonging numbering three or more?


\vspace{5cm}
\lab{} In our civilization, we have scientific notation (which can handle a lot), but still breaks down for very large, finite numbers.  Graham’s Number, for instance, is so large, even the number of digits could not be written down across the length of the whole universe.  What are some mathematical processes you know that could lead to astronomically large numbers?

\vspace{4cm}
\lab{} In the previous homework, you read about Cantor’s descriptions of infinities, the smallest of which is ``Countable Infinity'', or the Infinity of the Counting numbers.  Explain how there are as many even numbers as there are whole numbers, because of correspondence.

\vspace{4cm}
\lab{} Countable infinity is labeled $\aleph_0$, Aleph-null.  The next infinity is $\aleph_1$, the number of Irrationals, or Reals.  Explain why the Rationals are $\aleph_0$, not $\aleph_1$.

\vspace{4cm}
\lab{} Human language is incredible because after mastering a finite number of words, we are capable of produce an infinite variety of unique sentences.  Write a grammatical (though not necessarily sensible) English sentence that no one has ever made before, of 6 words or less.

\vspace{5cm}
\lab{} Which infinity is the number of possible English sentences (which are drawn from a finite vocabulary)?  Why?

\vspace{4cm}
\lab{} In technical vocabulary, describe why infinity is so hard to discuss and/or why our intuitions about it are not trustworthy.


\end{exercises}
\end{multicols*}