%!TEX root =  ../main.tex
\renewcommand{\columnseprule}{1.5pt}
\begin{multicols*}{2}
\rule[0.5\baselineskip]{0.4\textwidth}{1pt}
\noindent
\LabSection{To Infinity, and Beyond}\label{sec:0805p}
\begin{exercises}{sec:0805p}
\lab{} We have been careful to only use the infinity sign ($\infty$) in limits problems.  Create two different kinds of limits problems and answer them, one where the question involves limits and one where the answer is infinity.

\vspace{5cm}
\lab{} We have also solved indefinite and definite integrals.  Explain what $\int_1^{100}\frac{1}{x}dx$ means and answer it.

\vspace{5cm}
\lab{} There are some functions that we do not know the integral of, but are easy to figure out.  What is $\int_{-3}^{3}|x|dx$ asking for?  Sketch a graph and calculate the answer using geometry, not calculus.

\vspace{6cm}
\lab{} Let us consider a weirder integral: $\int_1^{\infty}\frac{1}{x}$. First, evaluate the anti-derivative and mark the ends we must evaluate, following this example: $\int_a^b f{x}dx=\left. F(x)\right|_a^b$.

\vspace{5cm}
\lab{} Now, take $\infty$ as shorthand for $\displaystyle\lim_{x\rightarrow\infty}F(x)$ and write the problem as a function with argument minus a limit of a function.  Solve.

\vspace{5cm}
\lab{} Write in technical terminology using whole sentences what you think the point of this problem set it.
\end{exercises}
\end{multicols*}