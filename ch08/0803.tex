%!TEX root =  ../main.tex

\subsection{Deriving}

\objective{Prove the Power Rule, Product Rule, and Quotient Rule, and apply them to arbitrary derivatives}


Coming back to Earth for a section from all this contemplation of infinities, we see the
extreme usefulness of it all.  The number $e$ is not the answer to any algebraic 
expression.  That is, one cannot build a polynomial --- even an irregular one with
rational coefficients and rational exponents --- that has $e$ as a solution.  $e$ is a
transcendental number, the result of an infinite process, though it itself is a finite number.
There is a very important sense in which $e$ contains an infinity within itself.

It takes a system like calculus to create $e$, and it returns the favor by unleashing a
vast reservoir of new waters for calculus to navigate.  The derivative of $e^x$ is $e^x$,
and its inverse --- $\ln{x}$ --- is just as amazing.  The inverse of $y=e^x$ can be written
as $x=e^y$.  By implicit differentiation, $dx = e^ydy$, and therefore $\frac{dy}{dx} =
\frac{1}{e^x}$.  Well, we began by saying $e^y=x$, so the derivative of $\ln{x}$ must
be $\frac{1}{x}$.  Hopefully, you have been curious for several chapters what
could ever have the derivative of $\frac{1}{x}$, since nothing could ever make that
via the Power Rule.

\personfeature[-2in]{\chapdir/pics/Leibniz_Hannover}{Gottfried Wilhelm Leibniz}{1646 -
1716}{was a German polymath and philosopher who occupies a prominent place in the history of mathematics and the history of philosophy, having developed differential and integral calculus independently of Isaac Newton. Leibniz's notation has been widely used ever since it was published. It was only in the 20th century that his Law of Continuity and Transcendental Law of Homogeneity found mathematical implementation (by means of non-standard analysis).
\href{https://en.wikipedia.org/wiki/Gottfried_Wilhelm_Leibniz}{Wikipedia}}

\subsection{Proofs}
Speaking of the Power Rule, we have been using it as we made it via Induction for 
some time now.  Not that Induction is a bad thing, but ln affords us a more elegant 
proof.  Suppose $y$ is defined as some function to a power:
$$
y = \left(f(x)\right)^n
$$
There is no immediately obvious way to differentiate the left side of this identity, but
what if we take the log of both sides first?  This is called the \textbf{logarithmic deriviative}.
$$
\ln{y} = \ln{\left(f(x)\right)^n}
$$
A happy consequence comes from the fact that a log of a power is the same as the
log times the power.
$$
\ln{y} = n\cdot\ln{f(x)}
$$
Now we can take the derivative of both sides implicitly.
$$
\frac{1}{y} \cdot y^\prime = n \cdot \frac{1}{f(x)}
$$
Solving for $y^\prime$, we get
$$
y^\prime = n \frac{y}{f(x)}
$$
If we substitute back in the original definition of $y$ (i.e. $\left(f(x)\right)^n$) and simplify,
we see
\begin{equation}
y^\prime = n\cdot{}\left(f(x)\right)^{n-1}
\end{equation}
the generalized Power Rule.


\subsection{Product and Quotient Rule}
The Product Rule can also be proven by Logarithmic Derivative, without limits, for any 
$y = f\cdot{}g$.
\begin{align*}
\ln{y}  = & \ln{f \cdot{} g}  \\
\frac{1}{y}y^\prime =& \left(\ln(f)\right)^\prime + \left(\ln(g)\right)^\prime \\
y^\prime =& y(\frac{1}{f}f^\prime + \frac{1}{g}g^\prime)  \\
 =& f\cdot{}g(\frac{1}{f}f^\prime + \frac{1}{g}g^\prime)  \\
\end{align*}
\begin{equation}
  =  g\cdot{}f^\prime + f\cdot{}g^\prime
\end{equation}

The same goes for the Quotient Rule, for any $y=\frac{f}{g}$.
\begin{align*}
\ln{y} &= \ln{\frac{f}{g}} \\
\frac{1}{y}y^\prime &= \left(\ln{f} - \ln{g}\right)^\prime \\
y^\prime &= y\left(\frac{1}{f}f^\prime - \frac{1}{g}g^\prime\right) \\
  &= \frac{f}{g}\left(\frac{f^\prime}{f} - \frac{g^\prime}{g}\right) \\
  &= \frac{f^\prime}{g} - \frac{g\cdot{}f^\prime}{g^2}\\
\end{align*}
\begin{equation}
  = \frac{g\cdot{}f^\prime - g\cdot{}f^\prime}{g^2}
\end{equation}

\subsection{Derivative Review}
Let us summarize all of the derivative shortcuts we have discerned.  You are 
responsible to prove without assistance all of these

We will use $u$ as a variable of differentiation because there might be a
(nested?) set of Chain Rules to apply before we get down to $x$.  Assuming
$u$ is a function of $x$:
\begin{equation}
\frac{d}{dx}u = \frac{du}{dy}\cdot{}\frac{dy}{dx}
\end{equation}

We have been given a looping cycle of trigonometric derivatives without proof:
\begin{equation}
\left(\sin{u}\right)^\prime = \cos{u}\frac{du}{dx}
\end{equation}
\begin{equation}
\left(\cos{u}\right)^\prime = -\sin{u}\frac{du}{dx}
\end{equation}
\begin{equation}
\left(-\sin{u}\right)^\prime = -\cos{u}\frac{du}{dx}
\end{equation}
\begin{equation}
\left(-\cos{u}\right)^\prime = \sin{u}\frac{du}{dx}
\end{equation}
Any polynomial or power function can be differentiated with the Power Rule:
\begin{equation}
\left(u^n\right)^\prime = n\cdot{}u^{n-1}\frac{du}{dx}
\end{equation}
Any exponential function can be differentiated as follows:
\begin{equation}
(b^u)^\prime = b^u\cdot{}\ln{u}\frac{du}{dx}
\end{equation}
Any logarithmic function can be differentiated as follows:
\begin{equation}
\left(\log_a{u}\right)^\prime = \frac{1}{\ln{a}}\cdot{}\frac{1}{u}\cdot{}\frac{du}{dx}
\end{equation}

Together with the Product and Quotient Rule above, almost any major
function should be differentiable for you now.



