%!TEX root =  ../main.tex

\subsection{Real Analysis}

\objective{Apply the basic principles of Analysis to solve improper integral and use hyperreal numbers}


An \textbf{improper integral} is a definite limit where one and/or the other bound is
an infinity.  For example, 
$$
\int_0^\infty e^{-x}dx
$$
In this case, we integrate like normal, but take the upper bound to be a limit, 
written more simply.  (No one wanted to write a limit on top of an integral!)  
$$
\int_0^\infty e^{-x}dx = \left.-e^{-x}\right|_0^\infty = \lim_{x\rightarrow\infty} -e^{-x} - -e^0
$$
The limit approaches 0 while the function evaluates to -1.  We say, therefore, that
the entire integral is assigned a value of 1.  What are we saying?  The area under
an infinite curve is a finite number?  Remarkable.

\subsubsection{Special Functions}
Some functions come up so often in higher mathematics that they are given
their own name, often involving Greek letters.  Many of them have wildly 
complicated definitions, and not a few are improper integrals.  One such
function is the Gamma function, which is defined thus:

\begin{equation}
\Gamma(z)=\int_{0}^{\infty}x^{z-1}e^{-x}dx
\end{equation}

We shall see in chapter 14 that there is very common function in math called
\textbf{factorial}.  The Gamma function is the \textbf{analytic continuation} of
the factorial function, such that $\Gamma(n) = (n-1)!$.  This has many uses in
probability theory, allowing us to calculate values in between the integers.


\subsection{Non-Standard Analysis}
When Leibniz and Newton invented Calculus in the mid-17th century, their discoveries were
a culmination of two centuries of fierce debate over the idea of infinitesimals.  Was the
continuum (we might use the were ``number line'' or Reals) made up on indivisible, smallest
elements or not?  Leibniz called the infinitesimals and Newton called them fluxions.  
The assumption of their existence had made it possible to grow beyond a dependence
upon Euclid and the Classics, and enabled the ability to make new discoveries in 
mathematics, along the lines of Analytic Geometry (the algebra of Geometry).


Unfortunately, Newton and Leibniz did not cultivate a rigor science of infinitesimals,
and later centuries have much preferred the work of Cauchy and Weierstrauss, 
because of its consistency and explanatory power.  However, given the antiquity of
the method of \textbf{exhaustion} and current understand of Planck Length, 
Planck Time, Planck Mass, etc., this textbook has attempted to keep some measure
of the method of infinitesimals at its core.  These methods were made rigorous in
the 1960s by Abraham Robinson and codified in a system of numbers known as 
\textbf{Hyperreal} Numbers, or simply, the Hyperreals.

The hyperreals are an extension of the reals, as the name implies.  We need only
posit two numbers: $\omega$ and $\epsilon$.  $\omega$ (the last letter of the
Greek alphabet, only always lowercase, pronounced oh-MEG-uh or oh-MAY-guh)
is analogous to infinity as most people think of it.  Infinity is \emph{supposed}
to be a concept, not a number, the idea of going on without end.  $\omega$ is
a number bigger than any real number, just smaller than infinity.  $\epsilon$ is
its reciprocal, a positive number smaller than any real number but not zero.

This leads to a system where numbers may be composed of two (even three)
parts.  For example, $3+2\epsilon$ may not be simplified any further.  But 
$\omega(3+2\epsilon)$ can be: $3\omega + 2$.  What good is such an arbitrary,
new system.  Well actually, its quite old and caused the invention of calculus!

\subsubsection{Old Definitions}
The Difference Quotient can be stated without limits using hyperreals as
\begin{equation}
f'(x) = \frac{f(x+\epsilon) - f(x)}{x+\epsilon -x}  \outnote{(Hyperreal Difference Quotient)}
\end{equation}

$e$ can be stated without limits as follows:
\begin{equation}
e = (1+\epsilon)^\omega  \outnote{(Hyperreal Definition of $e$)}
\end{equation}

\subsubsection{Derivative}
Let us do one example, with several others in the exercises, finding the derivative of 
$x^2$.
$$
(x^2)' = \frac{(x+\epsilon)^2-x^2}{\epsilon} = \frac{x^2+2x\epsilon+\epsilon^2-x^2}{\epsilon}
= 2x+\epsilon
$$
Unlike limits notation, the infinitesimal does not disappear.  Every $\epsilon$ step in
the $x$-direction produces a jump of $2x$ in the $y$-direction.  According to quantum
mechanics (and Zeno's Paradox), the jumps are done without intervening steps,
that at the smallest level, elementary particles proceed in a way quite unlike
our macroscopic world.
