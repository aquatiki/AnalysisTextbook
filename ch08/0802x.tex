\begin{defproblem}{0802:Hotel}
\begin{onlyproblem}
Hilbert's Hotel is a rather strange hotel: it has an infinite number of room.
Unfortunately, there is a guest in every room.
In the following situations, state the one instruction the manager must
give to the existing guests and the new guests to accommodate them all
and not have any room sharing.
\begin{enumerate}
\item A finite number ($n$) of guests arrive.
\item A infinite number of guests arrive.
\item An infinite number of busses arrive, each hold an infinite number of guests.
\item In infinite number of boats arrive in the harbor, each holding an infinite number of busses, each holding an infinite number of guests.
\end{enumerate}
\end{onlyproblem}

\begin{onlysolution}
\begin{enumerate}
\item have each of the existing guests move to their room number plus $n$
\item many solutions.  ex. have each existing guest move to 2 times his or her room number.  The new guests can fill in the odds
\item many solutions.  ex. having  numbered each bus with a prime number starting with 3 (call it $P_n$), and having numbered each person the bus with a number (call it $m$), assign each new guest a room number $P_n^m$.  Have all the existing guests move from their room (call it $q$) to $2^q$.
\item many solution
\end{enumerate}
\end{onlysolution}
\end{defproblem}


\begin{defproblem}{0802:BFF}
\begin{onlyproblem}
Having studied polynomial long division, consider what happens when the answer
is infinite.  Divide 1 by $1-x$ to many places.  What pattern emerges?  Proceed by
adding one term add a time: at what point does the graph start to look like the original
problem?  What would you estimate the effective domain of your infinite approximation
is?
\end{onlyproblem}
\begin{onlysolution}
$1+x+x^2+x^3+x^4+x^5+\dots$.  Many answers, ex. by six terms it resembles the original
from $-1/2$ to $1/2$.  Even with vastly large numbers, it still only works from (-1,1).
\end{onlysolution}
\end{defproblem}




\begin{defproblem}{0802:Grandi}
\begin{onlyproblem}
Luigi Grandi was a monk and mathematician, and he looked a very boring but infinite sum.
It was $1 - 1 + 1 - 1 + 1 - 1 + 1 -1 \dots$.  What are two possibilities this sum alternates 
between?  According to Morris Kline, Grandi knew these possibilities were mutually
exclusive, so he look for a third option.  He considered that his infinite sum could
be approximating by $\frac{1}{x+1}$ as $x$ approaches 1.  What value do you think
he attributed to his infinite sum?  Decide whether you think he was right or wrong
and argue why.
\end{onlyproblem}
\begin{onlysolution}

\end{onlysolution}
\end{defproblem}



\begin{defproblem}{0802:Gab}
\begin{onlyproblem}
Gabriel's Horn is an imaginary trumpet, thought up by the mathematician Evangelista 
Torricelli, and supposedly belonging to the archangel Gabriel.  
We will simplify the math ever so slightly for our purposes here.  Consider the
equation $y=\frac{1}{x^2}$ in the first quadrant only, that is, with a domain of $x\in{[1,\infty)}$.
(In the original problem, this curve was rotated around the $x$-axis to make a trumpet, but
we will remain in 2D.)  If an under-angel has to paint the perimeter of the curve, does it
need a finite or infinite amount of paint?  How do you know?  What is it had to paint
the area under the curve and above the $x$-axis: does it need an finite or infinite amount
of paint?  How do you know?

\end{onlyproblem}
\begin{onlysolution}
Infinite surface, finite area.  The surface never stops, so the painting would never stop.  But the
area sums to finite number.
\end{onlysolution}
\end{defproblem}





\begin{defproblem}{0802:Primes}
\begin{onlyproblem}
Long ago, in BC times, Euclid proved that there are infinite number of primes.  Today, one of
the hotbeds of mathematical research is the Twin Prime conjecture, that there are infinitely
many primes just two numbers apart, such as 5 and 7, or 11 and 13.  See if you can
duplicate Euclid proof by considering a number that is the product of all primes and adding 1
to it.
\end{onlyproblem}
\begin{onlysolution}

\end{onlysolution}
\end{defproblem}


\begin{defproblem}{0802:Power}
\begin{onlyproblem}
A ``power set'' of a set is all the sets you can make out of it, including the null set and the set
itself, of any size.  e.g. P$(\{1,2,3\}) = \left\{\{\varnothing\}, \{1\}, \{2\}, \{3\}, \{1,2\}, \{2,3\}, \{1,3\},\right.$
and $\left.\{1,2,3\}\right\}$.
Use a diagonal argument like Cantor's to prove the cardinality (size) of P$(\aleph_0)$ is $\aleph_1$.
\end{onlyproblem}
\begin{onlysolution}
Make a table of booleans (T/F) for whether a number is included or not.
\end{onlysolution}
\end{defproblem}




\begin{defproblem}{0802:A2}
\begin{onlyproblem}
It is thought that the set of all functions and relations has cardinality $\aleph_2$.  Propose
an argument for why this might be so.
\end{onlyproblem}
\begin{onlysolution}
Most functions and relations map the real numbers onto the real numbers.  Like a power set,
all possible combinations of the reals should yield a higher cardinality than the reals.
\end{onlysolution}
\end{defproblem}


\endinput
