%!TEX root =  ../main.tex
\ExSection{Exercises}\label{sec:0802x}
\begin{exercises}{sec:0802x}
\prob[0802Hotel] Hilbert's Hotel is a rather strange hotel: it has an infinite number of room.
Unfortunately, there is a guest in every room.
\marginfig[0in]{\chapdir/pics/hotel4}{David Hilbert's Infinite Hotel is a famous problem in mathematics.}
In the following situations, state the one instruction the manager must
give to the existing guests and the new guests to accommodate them all
and not have any room sharing.
\subprob A finite number ($n$) of guests arrive.
\subprob A infinite number of guests arrive.
\subprob An infinite number of busses arrive, each hold an infinite number of guests.
\subprob In infinite number of boats arrive in the harbor, each holding an infinite number of busses, each holding an infinite number of guests.


\prob[0802BFF]  Having studied polynomial long division, consider what 
happens when the answer is infinite.  Divide 1 by $1-x$ out to many places.  
What pattern emerges?  Proceed by adding one term add a time: at what point 
does the graph start to look like the original problem?  What would you estimate 
the effective domain of your infinite approximation is?


\prob[0802Grandi] Luigi Grandi was a monk and mathematician, and he looked 
at a very boring but infinite sum.  It was $1 - 1 + 1 - 1 + 1 - 1 + 1 -1 \dots$.  
What are two possibilities this sum alternates between, if evaluated after an
even or odd number of terms?  According to Morris Kline, Grandi knew these 
possibilities were mutually exclusive, so he look for a third option.  He considered 
that his infinite sum could be approximating by $\frac{1}{x+1}$ as $x$ approaches 1.  
What value do you think he attributed to his infinite sum?  Decide whether you 
think he was right or wrong and argue why.


\prob[0802Gabriel] Gabriel's Horn is an imaginary trumpet, thought up by the 
mathematician Evangelista  Torricelli, and supposedly belonging to the archangel Gabriel.  
We will simplify the math ever so slightly for our purposes here.  Consider the
equation $y=\frac{1}{x^2}$ in the first quadrant only, that is, with a domain of $x\in{[1,\infty)}$.
(In the original problem, a curve was rotated around the $x$-axis to make a trumpet, but
we will remain in 2D.)  If an under-angel has to paint the perimeter of the curve, does it
need a finite or infinite amount of paint?  How do you know?  What is it had to paint
the area under the curve and above the $x$-axis: does it need an finite or infinite amount
of paint?  How do you know?


\prob[0802Primes] Long ago, in B.C. times, Euclid proved that there are an infinite 
number of primes.  Today, one of the hotbeds of mathematical research is the 
\marginfig[0in]{\chapdir/pics/prime_factor_spiral_10000.png}{With the number 1-1000 laid
out in a spiral, the primes appear like semi-random radial arms.}
Twin Prime conjecture, that there are infinitely many primes just two numbers apart, 
such as 5 and 7, or 11 and 13.  Proceed by contradiction (i.e., show that the opposite
is impossible).  Was Euclid wrong?  Assume there are a \emph{finite} number of primes.
What are the possible statuses of ``the product of all primes plus 1''?


\prob[0802Power] A ``power set'' of a set is all the sets you can make out of it, including 
the null set and the set itself, of any size.  For example, P$(\{1,2,3\}) = \left\{\{\varnothing\}, 
\{1\}, \{2\}, \{3\}, \{1,2\}, \{2,3\}, \{1,3\},\right.$ and $\left.\{1,2,3\}\right\}$.
Use a diagonal argument like Cantor's to prove the cardinality (size) of 
P$(\aleph_0)$ is $\aleph_1$.


\prob[0802Aleph2] It is thought that the set of all functions and relations has 
cardinality $\aleph_2$.  Propose an argument for why this might be so.


\end{exercises}