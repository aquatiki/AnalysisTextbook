%!TEX root =  ../main.tex

\subsection{Indeterminate Forms}

\objective{Classify difficult limits problems and solve via certain techniques}


Just as we used limits to define the derive six chapters ago, now they return the favor
and derivatives themselves allow us to find certain limits.  We have seen how removable
discontinuities take the form $\frac{0}{0}$, but what are we to do if a derivative
produce that form?  In fact, there are many forms which mathematicians classify
as \textbf{indeterminate}

$$
\frac{0}{0} , \frac{\infty}{\infty} , 0\times\infty , \infty-\infty , 0^0 , ^\infty \text{, and } \infty^0 
$$

These are all indicative of a variety of answers, some finite, some infinite.  There is
no way to know what answer will predominate in the end without more work.
Consider this limit
\begin{equation}
\lim_{x\rightarrow0}\frac{5x}{\ln(x+1)}
\end{equation}
Obviously, if we attempt a direct evaluation, we get a meaningless answer.  But
consider what we know about this quotient function.  Every ``nudge'' in the $x$
direction (dx) will produce a small change in the top half of the fraction: 5dx, to
be precise.  Similarly, the change in the denominator (it's implicit differential)
is $\frac{1}{x+1}$dx.  As a fraction, this is $\cfrac{5dx}{\frac{1}{x+1}dx}$ or 
$5(x+1)$.  Notice this is not like the maneuvering we did with removable 
discontinuities; we have not constructed an identical but hole-less version
of the equation.  Our new equation is useful because it goes through the hole.


This technique for solving limits at indeterminate value is call \textbf{L'Hopital's}
Rule (LO-pee-talz), even though it was found by L'Hopital's collegue Bernoulli.

\subsection{Infinite Forms}
Not every form with a zero in the denominator or an infinity in it is indeterminate.
The following forms resolve in other ways, and cannot be solved with L'Hopital's
Rule:
$$
\frac{1}{0} , \frac{\infty}{1} , \frac{\infty}{0} , \infty + \infty , \infty \times\infty
$$

Some other fractions you should recognize from algebra and/or experience as zero:
$$
\frac{0}{1} = \frac{1}{\infty} = \frac{0}{\infty}
$$

To return to our earlier equation of $\frac{5x}{\ln(x+1)}$, what is the limit as $x$ 
approaches infinity?  Logarithmic graphs \emph{seem} to taper off, and direct 
evaluation is useless (plugging in infinity for $x$) because we get $\frac{\infty}{\infty}$.
But L'Hopital's Rule works here too, because that is an indeterminate form.
Our limit will be the same as 
\begin{equation}
\lim_{x\rightarrow0}5(x+1)
\end{equation}
which is infinite.