%!TEX root =  ../main.tex
\renewcommand{\columnseprule}{1.5pt}
\begin{multicols*}{2}
\rule[0.5\baselineskip]{0.4\textwidth}{1pt}
\noindent
\LabSection{Cosine, cosine, sine, sine}\label{sec:1003p}
\begin{exercises}{sec:1003p}
\lab{} Take regular size piece of blank, un-ruled paper in portrait orientation and make a 1 ft-long line segment from the bottom-left corner (which is the origin, labelled `0') to the appropriate spot on the top side (and label it `A').  We will call this line's length `one unit.'

\vspace{1cm}
\lab{} Connect point A by a 6-inch line segment to the right-side of the paper.  Label that point `K'.  Connect K back to O.  Neither measure nor label AK nor OK for length!  Proceeding from the top-left clockwise, label the other page corners as B, C, and D.

\vspace{1cm}
\lab{} Label $\angle AOK=\theta$ and $\angle KOD=\phi$.  Now label all appropriate other angles as $\phi$ or $\theta + \phi$.  Leave their complementary angles unmarked.  Is there another $\theta$ on the page?

\vspace{2cm}
\lab{}  Why can we rightly label OK as $\cos{\theta}$?  How do you know?

\vspace{2cm}
\lab{} By the same token, what can we label AK?

\vspace{3cm}
\lab{} Is it true, the way we are labeling these sides?  Measure angle $\theta$ and confirm AK is cosine-of-that-many-feet long.

\vspace{3cm}
\lab{}  Label OD as a product of cosines, using the definition of cosine.

\vspace{2cm}
\lab{} Continue labeling all the segments.

\vspace{1cm}
\lab{}  Write an equation relating the left and right edges of the paper.

\vspace{3cm}
\lab{}  Write an equation relating the top and bottom edges of the paper.  Solve for $\cos{(\theta+ \phi)}$.

\vspace{4cm}
\lab{} Describe what you think the point of this problem set is, in technical vocabulary, using whole sentences.
\end{exercises}
\end{multicols*}