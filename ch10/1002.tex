%!TEX root =  ../main.tex

\subsection{Reference Triangle Within}
Consider any reference triangle with legs $x$ and $y$ and $r$, and angles $\frac{\tau}{4}$
and $\theta$.  The other angle must be $\frac{\tau}{4} - \theta$.  What are $\sin\theta$ and
$\cos\theta$?  What are $\sin(\frac{\tau}{4}-\theta)$ and $\cos(\frac{\tau}{4}-\theta)$?

insert diagram

Now, at last, we are prepared to answer why half of the six trigonometric functions have the
prefix ``co-''.  It stands for complement!

six co-definitions


We have repeatedly seen that sine is the signed height of the reference triangle within the unit
circle, while cosine is the signed width.  Because these lengths are within the unit circle (a
circle of radius 1), they must be in a Pythagorean relationship.  This reference triangle within 
the unit circle provides an excellent visual to accompany the identity, $\sin^2\theta+\cos^2\theta
= 1$.  But through simple algebra manipulation -- dividing by $\sin^2\theta$ or $\cos^2\theta$ ---
it is easy to turn this identity into $1 + \tan^2\theta = \sec^2\theta$, or the more obscure
$\csc^2\theta + 1 = \cot^2\theta$.  What do they represent?

\subsection{Reference Triangles Without}
We have previously defined tangent as sine over cosine, which means we can set up the proportion:

\begin{equation}
\frac{\sin\theta}{\cos\theta} = \frac{\tan\theta}{1}
\end{equation}

This is not only algebra, but can be represented with similar triangles.  ``Sine is to cosine as tangent is to
1'', is the height of one triangle compared to its width, keeping a common angle and still being right in
another triangle whose height is tangent and width is 1.  This is a right triangle extending outside
the unit circle, but whole bottom edge is a radius thereof.

insert diagram

Similarly, there are two more proportions we must examine:
\begin{equation}
\frac{\cos\theta}{\sin\theta} = \frac{\cot\theta}{1}
\end{equation}
This is a way of saying the width vs. height of one triangle is cosine vs. sine, and that a similar right
triangle (with angle in common) has width cotangent and height 1.  

insert diagram

Exercise: show that this get the same length as 1 over tangent equals cotangent over 1.

There is also
\begin{equation}
\frac{1}{\cos\theta} = \frac{\sec\theta}{1}
\end{equation}

Geometrically this means, if we draw a triangle with hypotenuse 1 and adjacent of cosine, then there must exist a unique, similar triangle with hypotenuse secant and adjacent 1.


\subsection{Relative}
It should be apparent now that trigonometry is very different from algebra.  Every expression can be 
written in an infinite number of ways.  Every term is equal to 1 times itself (the Identity of Multiplication),
but 1 equals $\sin^2\theta + \cos^2\theta$ or $\sec^2\theta-\tan^2\theta$.  We practice turning 
trigonometry expressions into specific forms so that we will be fluent in simplify complicated 
expressions in the real world.

The co-function identities (above) show us what to do with $90^\circ/\frac{\tau}{4}$s inside of
trig functions, but what about other quadrants and negatives?  There are always lots of options
for how you might think about such things, so we will consider two ways to think about
simplifying $\sin(-x)$ and $\cos(-x)$.

Graphically, you can see that $y=\sin(x)$ is an odd function.  On the unit circle, if we begin at
standard position, we see that turning a positive angle yields a certain height above the $x$-axis,
while turning that same angle \emph{down} yields the opposite height.  Hence, $\sin(-x)=-\sin(x)$.
$y=\cos(-x)$ is an even function, begin symmetric about the $y$-axis.  On the unit circle, the 
horizontal displacement caused by any angle is the same as its negative angle.  Hence,
$\cos(-x)=\cos(x)$.

