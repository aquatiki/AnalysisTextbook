%!TEX root =  ../main.tex
\renewcommand{\columnseprule}{1.5pt}
\begin{multicols*}{2}
\rule[0.5\baselineskip]{0.4\textwidth}{1pt}
\noindent
\LabSection{Twists and Turns}\label{sec:0603p}
\begin{exercises}{sec:0603p}
\lab{} What if $x$-intercepts are not the only thing important about a graph?  Consider the function $\frac{1}{6}x^6-2x^4+8x^2-20$.  Knowing that the only Real zeros are $\pm2.8\dots$, sketch a graph from -3.5 to 3.5, with a $y$-scale of 2:1.



\vspace{4cm}
\lab{} Find the first derivative.  Factor it completely.




\vspace{3cm}
\lab{} Sketch a graph of $f^\prime(x)$.


\vspace{4cm}
\lab{} When is $f^\prime(x)$ positive?  Negative?  0?
What is $f(x)$ at those moments?


\vspace{3cm}
\lab{} Find $f^{\prime\prime}(x)$.  Factor it completely


\vspace{3cm}
\lab{} Sketch a graph of $f^{\prime\prime}(x)$.


\vspace{4cm}
\lab{} When is $f^{\prime\prime}(x)$ negative?  Positive?  0?
What is $f^\prime(x)$ at those moments?  What is $f(x)$ at the same times?



\vspace{3cm}
\lab{} Make a number-line/table of all the instances where the derivative or the second derivative is zero and record the sign of the other versions of the function.


\vspace{3cm}
\lab{} Make a rule what happens when the first derivative is 0, with the cases being the sign of the second derivative.



\vspace{3cm}
\lab{}  Describe in your own words what you think the point of this problem set is.
\end{exercises}
\end{multicols*}