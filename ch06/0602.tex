%!TEX root =  ../main.tex

\subsection{Polynomial Long Division}

\objective{Use and known when to use polynomial and synthetic division}


First, we recall long division.  Surprisingly, it can actually be more difficult than polynomial long division at times, so we should pick an easy example:

\vspace{5mm}

\longdiv{11}{3678}

We restrict our vision to the left two digits of the dividend, as we know that 11 will go 3 times into
36.  We write a 3 above the 6, write 33 below the 36 and subtract it.  Next, we bring down the 7
and do the same manner of work on 37.  The process repeats, but moves progressively right
and we get an answer of 334, but with a remainder of 4.  Since we most want a number that
when multiplied by 11 obtains 3678, we should write the answer as $334+\frac{4}{11}$ because
$\left(334+\frac{4}{11}\right)(11) = 3678$.


Now, polynomial long division is no different.  We start at the left edge of the dividend, and use 
only the leading term of the divisor.

\polylongdiv[stage=3]{x^3-x^2+x-1}{x+1}

We only regarded the leading (left) edge of the larger polynomial.  In this case, that is $x^3$.  
We might ask ourselves, ``What do I need to multiply $x$ by to get to $x^3$?''.  The answer is $x^2$.  
That term is the first in our quotient, and we multiply it by the entire divisor.  We subtract, 
and bring down the next term.

\polylongdiv[stage=4]{x^3-x^2+x-1}{x+1}

This process continues, only working on the left-edge as we go down.  

\polylongdiv[stage=13]{x^3-x^2+x-1}{x+1}

The final answer is $x^3-x^2+x-1 = (x+1)\left(x^2-2x+3-\cfrac{4}{x+1}\right)$.

\subsection{Synthetic Division}
One might ask, ``Isn't there a faster way?'' and indeed there is.  Let us first consider
how we might compactify the polynomial.  Consider the first two terms, and look for
a common factor:
$$
(x^3-x^2)+x-1
$$
Obviously, we can take out an $x^2$ from the first two terms.  Having condensed the
first two terms into one (because multiplication does not increase the term count),
now consider the greatest common factor of the new first two term:
$$
\left[x^2(x-1)+x\right]-1
$$
Every term has at least one $x$, so that can be safely factored out.  Look at what we have made,
called \textbf{nested form}:
$$
x\left[(x-1)x+1\right]-1
$$
Substituting any value into this function now only involves multiplication and addition/subtraction,
ever exponents.  This process can be represented in half-box, called \textbf{synthetic division}

\polyhornerscheme[showvar=true,resultbottomrule=true,resultleftrule=true,resultrightrule=true,x=-1]{x^3-x^2+x-1}

Study this carefully, and notice how it is the same answer as the coefficients of the polynomial
long division problem.  There are lots of things to notice.  The technique with synthetic division is
to remember that the first number falls straight down, but from then on we  ``multiply up, add down''.  
It is helpful to draw a box around the final number, to remember that it is special: it is the remainder.  
Reading right-to-left: 3 is the $x^0$ coefficient, $-2$ is on $x^1$, 1 is on $x^2$.  
In the long division, we divided by $x + 1$.  Here, we are plugging in a value, so it's $x + 1 = 0$, 
which is $x=-1$. 

\begin{example}[Synthetic Division]
Given $g(x) = x^3-7x+6$, find $g(2)$.  What does this tell you about $\frac{g(x)}{x-2}$?

\paragraph{Solution}
First, we write our box and the $x=$ we want to test:\\


\polyhornerscheme[showvar=true,stage=1,x=2]{x^3-7x+6}
(Note the zero.  Failure to notice this is crucial mistake!)


The first number passes down freely:
\polyhornerscheme[showvar=true,tutor=true,stage=2,x=2]{x^3-7x+6}


Then we ``multiply up...
\polyhornerscheme[showvar=true,tutor=true,stage=3,x=2]{x^3-7x+6}


..., add down''
\polyhornerscheme[showvar=true,tutor=true,stage=4,x=2]{x^3-7x+6}


and so on and so forth:
\polyhornerscheme[showvar=true,resultbottomrule=true,resultleftrule=true,resultrightrule=true,x=2]{x^3-7x+6}

The fact that the remainder is 0 is significant.  $(x-2)$ is a factor in $x^3-7x+6$.  We can write
this as $(x-2)(x^2+2x-3)=x^3-7x+6$.  The quadratic factors easily into (x+3)(x-1), leaving the
entire cubic equation solved.
\end{example}

\subsection{Complex Zeros}
What if nothing so need exists as a solution?  We might slightly alter our $g(x)$ to be
$x^3-2x^2-7x+2$.  There is no easy factorization (grouping won't work).  We are forced
to use more complicated algorithms.  In the past, mathematicians invented very complicated
methods for find, eliminating, and testing solutions to polynomial equations.  You will research
one of these in problem number one of the exercises.  But in our age, why not use a computer?
If done intelligently, we can get the same level of precision as in the past.

For example, if you take our new $g(x)$ and put it in your TI-8*, you can see --- either in the
TABLE or on the graph --- that it passes through (-3,0), which should lead you to want to
factor out (x+3).  You can use synthetic division as so:


\polyhornerscheme[showvar=true,resultbottomrule=true,resultleftrule=true,resultrightrule=true,x=-3]{x^3-2x^2-7x+24}

This proves that $(x+3)(x^2-5x+8)=x^3-2x^2-7x+24$.  However, now we have a new problem: how
can we solve $x^2-5x+8=0$?  It is not factorable.  Well, any quadratic can be solved with
the quadratic formula, so that will have to do.

$$
x=\frac{5\pm\sqrt{(-5)^2-4(1)(8)}}{2\cdot{}1}
=\frac{5\pm i\sqrt{7}}{2}
$$

As you can see, unfactorable quadratics will always have some quantity inside the square root,
once plugged into the quadratic formula.  This means they will produce either a pair of
irrational solutions, or else a pair of complex (imaginary) ones.  This has other to prove
(elsewhere) that such solutions \emph{always} occur in pairs.

For example, if we are building a polynomial, and we know solutions include 2 and $3+2i$,
then we can be sure that $3-2i$ is also a solution.  What factor would produce such zeros?
Well, x-2 is an obvious one, but what about the more difficult one?  If $x=3 \pm 2i$, what
would have lead to two possibilities like that?

\begin{align*}
  x & = 3 \pm 2i\\
  x - 3 & =  2i \\
  (x-3)^2 & = -5\\
  x^2-6x+9 &= -5\\
  x^2-6x+14 &= 0
\end{align*}

Since plus-or-minus solution come from square rooting, the inverse operation must be to
square both sides.  $(+2i)(+2i) = (-2i)(-2i) = -5$, so writing plus or minus does not matter.
