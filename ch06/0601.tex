%!TEX root =  ../main.tex

\subsection{Definition}

\objective{Describe and predict the general shape of a polynomial graph}


A polynomials is sum of terms, all of which are of the form $a\cdot{}x^n$, where $a$
is any rational number and $n$ is a whole number.

\subsubsection{Degree}
``In the long run, the biggest exponent always wins.''  While it may sound like a odd
cliche from a movie, it is certainly true.  What is more, there are only two possibilities
for polynomials: either larger and larger numbers are being raised to an even degree,
or they are being raised to an odd one.  Positive numbers grow more positive in either case.
Negative numbers become positive, when raised to an even degree.  They grow more
negative when raised to an odd degree.

Of course, in the short, all manner of other behaviors may be manifested, but it is 
important to consider limits at the infinities, so that we may be sure of the ultimate
leftward or rightward behavior of a function.

\subsubsection{Leading Coefficient}
While it is true that there are only two possibilities for the integer $n$ in the 
expression $x^n$ (either it is even, or it is odd), there are two more things that could 
happen, when we consider $a\cdot{}x^n$: $a$ could be positive or negative.
In summary, even polynomials go ``up'' at both ``ends'', unless the leading coefficient
is negative, in which case both ends go ``down''.  Odd functions go up to the right and
down to the left, unless the leading coefficient is negative, in which case the opposite
happens.

If we are designing a function ourselves, it can still be useful to consider
end-run behavior.  For example, we might construct a polynomial to have
$x$-intercepts 3, 5, and $-\frac{1}{2}$.  We would write $f(x)=(x-3)(x-5)(2x+1)$,
seemingly obfuscating the leading term.  However, if we multiply only the $x$ terms,
we can still find it quickly.  $x \cdot x \cdot 2x$ is $2x^3$, so we can see this is
a positive, odd polynomial, going up to the right and down to the left.

\subsubsection{$y$-intercept}
Another facet of polynomials that is still quickly discernible even in factored form
is the constant term.  This term (even if it is 0, i.e. absent) provides useful information.
When $x=0$ (which typically corresponds to something like an initial condition), the
only term not to cancel will be the constant one.  In the previous example, we can 
easily multiply the plain numbers: $-3 \cdot -5 \cdot 1 = 15$, so we know the
$y$-intercept of the function will be 15.  If we need a different number, the function
can always be scaled by multiplying on the outside by a constant, which could even
be negative.

\subsection{Factored Forms}
When we examined quadratics, there were some cases in which the vertex of the
parabola was found on the $x$-axis.  A similar case can be found in polynomials of
higher degree, and can be seen in their factored forms.  For example,
$f(x)= (x-2)^2(x+5)$ is a cubic equation  (with leading term $x^3$ and constant term
20) which seems to have only two zero's: 2 and -5.  However, the number 2 works as
a solution \emph{twice}, and so we say it has a \textbf{multiplicity} of 2.

Graphically, multiplicity means the graph resembles the exponent of the factored term.
This means $f(x)$ will behave like $x^2$ in the vicinity of 2.  Note that this behavior
may be upside-down, depending on the equation.  Generalizing even more, we can say
that even powers on a factor will result in the graph \emph{not} crossing the $x$-axis
at that zero, but only ``bouncing'' off.  On the other hand, odd powers on a factor
(even including 1!) will appear as the graph proceeding \emph{through} the $x$-axis.

