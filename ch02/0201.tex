%!TEX root =  ../main.tex

\subsection{First Discontinuities}

\objective{Find informal limits at holes and non-holes.}

In this section, we will explore the informal definition of a \textbf{limit}.  Have you ever had anyone
sneer at you over a lost opportunity and say, ``Should of, would of, could of''?  (It probably
sounded like ``shoulda, woulda, coulda''.)  The idea is that everyone can see a hypothetical 
in hindsight, though that does not avail you anything now.  Hypotheticals are situations describing
what was likely, or intended, or desired.  In mathematics, we often encounter functions
which appear \textit{as though} there is an expected value, only to have that spot not
even be part of the domain.

Limits are like the hypothetical situation of math.  Suppose your friend bought a ticket for a vacation,
and said they wanted to get away.  Then you didn't see this friend for a long time, and then you asked
them, ``Did you ever go on vacation?  Where were you going to go?'  You know they 
were in some sense headed somewhere, but from your perspective, it has yet to be determined if
he or she actual went and where your friend was even headed.  Notice how your second question
doesn't even depend on whether they went anywhere or not: it is about a hypothetical.  

Many functions give an indeterminate answer at certain gaps in their domain.  It is obvious
where they were ``headed'', but direct evaluation at that value in the independent variable is
not helpful.



\begin{derivation}{Indeterminate Form}
\index{indeterminate form}
For a given function $f(x)$, if $f(c)=\frac{0}{0}$, then the point $c$ represent a gap in the domain.
\end{derivation}


\subsection{Algebra Techniques}
From the perspective of algebra, there are four techniques for constructing 
precise replacements for many functions, replacements which are everywhere else the same,
but lack the particular ``hole''.

\subsubsection{Cancelling}
Consider the function $f(x) = \frac{x^2-4}{x-2}$.    Enter it in your calculator and try \Touche[style=function,principal={ZOOM},]
\Touche[style=number, principal=6].  How can it be just a line?  Why isn't it more complicated than that?  
Well, it is.  Try plugging in 2.  That is, what is $f(2)$?  $\frac{2^2-4}{2-2} = \frac{0}{0}$.


\begin{derivation}{Factor Removal}
When $f(c)=\frac{0}{0}$ and the removal of a common factor in the numerator and denominator yields a real number $d$,
then $(c,d)$ represents the location of a hole in $f(x)$.
\end{derivation}


Returning to our equation, $x^2-4$ in the numerator factors by the 
Difference of Squares to $(x-2)(x+2)$.  This is means
we can write $f(x)=\frac{(x-2)(x+2)}{x-2}$.  It is \emph{not} true that 
$\frac{(x-2)(x+2)} = x+2$: they differ in their domains.
However, $x+2$ is an \textbf{analytic continuation} of $\frac{x^2-4}{x-2}$, 
meaning is is everywhere the same as the
original but has an \textit{even larger} domain.  We may use $x+2$ to answer the q
uestion what $f(x)$ would output
at $x=2$, were it to exist there.

\subsubsection{Expanding}
Some functions obscure the factor that could be cancelled with further arithmetic.  For example, it is not obvious
what $g(x)=\frac{(x+3)^3-27}{x}$ will be at $x=0$\footnote{If you are especially keen, you might notice that this is
factorable as the difference of cubes in the numerator, but we will pretend no one saw that!}.  Sometimes a small
piece of arithmetic allows us to proceed as in the previous section.  in this case $(x+3)^3=x^3+9x^2+27x+27$.


\begin{align*}
\frac{(x^3+9x^2+27x+27)-27}{x} &=\\
\frac{x(x^2+9x+27)}{x} & \approx x^2+9x+27\\
& \rightarrow (0)^2+9(0)+27 \\
&\rightarrow 27\\
\end{align*}


\subsubsection{Complex Fractions}
\emph{Not to be confused with fractions involving complex numbers!}

Besides simple arithmetic, complex fractions can obfuscate the cancelling 
needed to simplify the presence
of a hole.  $\cfrac{2-x}{\frac{1}{x}-\frac{1}{2}}$ is indeterminate at $x=2$.  
However, if we simplify this fraction until it has a simple (non-fraction) numerator and 
denominator, the indeterminate form will evaporate.


\begin{align*}
\cfrac{2-x}{\frac{2}{2}\cdot\frac{1}{x}-\frac{1}{2}\cdot\frac{x}{x}} &= \cfrac{2-x}{\frac{2-x}{2x}} \\
&\approx \cfrac{2x(2-x)}{2-x} \\
&\approx 2x\\
&\rightarrow 2(2) = 4
\end{align*}



\begin{derivation}{Additional Factor}
When $f(c)=\frac{0}{0}$ and the multiplication by a common factor in the numerator and denominator yields a real number $d$,
then $(c,d)$ represents the location of a hole in $f(x)$.
\end{derivation}



\subsubsection{Conjugates}
What to multiply by can be difficult to decipher.  It often appears as though we might wish to square individual terms 
in the numerator or denominator.  For instance, the function $h(x) = \cfrac{\sqrt{x+1}-1}{x}$ is $\frac{0}{0}$ at $x=0$.
Clearly, we might wish to square only the upper-left corner of the fraction, in order to remove the square root.  The
secret is to recognize the numerator is of the form $a-b$, where $a=\sqrt{x+1}$ and $b=1$.  Any binomial multiplied
by it's conjugate yields the difference of squares (i.e. $(a+b)(a-b)=a^2-b^2$).  In this case, we must multiply top and
bottom by $\sqrt{x+1}+1$.


\begin{align*}
\cfrac{\sqrt{x+1}-1}{x} \cdot \cfrac{\sqrt{x+1}+1}{\sqrt{x+1}+1} &\approx \cfrac{(x+1)-1}{x(\sqrt{x+1}+1)}\\
&\approx \cfrac{1}{\sqrt{x+1}+1}\\
&\rightarrow \cfrac{1}{\sqrt{0+1}+1} = \frac{1}{2}
\end{align*}

\subsubsection{Direct Substitution}
Some time, there might seem to be a hole, but none exists.  In that case, we can simply plug
the input into the equation and get a result.

\inlinefig{\chapdir/pics/Continuidad_de_funciones_02.png}{Removable discontinuities are typically hole in the graph, places where there is no value to the function, but the expected value is straightforward to calculate.}

