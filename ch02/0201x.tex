%!TEX root =  ../main.tex
\noindent%
\ExSection\label{sec:0201x}
\marginfig[0in]{\chapdir/pics/Black_hole_Cygnus_X-1.jpg}{Scientists know many things about the holes in our universe by following the paths of expectation from observable phenomena.}
The following problems are available to examine yourself
and see if you are able to discern the correct technique and successfully apply it.
\begin{exercises}{sec:0201x}

\begin{multicols}{2}
\prob[0201Remove1] Find the limit of the given function at the given value.
This may require you to create a version of the given function that will 
allow you to plug in the given value.  The result should typically be a fraction.
\subprob $\dfrac{x^2+x-6}{x-2}$ at 2
\subprob $\dfrac{\sqrt{x+2}-3}{x-7}$ at 7
\subprob $\dfrac{(5+x)^2-25}{x}$ at -5
\subprob $\dfrac{x^3-6x+2}{x^2+2x-3}$ at 3
\subprob $\cfrac{\frac{1}{3+x} - \frac{1}{3}}{x}$ at 0
\subprob $\dfrac{1}{x\sqrt{1+x}}-\dfrac{1}{x}$ at 0


\prob[0201Remove2] 
\subprob $\dfrac{(x-1)^3+1}{x}$ at 0
\subprob $-\dfrac{2x^2+3x}{2x+3}$ at $-\frac{3}{2}$
\subprob $\cfrac{x}{\frac{1}{x-2}+\frac{1}{2}}$ at 2
\subprob $\dfrac{x^3-1}{x^3+5x^2-6x}$ at 1
\subprob $\dfrac{\sqrt{x^2+9}-5}{x+4}$ at -4
\subprob $\dfrac{\sqrt{x+\frac{29}{2}}-4}{x-\frac{3}{2}}$ at $\frac{3}{2}$

\end{multicols}
\end{exercises}
