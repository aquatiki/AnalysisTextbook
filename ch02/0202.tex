%!TEX root =  ../main.tex

\subsection{Epsilon-Delta}

\objective{Use the definition to evaluate limits.}


In the last section, you experiment with finding the values that \emph{would have}
occurred in a series of functions, if there had not been a \textbf{discontinuity} at a given
point.  The value that the function approaches is called a \textbf{limit}.  Because all the
examples to date have been \textbf{removable discontinuities}, it has simply been a matter
of algebra manipulation to produce a \emph{nearly} identical version of the function, but
simply one without the hole.

In the beginning, it was sufficient to define limits as casually as we have done, which is why
2.1 is laid out the way it is.  Later, mathematicians named Cauchy and Weierstrauss created
much more rigorous definitions to meet demand.  Let us built up such rigor.
\marginfig[-1in]{\chapdir/pics/tricky}{The function appears to be a decreasing exponential ... or does it?}

\index{TI-8*!zoom box}
Here is a negative example: $f(x)=\dfrac{|x-3|\cdot1.2^x}{(x-3)\cdot12^x}+2$.  Enter the 
function as $Y_1$ and begin with ZOOM-STANDARD.  You should have a decreasing
exponential function, with a horizontal asymptote at $y=2$.  Now you should become
suspicious.  What did the $x-3$ terms produce?  Is there something going on at $x=3$,
when both terms are 0 and the entire function is $\frac{0}{0}$?  If you consult the
TABLE, you will see there is an ERROR at 3.  Let us investigate.

Practice using the ZOOM-BOX feature, drawing new windows around (3,2) until the 
discontinuity becomes clear.  Here is our window and graph:

insert figure Xmin=2.999 Xmax=3.001 Xscl=5E-4 Ymin=1.9985 Ymax=2.0015 


How can we avoid being fooled like this?  What does it mean to say ``the limit as
$f(x)$ approaches $c$ is $L$''?

\marginfig[-1in]{\chapdir/pics/epsilondelta.png}{Whenever a point $x$ is within $\delta$ units of $c$, $f(x)$ is within $\epsilon$ units of $L$ \cite{epsilondelta}.}



\begin{derivation}{Definition of a Limit}\index{Limit!definition}
$$\lim_{x\rightarrow c} f(x)=L$$


The limit as $x$ approaches $c$ of $f(x)$ is $L$, if and only if
for any $\epsilon>0$ no matter how small, there is a $\delta>0$ such that
$x$ is with $\delta$ units of $c$ but $x\ne c$, that $f(x)$ is with $\epsilon$ units of $L$.
\end{derivation}



In less compact terms, this means for \emph{any} radius $\delta$ around $c$, there is exists
a radius $\epsilon$ around $L$ that $f(x)$ stays with in.  This definition, developed by
Cauchy, is the successor to Leibniz's idea of ``infinitesimals''\footnote{
See \cite{Alexander12}.}.


\subsubsection{Continuity}
It would certainly save a lot of time if we could classify functions by whether they will
ever fail to have a limit or not.  Smooth functions, where every minuscule change in the
input results in a minuscule change in the output are called \textbf{continuous}.  A function
$f(x)$ is continuous at point $a$, if and only if:\index{continuity}
\begin{enumerate}
\item $\displaystyle \lim_{x\rightarrow a} f(x)$ exists,
\item $f(a)$ exist, and
\item $\displaystyle \lim_{x\rightarrow a} f(x) = f(a)$.
\end{enumerate}
A function that is continuous at every point in its domain is said to be a \textbf{continuous
function}.  Continuous functions whose domain is all Real numbers are said to be ``continuous
over the Reals'' or ``continuous for Reals''.


\begin{example}{Discontinuous}
	\exProblem
Is the function $f(x)=\frac{1}{x}$ continuous for Reals?

	\exSolution
No.  In fact, there is \emph{no} $\epsilon$ we could pick that the
output would stay within for any $\delta$ around 0.  The asymptote at $x=0$ shows that the
graph increases \emph{without limit}.


\marginfig[-1.5in]{\chapdir/pics/Function-1_x.png}{Function $y=\frac{1}{x}$\cite{function1x}.}
\end{example}


\begin{example}{Testing the Definition}
	\exProblem
Show that
$$
\lim_{x\rightarrow 1} (5x-3)=2
$$
	\exSolution
According the definition of a limit, we set $c=1$, $f(x)=5x-3$ and let
$L=2$.  To show that $\displaystyle \lim_{x\rightarrow 1}(5x-3)=2$, we need to show that for
any number $\epsilon>0$, there exists a number $\delta>0$ such that for all $x$,

$$
0<|x-1|<\delta \quad \Rightarrow \quad |(5x-3)-2|<\epsilon 
$$

(The symbol $\Rightarrow$ is read ``implies''.)  If we can transform our second equation to 
contain the middle term of the first, we have succeeded:

\marginfig[-1in]{\chapdir/pics/epsilonover5}{$\epsilon-\delta$ around (1,2)}
\begin{align*}
|(5x-3)-2| & < \epsilon \\
|5x-5| & <\epsilon \\
5|x-1| & <\epsilon \\
|x-1| & < \epsilon \div 5\\
\end{align*}
The last line tells us that the original $\epsilon$-inequality will hold is we choose
$\delta < \epsilon \div 5$.


\end{example}