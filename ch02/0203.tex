%!TEX root =  ../main.tex

\subsection{Definition}

\objective{Interpret piece-wise functions and their limits}


In human observations, there is almost nothing which follows one equation for the 
entirety of its domain.  The only constant in the universe is change!  For example, 
population of the world, the stock market, or even one stock might generally follow one
equation for a significant stretch of time, but not forever.  So it is that most functions are 
defined in pieces, and are therefore called piece-wise functions.

For example, the absolute value function which you have been using for some time now is
actually a piecewise function.\index{Absolute Value!piece-wise}

We should formally define the nomenclature of a ``sided'' limit:

\begin{derivation}{Right-sided limit}\index{limit!right-sided}
``The limit as $x$ approaches $c$ from the right is $L$'' is true if and only if for every $\epsilon > 0$, there exists a $\delta > 0$ such that
$|f(x)-L|<\epsilon$ whenever $0<x-c<\delta$.

$$\lim_{x \to c^+}f(x) = L$$
\end{derivation}


\begin{derivation}{Left-sided limit}\index{limit!left-sided}
Similarly, ``The limit as $x$ approaches $c$ from the left is $L$'' is true if and only if for every $\epsilon > 0$, there exists a $\delta > 0$ such that
$|f(x)-L|<\epsilon$ whenever $0<c-x<\delta$.

$$\lim_{x \to c^-}f(x) = L$$
\end{derivation}


\personfeature[-3in]{George_Boole_color}{George
    Boole}{1815-1864, French}{was an English mathematician
    who said, ``No general method for the solution of questions 
    in the theory of probabilities can be established which does 
    not explicitly recognise, not only the special numerical bases 
    of the science, but also those universal laws of thought which 
    are the basis of all reasoning, and which, whatever they 
    may be as to their essence, are at least mathematical as to their form.''
    \href{https://en.wikipedia.org/wiki/George_Boole}{(Wikipedia)}}

\subsubsection{Boolean Variables}\index{booleans}
Perhaps surprisingly, your TI-8* can graph piece-wise functions.  We will start with a simple
piecewise-function :

$$
f(x)=
\begin{cases}
x, x<1\\
x^2,x\ge 1
\end{cases}
$$

\index{TI-8*!piece-wise}


If we wanted to graph the sections separately, we could make $Y_1=(X^2)/(X\ge{}1)$ and
$Y_2=(X)/(X<1)$.  (The equality and inequality signs are under 2ND-MATH --- TEST.)  This will 
allow you to make the different sections different colors or different shading, but that might be
what you want.  To graph everything in $Y_1$, use $(X^2)*(X\ge{}1)+(X)*(X<1)$.



\subsection{Other Discontinuities}
\reminder{\lefthand}{The different TI-8* behave differently around holes.  Newer calculators will attempt to make a hole apparent, while older models do not show it.

Only the new models draw points visibly, but even then they are very small.  We recommend
against even trying to represent them in the TI-8*.}


All together, there are five kinds of discontinuities.  We are only responsible to rigorously prove
instanes of the first two and the last:

\subsubsection{Removable}
Here the limit exist.  The graph has a hole in it, which may or may not be defined as a point somewhere unexpected.
\subsubsection{Jump}
The limits does not exist.  The graph is not continuous, but ``leaps'' from one output to another without passing in between, at one or more points.
\subsubsection{Infinite} 
The limit may or may not exist.  The function itself goes up and/or down without limit.  The most common example is a vertical asymptote.
\subsubsection{Oscillating} 
The limit does not exist.  The graph varies between outputs in way that never resolves.
\subsubsection{Domain} 
The limits does not exist because on one side, the function ceases to exist.


