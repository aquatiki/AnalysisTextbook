%!TEX root =  ../main.tex
\renewcommand{\columnseprule}{1.5pt}
\begin{multicols*}{2}
\rule[0.5\baselineskip]{0.4\textwidth}{1pt}
\noindent
\LabSection{Removing the Hole}\label{sec:0201p}
\begin{exercises}{sec:0201p}

\lab{} For most of the functions we will encounter in this class, there will be some 
``problem spot'' $c$ in their domain, where the function is undefined.  Typically, if 
we try to evaluate $f(c)$, we will get $\frac{0}{0}$, what is called an 
\gls{indeterminate form}.  For example, try plugging in 0 for $x$:
$$
f(x)=\frac{x}{(x+3)^3-27}
$$

\vspace{1cm}
\lab{} The solution (in this case) is to \emph{expand} the cubing term and let the 27s cancel.  
Work out the algebra, and show that the result
is $\frac{1}{27}$.

\vspace{2cm}
\lab{} A second technique is our old stand-by, \emph{factoring}.  Show that direct 
evaluation of this function at 5 yields an indeterminate form:
$$
m(c) = \frac{c^2-c-20}{c-5}
$$

\vspace{1cm}
\lab{}  Now, factor the numerator, cancel the common term, and show that the result equals 9.

\vspace{2cm}
\lab{} Thirdly, there is the more laborious problem of \emph{removing compound fractions}.  
Again, first show that $\frac{0}{0}$ results, in this case at 2:
$$
s(t)=\cfrac{t-2}{\frac{1}{t}-\frac{1}{2}}
$$

\vspace{1cm}
\lab{}  Next, get a common denominator in the denominator and then recall that division is the 
same as multiplying by the reciprocal.  Show that the result is -4.

\vspace{2cm}
\lab{} Finally, there is a trickiest trick of all: the conjugate.  A very common technique in 
higher maths is to recast a binomial by somehow multiplying by its \textbf{conjugate}.  
\index{conjugate} The conjugate of any binomial $a+b$ is $a-b$, and the same is obviously
true in reverse.  What is the conjugate of the following function's numerator?
$$
g(y) = \frac{\sqrt{y}-3}{y-9}
$$

\vspace{1cm}
\lab{}  Do we need this trick?  What happens to the function at 9?

\vspace{1cm}
\lab{} Since we are not resolving an equation, but an expression, the only why to multiply by the 
conjugate of the numerator of $g(y)$ it to do it to the numerator and the denominator.  However,
do not distribute $\sqrt{y}+3$ in the denominator, but leave it separate.  Show that result is
$\frac{1}{6}$.

\vspace{2cm}
\lab{}  In every case, we made a modified version of the original function through 
algebraic manipulation.  This new version is nearly identical but lacks the 
\textbf{removable discontinuity} of the original. Describe in your own words how our 
changes affect the domain.

\vspace{3cm}
\lab{} What did you learn as a result of doing this exploration that you did not know before?

\end{exercises}
\end{multicols*}