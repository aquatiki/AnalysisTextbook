%!TEX root =  ../main.tex

\subsection{At Infinities}

\objective{Determine when a limit does and does not exist, or is infinite.}

\personfeature[0in]{\chapdir/pics/Augustin_Cauchy.jpg}{Augustin-Louis Cauchy}{1789-1857}{was a French mathematician and physicist who made pioneering contributions to analysis. His book \textit{Cours d'Analyse} is frequently noted as being the first place that inequalities, and $ \delta -\epsilon$ arguments were introduced into Calculus. \href{https://en.wikipedia.org/wiki/Augustin-Louis_Cauchy}{Wikipedia}}



Because limits are asking questions that need not have simple number inputs and
need not have simple number outputs, we can evaluate limits involving infinities.
``Infinity'' simply means ``without end''.  Asking what a function approaches as
$x$ approaches infinity, graphically means ``what value is the output tending
towards as the input grows without bound?''.  Negative infinity is a term describing
the leftward trend of function.  


\begin{derivation}{Limit at Infinity}
For $f(x)$ a real function, the limit of $f$ as $x$ approaches infinity is L, 
means that for all  $\varepsilon >0$, there exists $c$ such 
that $|f(x) - L| < \varepsilon \text{whenever} x > c$. \footnote{$\forall \varepsilon > 
0 \; \exists c \; \forall x > c :\; |f(x) - L| < \varepsilon$}

$$\lim _{x\to \infty }f(x)=L$$

\end{derivation}


The same applies at negative infinity:


\begin{derivation}{Limit at Negative Infinity}\index{limit!at infinities}
For $f(x)$ a real function, the limit of $f$ as $x$ approaches negative infinity is L, 
means that for all $\varepsilon >0$ there exists $c$ such that 
$ |f(x) - L| < \varepsilon \text{whenever} x < c$.  \footnote{$\forall \varepsilon > 
0 \; \exists c \; \forall x < c :\; |f(x) - L| < \varepsilon$}


$$ \lim_{x \to -\infty}f(x) = L$$

\end{derivation}


\begin{example}
	\exProblem
Evaluate $\displaystyle \lim_{x \to -\infty}2^x$
	\exSolution
We can observe the graph or compute numerically that $2^x$ is getting closer
and closer to 0 as we move leftward.  We can get arbitrarily close to 0 by picking
whatever large, negative exponent we wish.  Hence, the answer is 0
\end{example}




\subsection{Infinite Forms}
Not only can we ``plug in'' infinities in limits problems, but we can also get
$\pm\infty$ as an answer.  Recall, however, that the left and right sided limits
must agree for the limit to exist.  Hence, we can say that the limit as $x$ 
approaches 0 of $x^{-2}$ is infinity, while the same limit taken on $\frac{1}{x}$
does not exist.


the limit of f as x approaches a is infinity, denoted

$$\lim_{x \to a} f(x) = \infty$$
means that for all $\varepsilon >0 \text{there exists}  \delta >0 \text{such that}  f(x) > \varepsilon \text{whenever}  |x - a| < \delta$.

These ideas can be combined in a natural way to produce definitions for different combinations, such as

$$f(x)=  \lim_{x \to \infty} f(x) = \infty, \lim_{x \to a^+}f(x) = -\infty$$.
For example:

$$ \lim_{x \to 0^+} \ln x = -\infty$$
