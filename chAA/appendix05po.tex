%!TEX root =  ../main.tex

\section{Properties and Operators}
\subsection{Closure}
Suppose you attempted to help a young child with some mathematic problems.  
Suppose further this child knew next
to nothing about fractions yet.  Perhaps they understand the notion of debt --- 
negative numbers --- but no greater
sets of numbers.  What can you talk to them about?  What must you avoid?  
Consider the following list of procedures you know:

\begin{itemize}
\item Addition
\item Subtraction
\item Multiplication
\item Division
\item Distribution of Multiplication over Addition
\item Squaring
\item Square roots
\end{itemize}

The question before us is one of \gls{closure}.  A set is closed under an operator 
if doing the operation on any member of
the set produces another member of the set.  For example, the integers are closed under addition, subtraction, and multiplication, but not division.

\begin{example}
\exProblem
Are the Real numbers closed under square rooting?

\exSolution
No.  Disproof by counterexample: $\sqrt{-4} = 4i$
\end{example}

\subsection{Identity and Invertibility}
Over a given set, most operators have an \textbf{identity element} that leaves 
them unchanged.  For example, the
identity of addition is 0, because adding zero does not change a number.  
Similarly, multiplication has
the identity element of 1, because multiplying by one leaves a number unchanged.

\begin{example}
\exProblem
What is the identity of set union?

\exSolution
$\varnothing$ when joined to any set will not change the set in any way.
\end{example}

It is possible to have more than one identity for a given operator over a particular set.  
One effect of having an
identity element is that it becomes possible to define ``undoing'' a particular operator.  For instance,
multiplicative inverse of every number except 0 is one, which shows that multiplication by $m$ can be
undo by multiplying by $\frac{1}{m}$.  The additive inverse of some number $x$ is $-x$, because together
they add to the additive identity 0.  An operation may be invertible without being commutative (see below)

\subsection{Associativity}
Suppose we have one operator and three operands.  In what order do we proceed?  Many times,
it does not matter.  Addition and multiplication are associative over most numbers, so $(3+2)+1$ 
is the same as $3+(2+1)$ and $(2\cdot 3) \cdot 4$ is the same as $2\cdot(3\cdot 4)$.  Division is
not associative, and neither are powers.


\reminder{\lefthand}{A cute way to remember this property is to think of friends associating together.  If three girls are friend --- call them Jill, Sandra, and Bree --- then it doesn't matter which we we draw the association.  Jill and Sandra are friends, and together they're friends with Bree.  That is the same as Sandra and Bree being friends, and  together being friends with Jill.  $(J \cup S) \cup B = J \cup (S \cup B)$.}


\begin{example}
\exProblem
Is rock-paper-scissors\footnote{Or even better, rock-paper-scissors-lizard-Spock!} associative?

\exSolution
No.  Disproof by counterexample:\\
(rock vs. paper) vs. scissors $\rightarrow$ paper vs. scissors $\rightarrow$ scissors\\
rock vs. (paper vs. scissors) $\rightarrow$ rock vs scissors $\rightarrow$ rock
\end{example}

\subsection{Commutativity}
Many operators do not care in which order the operands come.  Addition and multiplication are the same
forwards and backwards!  But subtraction and division yield very different results when done in the
reverse.  This property is called commutativity.

\subsection{Distributivity}
A very complicated property, yet a very well known one, is the distributive property.  This property
requires \emph{two} operators to be performed on a set.  For example, multiplication \emph{is}
distributive over addition.  e.g. $2(3+4) = 2\cdot7 = 2\cdot3 + 2\cdot4 = 6 + 8 = 14$.

\begin{example}
\exProblem
Is division distributive over addition?  Are powers distributive over multiplication?

\exSolution
No and yes.

No.  $2\div(3+4) \ne 2\div3 + 2\div4$.
Yes.  $(3\cdot4)^2 = 12^2 = 3^2\cdot4^2 = 9\cdot16 = 144$
\end{example}

Exercise: clock-math closure




