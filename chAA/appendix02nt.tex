%!TEX root =  ../main.tex

\section{Kinds of Numbers}\index{Kinds}\label{sec:AAKinds}

\objective{Explain the Reals and how to build them.}


Numbers are abstract entities which represent quantities, measurements, or arrangements.  
There are various kinds of numbers, and thinking about the sizes and definitions of these
kinds is a surprisingly vast and complicated field.  Perhaps this enterprise is as old as 
mathematics, beginning when someone thought, ``There is no number so large that
one cannot be added to it''. 

It is assumed that all human cultures with counting began with the \gls{natural numbers}, 
that is $\{1, 2, 3, ...  \}$.  The symbols for ``natural numbers'' is $\mathbb{N}$.  For many areas of life,
analyses need proceed no further.  But, various cultures at various times have needed to 
distinguish zero from an actual lack of an answer.  More formally, $0 \ne \varnothing$.
, and it plays an important role in mathematics as a number.  We call
0, 1, 2, 3, ... the whole numbers, or $\mathbb{W}$. 


\begin{example}
\exProblem
If $A \in \mathbb{N}$ and $B \in \mathbb{W}$, does $(A \cap \{0\} ) \cap B = \varnothing$?

\exSolution
$A \cap \{0\}$ is asking that the natural numbers and zero have in common, which is nothing.
$B$ and nothing have nothing in common.  Therefore, yes, $(A \cap \{0\} ) \cap B = \varnothing$
\end{example}


Many mathematics textbooks discuss negative numbers next, but we will save that discussion
for the next section.  Suffice it to say, whole numbers and their negative counterparts make up
the \gls{integers} ($\mathbb{Z}$).

Most cultures have a need to portion out large things, and these portions are typically thought of
as proportional.  That is, we imagine a ratio of possessed pieces to overall divisions.  The most
common fraction is $\frac{1}{2}$, which means division into two pieces and existence of only one
of those pieces.  Quite plainly, we can call these numbers \gls{rational numbers} because they 
can be expressed as ratios.  The mathematical symbol is $\mathbb{Q}$, because the Latinate
word for a ratio is `quotient'.

\begin{derivation}{Positive Rational Number}
A positive rational number may be written as $\frac{a}{b}$, where $\{a | a \in \mathbb{W}\}$ and
$\{b | b \in \mathbb{N}\}$.  That is, the denominator may not be zero.
\end{derivation}


%${1+\cfrac{1}{2+\cfrac{1}{3+\cfrac{1}{4+\cfrac{1}{5+\cfrac{1}{6+\ddots}}}}}}$

\begin{derivation}{Decimal to Fraction}
Any repeating or terminating decimal may be written as a rational number and visa versa.
\end{derivation}


\begin{example}
\exProblem
Convert $1.\overline{126}$ into a ratio.

\exSolution
Let $x = 1.\overline{126}$.  Because there are three repeating digits, we multiply by $10^3$.

\begin{align*}
1000x & = & 1126.\overline{126}\\
- x & = & -1.\overline{126}\\
\hline
999x & = & 1125.000
\end{align*}
$$
\therefore x = \frac{1125}{999} = \frac{125}{111}
$$
\end{example}

\reminder{\lefthand}{You calculator can help save time by reducing fractions for you.   
\Touche[style=function,principal={MATH},]
($\blacktriangleright$FRAC) is a powerful tool.
}




\subsection{Irrationals}
Some numbers may not be expressed as fractions.  They are called \gls{irrational numbers} and 
their symbol is $\mathbb{I}$.  Some irrational numbers are nevertheless solutions to algebra problems, 
such as $\sqrt{2}$ or $10^{2.3}$.  These are called the \gls{algebraic} numbers and are symbolized
with a $\mathbb{A}$.  Others may be described, but are not the result of arithmetic, numbers such 
as $\pi$ or $e$.  Such values are called \gls{transcendental} numbers, and they are proven to be the
vast majority of numbers.




\ExSection[Exercises]
Convert repeating decimals to fractions
\begin{exercises}{sec:AAKinds}
\prob{} Convert $0.11111...$ into a fraction
\begin{answer}
$\frac{1}{9}$
\end{answer}

\prob{}Convert $1.212121212...$ into a fraction

\prob{}Convert $0.123123123123...$ into a fraction

\prob{}Convert $0.123456789$ into a fraction

\prob{}Convert $0.12345678901234567890123...$ into a fraction

\end{exercises}

\begin{exercises}{sec:AAKinds}
\prob{} Put in order without a calculator $\sqrt{10}, \pi, \frac{22}{7}, 3.14$
\end{exercises}
