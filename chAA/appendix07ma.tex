%!TEX root =  ../main.tex

\section{Matrices}   

\paragraph{Matrix}
A \textbf{matrix} (plural matrices) is a rectangular array of numbers, symbols, or expressions, arranged in meaningful rows and columns.  The individual items in a matrix are called its elements or entries.


\paragraph{Size, Entries}
A matrix's size is described by the number of rows, by the number of columns.  If a matrix is given
a name, an entry may be referred to by a subscript of row and column on that letter.  For example,
on matrix \textbf{[A]}, one might refer to the entry in the second row and third column as
\textbf{A}$_{2,3}$

\begin{example}
\exProblem
Given that A is 1234 what is $A_{1,4}$?

\exSolution
4
\end{example}

\subsection{Addition and Scalars}
Matrices may be added if and only if they are the exact same size.  A matrix maybe multiplied by
a number (called a \gls{scalar}), which is simply multiplied against every element in the matrix.
Two matrices are added just by adding the for corresponding entries, i.e. $A_{i,j}+B_{i,j}$ 
produces the new entry at $i,j$.

\begin{example}
\exProblem
If A is 1234, what is 2A?

\exSolution
2468?
\end{example}

\begin{example}
\exProblem
If A is 1234 and B is 0102 what is A+B?

\exSolution
1336
\end{example}

\subsection{Matrix Multiplication}
The product of two matrices is the coming together of rows of the first, with columns of the second.  For example,
to compute the top left entry in the product of two matrices, one multiplies each entry in the first row of the first matrix,
against the corresponding entry in the first column of the second matrix.  (See below for a helpful visual.)  Naturally,
this means that the rows and columns must match up.




\subsection{Square Matrices}
identity, inverses, determinants 
\subsection{Gaussian Elimination}
augmented matrices



