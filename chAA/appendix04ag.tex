%!TEX root =  ../main.tex


\section{Analytic Geometry}\label{sec:AG}


\objective{Calculate distances and midpoints on the Cartesian coordinate plane for order pairs from
functions and relations.}


\subsection{Cartesian Plane}
Descartes did many strange things in philosophy, but he was a real boon for mathematics.  The idea
of graphing values of one variable ($x$) to the left and right, while simultaneously graphing another
variable ($y$) on the up and down is credited to him.  This is called the \gls{Cartesian plane} 
or ``Rectangular Coordinates''.

\subsubsection{Quadrants}
It is customary to split the coordinate system into four quadrants, numbered with Roman numerals.
Quadrant I is where $x$ and $y$ are both positive, to the upper right.  Quadrant II is where $x$ is negative
but $y$ is positive, to the upper left.  Quadrant III is where both are negative, to the lower left.  Quadrant
IV is where $x$ is positive and $y$ is negative, to the lower right.

\subsection{Triangles}
Just as we might find the middle between two numbers on the number line, 
so too we can find a point midway 
between two other points, simply by taking the average of their $x$s and the average of their $y$s.

\begin{derivation}{Midpoint}
Given two points $(x_1,y_1)$ and $(x_2,y_2)$ the arithmetic mean of the $x$'s is $x$ of the midpoint,
and the same for $y$.

Midpoint $(x,y)=(\frac{x_1+x_2}{2},\frac{y_1+y_2}{2})$
\end{derivation}

If we draw a right-triangle with its hypotenuse being the shortest path between two points, and its legs going
strictly left-to-right and up-to-down, then calculating their distance is simply the Pythagorean Theorem.

\begin{derivation}{Pythagorean Theorem}
The distance from $(x_1,y_1)$ to $(x_2,y_2)$ is the hypotenuse of a right triangle with legs
$|x_2-x_1|$ and $|y_2-y_1|$.

\begin{equation}
d=\sqrt{(x_2-x_1)^2+(y_2-y_1)^2}
\end{equation}
\end{derivation}

\subsection{Functions and Relations}

Chapter 1 proper of this textbook is about functions, but even more basic are relations.  
For number, it will suffice
to say that any equation with $x$s and $y$s in it is a 
relation\footnote{Note that these need not be ``nice''.  For example,
$x^2y^2=4$ produces an infinite, four-pointed star, but must be put into the TI-8* as \texttt{Y1=sqrt(4/x\^2)} and 
\texttt{Y2=-sqrt(4/x\^2)}.}.  

like 
\begin{derivation}{Relation}
A relation between two sets is a collection of ordered pairs containing one object from each set. 
If the object $x$ is from the first set and the object $y$ is from the second set, then the objects are said to be related if the ordered pair $(x,y)$ is in the relation.
\end{derivation}

\ExSection[Exercises]
\begin{exercises}{sec:AG}
\index{Absolute Value!of y}
\prob{}Graph the relation $x=|y|$.  How could we put this into our TI-8*?



\prob{}Use the Pythagorean theorem to find 12 lattice points 5 units from the origin.

\prob{}Find the relation describing all the ordered pairs 5 units from the origin.

\prob{}What are the points 5 units away from 4,2 with a y value of 10?

\prob{}What are the points 6 units away from -2,-1 with a x value of -10

\prob{}What are the points root 10 units away from 0,2 and root 10 units away from 2,0?
\end{exercises}
