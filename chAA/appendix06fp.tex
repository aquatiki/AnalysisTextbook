%!TEX root =  ../main.tex

\marginfig[1in]{\chapdir/pics/Factor_Tree_of_42.png}{Prime decomposition}
\section{Factoring and Primes}\label{sec:primes}
Humans seeks to understand the fundamental nature of numbers.  We have learned
many things, but the underlying, deep structure is not fully known.  We believe will know
a great deal more if we can understand the distribution of Prime Numbers.


\begin{derivation}{Prime Number}
A natural number greater than 1 that has no positive divisors other than 1 and itself.
\end{derivation}


Every number is either prime or can be written as the product of primes.  Primes turn up
in many unexpected places, such as the infinite sums of the Zeta function, or the infinite
list of all possible Pythagorean Triples.  


\begin{derivation}{Fundamental Theorem of Arithmetic}
Every integer greater than 1 either is prime itself or is the unique product of prime numbers.
\end{derivation}


Factorizing numbers show the algebraic nature of all natural numbers.  As we seek 
to simplify algebraic expression through factorization, we find many helpful patterns which 
are important to know both forwards (distribution of multiplication over addition) and 
backwards.

\subsection{Patterns of Distribution}
\marginfig[-0.5in]{\chapdir/pics/Kvadreringsregeln.png}{Squaring a binomial}
The square of a binomial always follows a pattern, which can be seen algebraically or
via algebra blocks.  A binomial multiplied by its conjugate produces the difference of squares.
When the first element of two binomial is the same, the result will have like terms that combine
as a sum.

\begin{equation}
(a + b)^2 = a^2 + 2ab + b^2
\end{equation}
\begin{equation}
(a + b) (a - b) = a^2 - b^2
\end{equation}
\begin{equation}
(x + a) (x + b) = x^2 + (a+b)x + ab
\end{equation}

\subsection{Factoring Quadratics}
\marginfig[0.5in]{\chapdir/pics/alge_tiles_ex_5.JPG}{Algebra tiles can be a helpful way to visualize factoring simple 	polynomials.}

When factoring, the first check should always be for a common factor to take out.  Then,
the above techniques seen in distribution work in reverse for factoring.  If a quadratic expression
is of the form $ax^2+bx+c$, there are two very different techniques, depending upon whether
$a=1$ of not.  

\begin{derivation}{Simple Squared Term}
After checking for a GCF,\\
Perfect Square Binomial: $a^2 + 2ab + b^2 = (a+b)^2$\\
Difference of Squares: $a^2 - b^2 = (a+b)(a-b)$\\
``Adds to, Multiplies to'': $x^2 + (a+b)x + ab = (x + a)(x + b)$\\
\end{derivation}

\subsubsection{Split the middle}
When $a\ne1$, more elaborate techniques are called for.  Typically, working with $a<0$ is much
more difficult, so factoring out a negative is helpful.  Next, splitting the linear term in two may
yield results.  Looking at $ax^2+bx+c$, seek to find factors of $a\cdot c$ which add to $b$.  
Factor the terms, two at a time (called ``factoring by grouping''.  In large polynomials, groups of
arbitrary size are possible.)  You will know you have succeed when you have two different
terms multiplied agains the same term in parentheses.

\begin{example}
\exProblem
Factor $15x^2-11x-14$.

\exSolution
Begin by multiplying $15\cdot(-14)=-210$.  We are looking for factors of $-210$ which are 11 apart.
Only 10 and and $-21$ fit the bill.

$15x^2-11x-14$\\
$15x^2+10x-21x-14$\\
$5x(3x+2)-7(3x+2)$\\
$(5x-7)(3x+2)$
\end{example}

Sometimes, none of the above techniques yields any results.  There are two equally valid
moves at that point, but one of which is made from the other.

\subsection{Complete the Square and the Quadratic Formula}
First, if $a\ne1$, factor it out of the terms with an unknown.  
Second, use the definition of a Perfect Square Binomial
to discover what constant would be needed to ``complete the square''.  Insert this term and its
additive inverse (which is like adding zero) and factor to a square.

\begin{example}
\exProblem
Rewrite by completing the square: $3x^2 + 12x + 7$

\exSolution
$3(x^2+4x) + 7$\\
$3(x^2+4x+4-4) + 7$\\
$3[(x+2)^2-4] + 7$\\
$3(x+2)^2-5$
\end{example}

It's derivation is an exercise, but it is quite possible to complete the square on $ax^2+bx+c=0$ 
and produce the famous Quadratic Formula:

\begin{derivation}{Quadratic Formula}
For any quadratic $y=ax^2+bx+c$, the values of $x$ where $y=0$ are the two solutions
$x=\dfrac{-b\pm\sqrt{b^2-4ac}}{2a}$
\end{derivation}



